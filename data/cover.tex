
%%% Local Variables:
%%% mode: latex
%%% TeX-master: t
%%% End:

% 中国海洋大学研究生学位论文封面
% 参考:中国海洋大学研究生学位论文书写格式20130307.doc

% 为避免出现错误,下面保留[清华大学学位论文模板原有定义无需修改],
% 请直接跳到后面[中国海洋大学学位论文模板部分请根据自己情况修改]。

%%%%%%%%%%%%%%%%%%%%%%[清华大学学位论文模板原有定义无需修改]%%%%%%%%%%%%%%%%%%%%%%%
\secretlevel{绝密} \secretyear{2100}

\ctitle{清华大学学位论文 \LaTeX\ 模板\\使用示例文档}
% 根据自己的情况选,不用这样复杂
\makeatletter
\ifthu@bachelor\relax\else
  \ifthu@doctor
    \cdegree{工学博士}
  \else
    \ifthu@master
      \cdegree{工学硕士}
    \fi
  \fi
\fi
\makeatother


\cdepartment[计算机]{计算机科学与技术系}
\cmajor{计算机科学与技术}
\cauthor{薛瑞尼} 
\csupervisor{郑纬民教授}
% 如果没有副指导老师或者联合指导老师,把下面两行相应的删除即可。
\cassosupervisor{陈文光教授}
\ccosupervisor{某某某教授}
% 日期自动生成,如果你要自己写就改这个cdate
%\cdate{\CJKdigits{\the\year}年\CJKnumber{\the\month}月}

% 博士后部分
% \cfirstdiscipline{计算机科学与技术}
% \cseconddiscipline{系统结构}
% \postdoctordate{2009年7月——2011年7月}

\etitle{An Introduction to \LaTeX{} Thesis Template of Tsinghua University} 
% 这块比较复杂,需要分情况讨论:
% 1. 学术型硕士
%    \edegree:必须为Master of Arts或Master of Science(注意大小写)
%              “哲学、文学、历史学、法学、教育学、艺术学门类,公共管理学科
%               填写Master of Arts,其它填写Master of Science”
%    \emajor:“获得一级学科授权的学科填写一级学科名称,其它填写二级学科名称”
% 2. 专业型硕士
%    \edegree:“填写专业学位英文名称全称”
%    \emajor:“工程硕士填写工程领域,其它专业学位不填写此项”
% 3. 学术型博士
%    \edegree:Doctor of Philosophy(注意大小写)
%    \emajor:“获得一级学科授权的学科填写一级学科名称,其它填写二级学科名称”
% 4. 专业型博士
%    \edegree:“填写专业学位英文名称全称”
%    \emajor:不填写此项
\edegree{Doctor of Engineering} 
\emajor{Computer Science and Technology} 
\eauthor{Xue Ruini} 
\esupervisor{Professor Zheng Weimin} 
\eassosupervisor{Chen Wenguang} 
% 这个日期也会自动生成,你要改么?
% \edate{December, 2005}

% 定义中英文摘要和关键字
\begin{cabstract}
  论文的摘要是对论文研究内容和成果的高度概括。摘要应对论文所研究的问题及其研究目
  的进行描述,对研究方法和过程进行简单介绍,对研究成果和所得结论进行概括。摘要应
  具有独立性和自明性,其内容应包含与论文全文同等量的主要信息。使读者即使不阅读全
  文,通过摘要就能了解论文的总体内容和主要成果。

  论文摘要的书写应力求精确、简明。切忌写成对论文书写内容进行提要的形式,尤其要避
  免“第 1 章……;第 2 章……;……”这种或类似的陈述方式。

  本文介绍清华大学论文模板 \thuthesis{} 的使用方法。本模板符合学校的本科、硕士、
  博士论文格式要求。

  本文的创新点主要有:
  \begin{itemize}
    \item 用例子来解释模板的使用方法;
    \item 用废话来填充无关紧要的部分;
    \item 一边学习摸索一边编写新代码。
  \end{itemize}

  关键词是为了文献标引工作、用以表示全文主要内容信息的单词或术语。关键词不超过 5
  个,每个关键词中间用分号分隔。(模板作者注:关键词分隔符不用考虑,模板会自动处
  理。英文关键词同理。)
\end{cabstract}

\ckeywords{\TeX, \LaTeX, CJK, 模板, 论文}

\begin{eabstract} 
   An abstract of a dissertation is a summary and extraction of research work
   and contributions. Included in an abstract should be description of research
   topic and research objective, brief introduction to methodology and research
   process, and summarization of conclusion and contributions of the
   research. An abstract should be characterized by independence and clarity and
   carry identical information with the dissertation. It should be such that the
   general idea and major contributions of the dissertation are conveyed without
   reading the dissertation. 

   An abstract should be concise and to the point. It is a misunderstanding to
   make an abstract an outline of the dissertation and words ``the first
   chapter'', ``the second chapter'' and the like should be avoided in the
   abstract.

   Key words are terms used in a dissertation for indexing, reflecting core
   information of the dissertation. An abstract may contain a maximum of 5 key
   words, with semi-colons used in between to separate one another.
\end{eabstract}

\ekeywords{\TeX, \LaTeX, CJK, template, thesis}
%%%%%%%%%%%%%%%%%%%%%%%%%%%%%%%%%%%%%%%%%%%%%%%%%%%%%%%%%%%%%%%%%%%%%%%%%%%%%%%%

%%%%%%%%%%%%%%%%%%[中国海洋大学学位论文模板部分请根据自己情况修改]%%%%%%%%%%%%%%%%%%%
% 中国海洋大学研究生学位论文封面
% 必须填写的内容包括(其他最好不要修改):
%   分类号、密级、UDC
%   论文中文题目、作者中文姓名
%   论文答辩时间
%   封面感谢语
%   论文英文题目
%   中文摘要、中文关键词
%   英文摘要、英文关键词
%
%%%%%[自定义]%%%%%
\newcommand{\fenleihao}{}%分类号
\newcommand{\miji}{}%密级 
                    % 绝密$\bigstar$20年 
                    % 机密$\bigstar$10年
                    % 秘密$\bigstar$5年
\newcommand{\UDC}{}%UDC
\newcommand{\oucctitle}{基于图像处理的垂测电离图E区描迹自动判读方法}%论文中文题目
\ctitle{基于图像处理的垂测电离图E区描迹自动判读方法}%必须修改因为页眉中用到
\cauthor{孙晓庆}%可以选择修改因为仅在 pdf 文档信息中用到
\cdegree{工学硕士}%可以选择修改因为仅在 pdf 文档信息中用到
\ckeywords{\TeX, \LaTeX, CJK, 模板, 论文}%可以选择修改因为仅在 pdf 文档信息中用到
\newcommand{\ouccauthor}{孙晓庆}%作者中文姓名
%\newcommand{\ouccauthor}{***}%外审时用到
%\newcommand{\ouccsupervisor}{姬光荣教授}%作者导师中文姓名
%\newcommand{\ouccdegree}{博\hspace{1em}士}%作者申请学位级别
%\newcommand{\ouccmajor}{海洋信息探测与处理}%作者专业名称
%\newcommand{\ouccdateday}{\CJKdigits{\the\year}年\CJKnumber{\the\month}月\CJKnumber{\the\day}日}
%\newcommand{\ouccdate}{\CJKdigits{\the\year}年\CJKnumber{\the\month}月}
\newcommand{\oucdatedefense}{           }%论文答辩时间
%\newcommand{\oucdatedegree}{2009年6月}%学位授予时间
\newcommand{\oucgratitude}{谨以此论文献给我的导师和亲人!}%封面感谢语
\newcommand{\oucetitle}{Automatic traces scaling of E region from ionograms based on image processing}%论文英文题目
%\newcommand{\ouceauthor}{Haiyong Zheng}%作者英文姓名
\newcommand{\oucthesis}{\textsc{OUCThesis}}
%%%%%默认自定义命令%%%%%
% 空下划线定义
\newcommand{\oucblankunderline}[1]{\rule[-2pt]{#1}{.7pt}}
\newcommand{\oucunderline}[2]{\underline{\hskip #1 #2 \hskip#1}}

% 论文封面第一页
%%不需要改动%%
\vspace*{5cm}
{\xiaoer\heiti\oucgratitude

\begin{flushright}
---\hspace*{-2mm}---\hspace*{-2mm}---\hspace*{-2mm}---\hspace*{-2mm}---\hspace*{-2mm}---\hspace*{-2mm}---\hspace*{-2mm}---\hspace*{-2mm}---\hspace*{-2mm}---~\ouccauthor
\end{flushright}
}

%\begin{comment}

\newpage 
%\mbox{} 
%\newpage

% 论文封面第二页
%%不需要改动%%
\vspace*{1cm}
\begin{center}
  {\xiaoer\heiti\oucctitle}
\end{center}
\vspace{10.7cm}
{\normalsize\songti
\begin{flushright}
{\renewcommand{\arraystretch}{1.3}
  \begin{tabular}{r@{}l}
    学位论文答辩日期:~ & \oucunderline{2.5cm}{\oucdatedefense} \\
    指导教师签字:~ & \oucblankunderline{5cm} \\
    答辩委员会成员签字:~ & \oucblankunderline{5cm} \\
    ~ & \oucblankunderline{5cm} \\
    ~ & \oucblankunderline{5cm} \\
    ~ & \oucblankunderline{5cm} \\
    ~ & \oucblankunderline{5cm} \\
    ~ & \oucblankunderline{5cm} \\
    ~ & \oucblankunderline{5cm} \\
  \end{tabular}
}
\end{flushright}
}

\newpage 
%\mbox{} 
%\newpage

% 论文封面第三页
%%不需要改动%%
\vspace*{1cm}
\begin{center}
  {\xiaosan\heiti 独\hspace{1em}创\hspace{1em}声\hspace{1em}明}
\end{center}
\par{\normalsize\songti\parindent2em
本人声明所呈交的学位论文是本人在导师指导下进行的研究工作及取得的研究成果。据我所知,除了文中特别加以标注和致谢的地方外,论文中不包含其他人已经发表或撰写过的研究成果,也不包含未获得~\oucblankunderline{7cm}(注:如没有其他需要特别声明的,本栏可空)或其他教育机构的学位或证书使用过的材料。与我一同工作的同志对本研究所做的任何贡献均已在论文中作了明确的说明并表示谢意。
}
\vskip1.5cm
\begin{flushright}{\normalsize\songti
  学位论文作者签名:\hskip2cm 签字日期:\hskip1cm 年 \hskip0.7cm 月\hskip0.7cm 日}
\end{flushright}
\vskip.5cm
{\setlength{\unitlength}{0.1\textwidth}
  \begin{picture}(10, 0.1)
    \multiput(0,0)(0.2, 0){50}{\rule{0.15\unitlength}{.5pt}}
  \end{picture}}
\vskip1cm
\begin{center}
  {\xiaosan\heiti 学位论文版权使用授权书}
\end{center}
\par{\normalsize\songti\parindent2em
本学位论文作者完全了解学校有关保留、使用学位论文的规定,并同意以下事项:
\begin{enumerate}
\item 学校有权保留并向国家有关部门或机构送交论文的复印件和磁盘,允许论文被查阅和借阅。
\item 学校可以将学位论文的全部或部分内容编入有关数据库进行检索,可以采用影印、缩印或扫描等复制手段保存、汇编学位论文。同时授权清华大学“中国学术期刊(光盘版)电子杂志社”用于出版和编入CNKI《中国知识资源总库》,授权中国科学技术信息研究所将本学位论文收录到《中国学位论文全文数据库》。
\end{enumerate}
(保密的学位论文在解密后适用本授权书)
}
\vskip1.5cm
{\parindent0pt\normalsize\songti
学位论文作者签名:\hskip4.2cm\relax%
导师签字:\relax\hspace*{1.2cm}\\
签字日期:\hskip1cm 年\hskip0.7cm 月\hskip0.7cm 日\relax\hfill%
签字日期:\hskip1cm 年\hskip0.7cm 月\hskip0.7cm 日\relax\hspace*{1.2cm}}

%\end{comment}

\newpage 
%\mbox{} 
%\newpage

\pagestyle{plain}
\clearpage\pagenumbering{roman}

% 中文摘要
%%[需要填写:中文摘要、中文关键词]%%
\begin{center}
  {\sanhao[1.5]\heiti\oucctitle\\\vskip7pt 摘\hspace{1em}要}
\end{center}
{\normalsize\songti

  \indent
作为“太阳活动的反光镜”和“大气扰动的放大镜”,电离层研究对于学术研究有重要的意义,同时也有非常大的应用价值。当今电离层探测技术在不断的提高,目前全世界观测记录电离图的数字测高仪的数量也越来越多,使每年至少观测记录400万张以上的电离图,并且数量仍将持续增加。目前电离图的判读还是靠专业人员进行度量,这种传统的人工度量方法费时、费力、实时性差,且存在度量者主观因素造成的度量差异,已逐渐不能满足实际要求。随着人类对空间科学的研究不断深入,对电离层的长期、实时监测及预报要求的日趋迫切,因此垂测电离图的自动判读研究具有重要的意义。

本文总结了电离图人工度量经验及电离图E区描迹类型与特点,基于图像处理技术提出了垂测电离图E区描迹自动判断算法。主要工作包括:

\begin{enumerate}
    \item系统介绍了电离层结构及电离层垂直探测相关知识,给出了需要自动度量的电离图参数定义,并对国内外电离图自动判读算法进行总结分析,为本文方法的研究奠定了重要的基础。根据对电离层基础内容的分析,明确了本课题的工作目标及意义,并提出了本文判读方法的整体流程。算法主要分为电离图E区F区分割、E区描迹类型识别与参数度量两部分。
    \item电离图E区F区分割。本文根据原始电离图上的描迹分布特点对电离图进行了去除垂线噪声和阈值分割预处理。根据垂测电离图E区F区之间存在谷区的特点,利用水平投影积分法对预处理后的二值电离图进行零区查找,并结合电离图人工度量知识、参数经验值及零区特征,判断零区的形成原因,从中选择最优零区的中线作为垂测电离图E区和F区的分割线。
    \item 电离图E区描迹类型识别及参数度量。由于电离图E区描迹没有规律性,种类多并且还存在不同程度的扩散,因此设计E区电离图描迹判读算法中,本文全面地考虑了电离图E区上的11种Es层类型的描迹。E区电离图的自动判读算法主要包括以下步骤:1)根据描迹形成特点,提出了基于密集点查找的描迹自动检测算法,对存在描迹扩散现象和噪声影响的电离图都有很好的效果;2)为了对描迹的扩散程度进行评价,本文提出了扩散密度的概念,并利用扩散密度、区域宽度等特征将描迹区域分为三类;3)根据电离图上描迹区域的类型将电离图分为五类,并分别设计描迹判读算法。对于无描迹电离图直接给出度量结果;对于复杂电离图进行标记由人工进行度量;对于剩余三类电离图,根据描迹特征,结合人工度量经验,运用图像处理技术提出了描迹分割、去除Es层多次反射、描迹特征提取算法。4)根据度量规则,确定描迹类型并获取E区电离图的相关参数。
\end{enumerate}

利用不同地区采集的大量电离图对本文算法进行实验验证,实验结果表明了本文算法的有效性和可行性。

}
\vskip12bp
{\xiaosi\heiti\noindent
关键词:\hskip1em 图像处理;图像分析;垂测电离图;自动判读}

\newpage 
%\mbox{} 
%\newpage

% 英文摘要
%%[需要填写:英文摘要、英文关键词]%%
\begin{center}
  {\sanhao[1.5]\heiti\oucetitle\\\vskip7pt Abstract}
\end{center}
{\normalsize\songti

As the ``mirror of solar activity'' and ``magnifier of atmospheric disturbance'', ionospheric research has a very important academic significance and socio-economic value. With the development of ionosphere detection technology, the number of the ionosondes which is used to record ionograms is increasing in the whole world, and the annual number of ionograms is more than 400 million. The number of ionograms will continue increasing. But ionospheric characteristic parameters are still mainly obtained by the well-experienced operators, which is not only time-consuming and laborious, but also can not avoid the scaling error caused by the subjective factors. So it can not meet the practical needs. With a better understanding space science, the requirement to long and real time monitoring  and forecast of ionogram becomes urgent, so the automatic scaling of the ionogram is significant.

Through summarizing the artificial experiences and feature  of E Region trace, a new method for scaling the E region traces in ionogram based on the image processing is proposed. The work mainly contains:


\begin{enumerate}
\item  The background information of ionospheric sounding are introduced, which includes the definition of the ionospheric parameters in ionogram and the summary of research status, laying foundation for the research on scaling algorithm. According to the analysis of ionospheric basis knowledge, we confirm the research content and research sense of this subject and propose the overall flow of algorithm. The proposed algorithm includes two steps: segmenting E region image and scaling E region ionogram. 
\item Segmenting E region image  from ionogram. We do the preprocessing operations includes removing the vertical noise and threshold segmentation based on the distribution characters of the trace on the ionogram.  Because there is the blank valley between the E region and F region, we use the horizontal projection method to find the zero area in pre-processed binary image,and find the cause of the zero area in ionogram base on the artificial measurement knowledge, experimental reference data and zero area's features. Then we choose central line of the optimal zero area as the division line of E region and F region. 
\item  Scaling the E region ionogram. The traces in the E region have diffusing phenomenon and many types of trace. In the process of designing E region automatic scaling algorithm, we consider 11 Es types of trace. The E region automatic scaling algorithm mainly contains: 1)Trace detection algorithm based on dense point searching is proposed for effective trace extraction of E region from the ionograms with spread phenomenon and noise. 2) In order to evaluate the diffusion degree of the trace, we put forward the concept of diffusion density.  The traces are divided into three classes based on some features, such as diffusion density, width of the area. 3) We divided the ionogram into five classes.  If there are no trace on the ionogram, we can get the scale result directly;  The complex ionogram is marked automatically, and it will be scale artificially; For other three classes ionograms, the algorithms of trace segmentation, removing multiple reflections, feature extraction, trace identification are designed based on the image processing and artificial measurement experience. 4)Finally we obtain the related parameters of E region according to the scaling rules.

\end{enumerate}

In this thesis, we have performed a series of experiments to confirm the effectiveness and feasibility of the method for scaling the E region traces in ionogram.
}
\vskip12bp
{\xiaosi\heiti\noindent 
\textbf{Keywords:\enskip Image processing; Image analysis; Ionogram; automatic scaling}}
%%%%%%%%%%%%%%%%%%%%%%%%%%%%%%%%%%%%%%%%%%%%%%%%%%%%%%%%%%%%%%%%%%%%%%%%%%%%%%%%
%\newpage 
%\mbox{} 
%\newpage
