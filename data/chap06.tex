\chapter{总结与展望}
\label{cha6}
本章主要对本文提出的垂测电离图E区描迹自动判读方法进行总结与展望。
%=============================================================================================================
\section{总结}
\label{6_1}

电离层探测结果的自动判读对电离层结构状态的研究提供大量的数据,同时可以克服人工度量费时、费力、度量结果存在主观差异等问题,而且可以为电离层相关研究提供实时的电离层监测数据。本文通过对大量电离图的人工度量经验进行总结,基于图像处理、图像分析方法,对垂测电离图E区描迹自动度量方法进行了研究,主要工作内容如下:
\begin{enumerate}
\item 电离层的相关知识及E区电离图的人工度量经验总结。本文对电离层形成机制、电离层分层结构和探测原理进行总结,为自动判读算法的提出奠定了基础;对垂测电离图参数度量标准进行介绍,并总结电离图E区描迹人工度量步骤,明确了研究意义并确定了电离图E区描迹自动度量流程。电离图的自动度量算法主要包括:电离图E区F区分割和电离图E区描迹自动判读。
\item电离图E区F区分割算法研究。将电离层垂直探测数据转换为垂测电离图,观察电离图像素灰度值分布特点,去除电离图上的噪声。利用水平投影积分法对去噪后的二值电离图进行电离图零区查找,提取零区特征结合电离图参数经验值,确定最优零区的中线作为E区F区的分割线。
\item电离图E区描迹自动判断算法研究。首先,本文的基于密集点查找的描迹自动检测算法,可以有效地克服传统的电离图预处理方法造成的描迹信息丢失现象;同时算法能够为具有扩散现象的描迹找到闭合的边界,方便于接下来的描迹提取工作;其次,本文提出了扩散密度的概念,对于检测到的描迹区域进行扩散程度评价,并结合其他特征对描迹的区域进行分类;再次,根据描迹区域的类型对电离图进行分类,排除复杂电离图和无描迹电离图;然后,根据描迹的形态特征对描迹进行分割、去除多次反射描迹、特征提取、类型识别与度量。在本文的算法中,创新性地将人工识别Es层描迹的经验转换为计算机量化电离图特征;同时针对E区描迹的多样性,设计了不同的描迹特征提取算法。
\item经过大量实验验证,本文算法对不同地区和不同时间段采集的电离图都有很好的适应性与发展前景,并取得较高精度。
\end{enumerate}

%=============================================================================================================
\section{展望}
\label{6_2}
电离层的形成机制和影响因素的复杂性导致垂直探测电离图的多样性与复杂性,对于特殊的频高图中,不同的度量者也会有不同的判读结果。通过实验验证,本文的电离图E区描迹自动判读算法是可行的,但在实际应用中,我们可以对很多方面进行完善:
\begin{enumerate}
\item本文的电离图E区描迹自动判读算法是基于人工度量方法提出的,目前对于人工度量流程的细化程度不够。比如在垂测电离图上的描迹不完整时,人工度量结果一般采用添加符号进行说明,但在本文算法中,对自动度量结果的细化程度不够,我们还需要对识别和度量进一步的细化(如添加特定符号来说明电离图的特殊情况等)。
\item本文所提出的垂测电离图E区描迹自动判读算法仅依靠人工度量经验,根据描迹的位置、形状、扩散状况等特征来判断描迹类型,而没有将电离层的物理模型和物理意义,融入到本文的算法中。在电离层的实际研究中,存在很多成熟的电离层参考模型,比如IRI模型。我们可以将现有的电离层参考模型与基于描迹特征的电离图识别算法相结合,来提高算法的准确率。
\item电离图E区描迹的种类多样性,同时同一类型描迹的大小、形状也都不尽相同,通过观察有限张电离图总结出的人工度量经验具有一定的局限性,并不能适用于所有的电离图。我们可以运用模式识别知识,将大量含有人工度量结果的电离图作为训练集,训练分类器,进而对未来探测的电离图进行类型识别与参数度量。
\item随着人工智能和云计算的发展,通过对大量电离图数据进行深度学习,形成分类器,依托云计算平台资源对大量电离图进行自动获取电离图参数,这样的方法将更加高效、智能。同时对于大量电离图的自动度量结果也可利用大数据技术进行分析。
\end{enumerate}


