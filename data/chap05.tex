%%% Local Variables: 
%%% mode: latex
%%% TeX-master: t
%%% End: 

\chapter{实验分析}
\label{cha5}
为了验证本文算法的可行性,本章主要介绍运用本文算法对我国不同地区采集的电离图进行实验,并对实验结果分析。
%=============================================================================================================
\section{实验结果}
\label{5_1}
本文采用测试数据集是由北京(2013年6月)、广州(2013年9月)、昆明(2011年12月)、满洲里(2014年2月)、新乡(2011年4月)五个地区不同月份采集的3576张电离图构成,其中每个月份的数据是由该月每一天连续整点采集的电离图构成。本文参考了URSI准则中使用的评价方法,对数据集内的电离图使用本文自动度量算法得到的结果进行分析。
        
参数$foEs$、$fbEs$、$foE$选取的误差允许范围为1MHz,即如果自动度量结果与人工度量结果之差的绝对值小于1MHz,就认为这个参数的自动度量结果是可以被接受的;参数${h^{'}E}$和${h^{'}Es}$选取的误差允许范围为10千米,即自动度量结果与人工度量结果之差的绝对值小于10千米,就认为自动度量结果是可以被接受的。表~\ref{表5_1}为自动度量参数$foEs$、$fbEs$、${h^{'}Es}$的可接受率,表~\ref{表5_2}为自动度量参数$foE$、${h^{'}E}$的可接受率。
\begin{table}
\caption{自动度量参数$foEs$、$fbEs$、${h^{'}Es}$的可接受率}
\centering
\begin{tabular}{l|c|c|c|c|c|c}
\toprule
\multirow{3}{*}{参数} & \multicolumn{3}{c|}{$N_{Es}$:1880张} & \multicolumn{2}{c|}{$N_{NoEs}$:1696张} & \multirow{2}{*}{$R_{all\_accepted\_*}$} \\
\cline{2-6}
& \multicolumn{2}{c|}{$N_{det\_Es}$} & \multirow{2}{*}{$N_{f\_det\_Es}$} & \multirow{2}{*}{$N_{undet\_Es}$} & \multirow{2}{*}{$N_{w\_undet\_Es}$} & \\
\cline{2-3}
& $N_{accepted\_*}$ & $N_{unaccepted\_*}$ & & & & \\
\cline{1-7}
$foEs$ & 1480 & 158 & 242 & 1418 & 278 & 81.03\% \\
$fbEs$ & 1360 & 278 & 242 & 1418 & 278 & 77.68\%\\
${h^{'}Es}$ & 1239 & 399 & 242 & 1418 & 278 & 74.29\%\\
\bottomrule
\end{tabular}
\label{表5_1}
\end{table}

由表~\ref{表5_1}可知,在3576张电离图的人工度量结果中,无Es层描迹的电离图总数($N_{NoEs}$)为1696,在1696张电离图的自动度量结果中,自动度量未检测到Es层描迹的电离图总数($N_{undet\_{Es}}$)为1418张。由公式~\ref{式5_1}可知,在人工度量电离图无Es层描迹的情况下,自动度量算法也未检测到Es层描迹的正确率($R_{undetected\_{Es}}$)为83.6\%。

\begin{linenomath}
\begin{align}
R_{undetected\_{Es}}=N_{undet\_{Es}}/N_{NoEs}
\label{式5_1}
\end{align}
\end{linenomath}

由表~\ref{表5_1}可知,在3576张电离图的人工度量结果中,存在Es层描迹的电离图总数($N_{Es}$)为1880张,在1880张电离图的自动度量结果中,未能成功检测到Es层描迹的电离图总数($N_{f\_det\_Es}$)为242张,成功检测到Es层描迹的电离图总数($N_{det\_Es}$)为1638张。由公式~\ref{式5_2}可知,在人工度量结果中电离图存在Es层描迹的情况下,自动度量算法检测Es层描迹的正确率($R_{detected\_Es}$)为87.13\%。   
\begin{linenomath}
\begin{align}
R_{detected\_{Es}}=N_{det\_{Es}}/N_{Es}
\label{式5_2}
\end{align}
\end{linenomath}

由表表~\ref{表5_1}可知,在自动检测到的存在Es层描迹的1638张电离图中,参数${h^{'}Es}$、$foEs$、$fbEs$的自动度量结果可以被接受的电离图总数($N_{accepted\_foEs}$)、$N_{accepted\_fbEs}$、$N_{accepted\_h'Es}$)分别为:1480张、1360张、1239张。根据公式~\ref{式5_3}、~\ref{式5_4}、~\ref{式5_5}可知,在自动度量算法检测到存在Es层描迹的电离图中,Es层参数自动度量结果的可接受率($R_{accepted\_foEs}$、$R_{accepted\_fbEs}$、$R_{accepted\_h'Es}$)分别为:90.35\%、83.03\%、75.64\%。
\begin{linenomath}
\begin{align}
R_{accepted\_foEs}=N_{accepted\_foEs}/N_{det\_Es}
\label{式5_3}
\end{align}
\end{linenomath}
\begin{linenomath}
\begin{align}
R_{accepted\_{fbEs}}=N_{accepted\_fbEs}/N_{det\_{Es}}
\label{式5_4}
\end{align}
\end{linenomath}
\begin{linenomath}
\begin{align}
R_{accepted\_{h'Es}}=N_{accepted\_h'Es}/N_{det\_{Es}}
\label{式5_5}
\end{align}
\end{linenomath}

为了对算法的整体正确率进行评价,考虑数据集内所有电离图的识别情况,可根据公式~\ref{式5_6}、~\ref{式5_7}、~\ref{式5_8}计算Es层参数$foEs$、$fbEs$、${h^{'}Es}$的可接受率($R_{all\_accepted\_foEs}$、$R_{all\_accepted\_fbEs}$、$R_{all\_accepted\_h'Es}$)分别为81.03\%、77.68\%、74.29\%。

\begin{linenomath}
\begin{align}
R_{all\_accepted\_foEs}={(N_{accepted\_foEs}+N_{undet\_Es})}/{(N_{Es}+N_{NoEs})}
\label{式5_6}
\end{align}
\end{linenomath}
\begin{linenomath}
\begin{align}
R_{all\_accepted\_{fbEs}}={(N_{accepted\_fbEs}+N_{undet\_Es})}/{(N_{Es}+N_{NoEs})}
\label{式5_7}
\end{align}
\end{linenomath}
\begin{linenomath}
\begin{align}
R_{all\_accepted\_{h'Es}}={(N_{accepted\_h'Es}+N_{undet\_Es})}/{(N_{Es}+N_{NoEs})}
\label{式5_8}
\end{align}
\end{linenomath}

\begin{table}
\caption{自动度量参数$foE$、${h^{'}E}$的可接受率}
\centering
\begin{tabular}{l|c|c|c|c|c|c}
\toprule
\multirow{3}{*}{参数} & \multicolumn{3}{c|}{$N_{E}$:1337张} & \multicolumn{2}{c|}{$N_{NoE}$:2239张} & \multirow{2}{*}{$R_{all\_accepted\_*}$} \\
\cline{2-6}
& \multicolumn{2}{c|}{$N_{det\_E}$} & \multirow{2}{*}{$N_{f\_det\_E}$} & \multirow{2}{*}{$N_{undet\_E}$} & \multirow{2}{*}{$N_{w\_undet\_E}$} & \\
\cline{2-3}
& $N_{accepted\_*}$ & $N_{unaccepted\_*}$ & & & & \\
\cline{1-7}
$foE$ & 1039 & 34 & 264 & 2048 & 191 & 86.32\% \\
${h^{'}E}$ & 757 & 316 & 264 & 2048 & 191 & 78.43\%\\
\bottomrule
\end{tabular}
\label{表5_2}
\end{table}

由表~\ref{表5_2}可知,在3576张电离图的人工度量结果中,无E层描迹的电离图总数($N_{NoE}$)为2239。这2239张电离图的自动度量结果中,自动度量也未检测到E层描迹的电离图总数($N_{undet\_E}$)为2048张。根据公式~\ref{式5_9},在人工度量电离图无E层描迹的情况下,自动度量算法也未检测到E层描迹的正确率($R_{undetected\_E}$)为91.47\%。
\begin{linenomath}
\begin{align}
R_{undetected\_E}=N_{undet\_E}/N_{NoE}  
\label{式5_9}
\end{align}
\end{linenomath}
   
由表~\ref{表5_2}可知,在3576张电离图的人工度量结果中,存在E层描迹的电离图总数($N_{E}$)为1337张,这1337张电离图的自动度量结果中,未能成功检测到E层描迹的电离图总数($N_{f\_det\_E}$)为264张,能够成功检测到E层描迹的电离图总数($N_{det\_E}$)为1073张。根据公式~\ref{式5_10}可知,在人工度量结果中电离图存在E层描迹的情况下,自动度量算法能检测E层描迹的正确率($R_{detected\_E}$)为80.25\%。  
\begin{linenomath}
\begin{align}
R_{detected\_E}=N_{det\_E}/N_{E}  
\label{式5_10}
\end{align}
\end{linenomath}

在自动算法检测到的存在E层描迹的1073张电离图中,自动度量参数$foE$、${h^{'}E}$的自动度量结果中可以被接受的电离图总数($N_{accepted\_foE}$、$N_{accepted\_h'E}$)分别为:1039张、757张;根据公式~\ref{式5_11}、~\ref{式5_12}可知,在自动度量算法检测到存在E层描迹的电离图中,自动度量结果的可接受率($R_{accepted\_foE}$、$R_{accepted\_h'E}$)分别为96.83\%, 70.55\%。
\begin{linenomath}
\begin{align}
R_{accepted\_foE}=N_{accepted\_foE}/N_{det\_E}  
\label{式5_11}
\end{align}
\end{linenomath}
\begin{linenomath}
\begin{align}
R_{accepted\_h'E}=N_{accepted\_h'E}/N_{det\_E}              
\label{式5_12}
\end{align}
\end{linenomath}    
为了对算法的整体正确率进行评价,考虑所有电离图的识别情况,可以根据公式~\ref{式5_13}、~\ref{式5_14},计算出E层参数$foE$、$h^{'}E$的可接受率($R_{all\_accepted\_foE}$、$R_{all\_accepted\_h'E}$),分别为86.32\%、78.43\%。
\begin{linenomath}
\begin{align}
R_{all\_accepted\_foE}=(N_{accepted\_foE}+N_{undet\_E})/(N_{E}+N_{NoE})    
\label{式5_13}
\end{align}
\end{linenomath}
\begin{linenomath}
\begin{align}
R_{all\_accepted\_h'E}=(N_{accepted\_h'E}+N_{undet\_E})/(N_{E}+N_{NoE})    
\label{式5_14}
\end{align}
\end{linenomath}
    
通过将数据集内的电离图用本文的自动度量方法得到的Es层描迹类型识别结果和人工度量得到的结果进行对比。如果电离图的自动度量方法检测到的Es层描迹个数和类型与人工度量结果完全相同,就认为该电离图的Es层描迹的类型识别是完全正确。本文算法在数据集内电离图Es层描迹的类型识别正确率为55.70\%。


%=============================================================================================================
\section{分析}
\label{5_2}
通过对实验结果进行分析可知,本文算法在描迹检测和参数度量方面取得了较高的正确率。本文的Es层描迹类型识别算法主要根据电离图人工度量经验以及人工度量知识提出的,实际上,人工度量知识也是根据早期电离层人工度量分析给出的,取决于人工经验具有很大的不确定性。本文算法对于常规的E区描迹具有较好的识别效果,但在部分电离图上还存在比较特殊的描迹,对于这类电离图,不同的度量者对同一张电离图的识别结果也存在一定的差异性。所以,特殊描迹电离图是降低Es层描迹识别正确率的主要原因。















































