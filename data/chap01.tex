
%%% Local Variables:
%%% mode: latex
%%% TeX-master: t
%%% End:

\chapter{绪论}
\label{cha1}

\section{课题研究背景及意义}
\label{1_1}

电离层是地球磁层的内界,由于它会影响无线电波的传播,同时有望成为自然灾害的预警器,因此对电离层的监测及研究对科技发展和人类生活都有重要意义。电离层的扰动会对电离层的飞行器、无线电通信、导航定位系统、雷达探测系统造成一定的影响。此外,研究表明,地球上的自然灾害例如火山爆发、地震、海啸、剧烈的台风和雷暴等会引起电离层的扰动\cite{liu2012seismo}\cite{maruyama2011ionospheric},因此,电离层研究对于学术研究有重要的意义,同时也有非常大的应用价值\cite{luque2009emergence}\cite{engwall2009earth}\cite{brambles2011magnetosphere}。
      
目前我国的信息化速度在高速发展,卫星通信、卫星导航和星载雷达系统已经被广泛的应用于人们的生活中,同时也用于我国的军事建设领域。现代的军事战争是高科技的战争也是信息化战争,基于空间技术的各种信息化军事平台和装备在信息获取、传输和使用中有重要作用\cite{许正文2005电离层对卫星信号传播及其性能影响的研究}。电离层约是地面60km以上到磁层顶之间的整个空间。电离层既可以反射来自地面的无线信号,同时也可以改变穿过电离层的高频电波的传播方向、相位、振幅及偏振状态、速度等状态\cite{李建胜2011电离层对雷达信号和导航卫星定位影响的分析与仿真研究}。通过观察电离层对星载无线电系统的影响可以知道空间天气对信息系统会产生影响,也可以利用信息系统的变化来辅助预测天气情况。电离层的强烈扰动异常会影响GPS定位、信号质量下降、天基雷达成像质量降低、通信异常。因此,电离层的实时监测和研究对保障无线电通信的可靠运行、提高卫星定位系统的精度、促进军事的信息化建设都有重要意义。     
     
同时,多项研究表明地震与电离层的异常活动存在一定的关联性,在地震前后电离层都会出现不同程度的扰动\cite{pulinets2004ionospheric}\cite{liu2004pre}\cite{maruyama2011ionospheric}。国际上几十年的观测研究表明,地震与电离层的异常活动之间存在一定的关系。文献\cite{maruyama2011ionospheric}\cite{liu2012seismo}提到了日本福岛和中国四川省汶川县分别发生了九级和八级的强烈地震都导致了电离层发生了激烈并且异常的变化。经过几十年的研究,科学界已证明,除地震外,我们熟知的海啸、飓风等现象都会导致电离层发生变化。所以在一定程度上可以通过监测电离层是否发生异常变化,来预测是否会发生严重的自然灾害。
          
人类从1899年就开始了电离层的相关研究,对电离层研究的主要方法是利用电波在等离子中的各种效应进行电离层探测。世界各国已经有了各种电离层探测方法,出现最早、使用最广泛的是垂直探测\cite{reinisch2009new}\cite{galkin2009accuracy}。
     
电离层垂直测高仪的原理是从地面向电离层发射具有不同频率的脉冲波,脉冲波经过反射回到地面,垂直探测仪在地面接收脉冲波的反射回波,同时记录在这一过程中所用的时间,由此得到我们熟知的以频率为横坐标、以虚高为纵坐标的电离图, 通过对电离图上的描迹的类型识别与参数度量,可以获得电离层的特征参数,再对大量电离的特征参数进行反演计算可以得到电子浓度剖面。研究者通过它能够了解被探测区域的电离层情况,这有利于更进一步对电离层的各种特性进行探索和对可能产生的电离层异常扰动进行预报。因此垂测电离图的自动度量算法研究具有重要意义,所以一直受到科研工作者的关注。
     
考虑到电离层监测对于通信和自然灾害预警的重要性,目前全世界已有数百台数字测高仪在连续的观测记录电离图,平均每天可达到1.1万张,同时世界各地对电离层研究的趋势仍在持续上升,并且随着信息处理技术的不断发展,电离层测高仪及其得到电离图的数量将会持续增长\cite{galkin2011global}。我国已在北京、广州、西安、新疆、青岛等不同地区建立了电离层观测站。目前对于垂测电离图的识别与度量主要还是由专业的度量人员依靠度量知识与经验人工完成度量工作。传统的人工度量方法,存在很强的主观性,即不同的度量者对于同一张电离图会得到不同的度量结果,为以后的电离层反演研究带来一定的误差。此外,由于目前的人工度量人员有限,无法实现24小时实时度量。综上所述,人工度量电离图的方式已不再满足现在的电离图研究需求。随着人类对空间科学研究的不断深入,研究者想要获得大量的、准确的电离图度量结果,因此垂测电离图的自动判读研究具有重要的意义。     
   
\section{国内外研究现状}
\label{1_2}

从二十世纪60年代至今,国内外科研工作者在垂测电离图自动判读方面做了很多工作,也取得了较大的进展。目前垂测电离图自动判读主要思路包括以下步骤:去除电离图噪声、查找及判识描迹、获取电离层相关参数。
      
首先,去除电离层测高仪探测数据中无用的数据信息;其次,根据电离层特性确定描迹信息,并去除噪声;再次,根据电离层描迹特征提取各层描迹;最后,根据URSI\cite{piggott1972handbook}国际判识标准运用曲线拟合等算法获取电离层相关参数。电离图的自动度量算法研究思路是相同的,但是不同研究者从不同的学科领域出发来完成电离图自动判读算法的每个步骤。
     
 目前应用比较广泛的电离图自动方法有以下三种:ARTIST\cite{galkin2008new}、Autoscala\cite{scotto2012automatic}和ESIR\cite{sojka2009sounding}。
        
美国洛威尔大学提出的ARTIST方法具有识别率高、应用广泛等特点\cite{reinisch1983automatic}\cite{reinisch2005recent}。此方法通过对每个反射回波设定阈值来去除电离图数据的噪声;然后利用特定角度的边缘点检测算法和人工神经网络进行描迹提取,最后利用曲线拟合法对电离图描迹进行识别与度量。该方法与UMLCAR 研制的 DPS 系列测高仪配合使用,可以得到较高的度量精度,但该算法用于中国电波传播研究所所用的测高仪并不能取得较好的效果。
      
意大利学者提出了Autoscala方法。该方法运用 correlation 技术结合曲线拟合方法对F2 层\cite{pezzopane2007automatic}\cite{scotto2008removing}\cite{pezzopane2010highlighting}、Es 层\cite{scotto2007method}和 F1 层\cite{pezzopane2008method}描迹分别进行描迹判识、描迹拟合和电离层参数获取,其优点是可忽略极化信息,利用长期的电离图度量所产生的经验值来判断电离图上是否存在某层描迹。
      
美国空间环境公司(SEC)研发的电离图自动度量方法为ERIS\cite{rice2009expert}\cite{sojka2009sounding}。该方法的特点是通过利用电离层的物理模型将原始电离图转换为ESIR 格式,然后对描迹数据的完整性和电离层是否为非常规的复杂电离图进行判断;如果电离图时常规电离图,基于模式识别技术对电离图进行度量。

此外,澳大利亚的Fox和Blundell\cite{fox1989automatic}也提出了电离图自动判读方法。该方法主要运用曲线拟合和外推法来提取描迹,并提取识别描迹类型需要的各类特征,同时判断原始电离图是否与描迹图一致,实现对电离图各层描迹的判识,最合根据URSI标准利用外推法获取电离图参数;但是该算法仅能用于只记录O波的测高仪。日本的研究者\cite{igi1993automatic}也对电离图自动判读做了相关研究,该方法主要运用了数学和模式识别技术对电离层各层参数进行自动判读,该方法的最大特点是不用区分寻常波与非寻常波。此外,台湾学者Tsai 和 Berkey 等\cite{tsai2000ionogram}、 俄罗斯的Pulinets\cite{pulinets1995automating} 和土耳其的 Arikan 等\cite{kan2002new}都对电离图自动判读研究做了有意义的工作。
      
近几年来,我国学者在电离图自动判读算法中做了大量的研究。中国科学院的宁百齐、丁宗华等\cite{丁宗华2006电离层频高图自动度量与分析的方法研究}\cite{ding2007automatic}\cite{丁宗华2007电离层频高图参数的实时自动度量与分析}通过对电离层的电子浓度剖面进行 Empirical Orthogonal Function分解的得到经验正交函数集,通过对经验正交函数集进行反演计算,得到对应的电离图描迹,并与探测到的电离图数据进行对比,找到与原电离最接近的描迹,获取电离图的度量参数。中国学者柳文等\cite{柳文2009基于}也提出了电离图自动判读算法。首先利用IRI模型反演电离图E层描迹,同时借鉴国外方法对F层描迹进行搜索,同时结合电离图的参数经验值,从搜索到的描迹中,选择有效描迹并进行判读。此外,中国电波所的研究者\cite{凡俊梅2009电离层斜向传播模式的智能识别}和哈尔滨工业大学的邓维波等\cite{李雪2010返回散射电离图智能判读}分别对斜测电离图和返回散射电离图的自动判读进行了相关研究。
 
综上,研究者对电离图自动判读方法的研究主要根据电离层的物理特性,利用电离图人工度量经验和度量参数经验值,运用数学等领域的方法对描迹进行自动提取和判读。目前存在的很多电离图自动判读方法基于电离图度量参数的经验值提出、依赖于电离层的物理模型\cite{bilitza2008international}、同时很多判读方法只适用于特定的测高仪,因此这些方法想要被广泛的推广和应用还存在一定的困难。
     
随着图像处理和模式识别领域的发展,为电离图的自动判读提供了新的思路。我们将电离图描迹的频率、虚高及其信号强度可以由电离图灰度图像的位置和灰度值表示。基于电离层的分层结构及各层描迹形态特征,运用图像处理、图像分析技术,结合电离图的解释与度量标准,实现电离图上的描迹检测与识别。
     
%=============================================================================================================================
\section{课题来源}
\label{1_3}
本文课题来源于青岛市科技发展计划项目“基于图像分析的垂测电离图自动判读研究”(批准号:13-1-4-223-jch)和中央高校基本科研业务费项目“电离层垂直探测频高图自动解释及度量”(批准号:201313011)。 本文利用中国电波传播研究所提供的中国不同地区不同时间的五万张电离图为数据集,进行了电离图E区描迹自动判读方法的研究。

%=============================================================================================================================
\section{主要工作及安排}
\label{1_4}

第一章:绪论。主要介绍了电离图自动度量的研究背景、意义及国内外的研究进展。
     
第二章:电离层及垂测电离图。包括电离图的形成原因及其特征、电离层的主要探测方法、垂测电离图及其人工度量经验,以及本文的整体方案流程。在电离层探测技术的介绍中重点介绍了垂直探测电离层技术。
     
第三章:垂测电离图分割。主要包括将垂测数据转换为电离图、电离图预处理、电离图E区F区分割。
     
第四章:E区电离图类型识别及参数度量。根据电离图E区描迹的人工度量步骤,提出了电离图E区描迹的自动度量算法,算法主要包括以下几个步骤:电离图E区描迹检测、基于扩散程度的描迹区域分类、E区电离图分类、描迹分割、去除Es层描迹多次反射、描迹特征提取、识别描迹类型并获取相关参数。
     
第五章:实验分析。运用本文的方法对中国大量电离图进行实验,并对结果进行分析。
     
第六章:总结与展望。对本文提出的电离图E区描迹自动度量方法的进行总结,并针对方法存在的问题和不足对课题进行展望。
%=============================================================================================================================


