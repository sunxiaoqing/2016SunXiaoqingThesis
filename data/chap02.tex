
%%% Local Variables: 
%%% mode: latex
%%% TeX-master: t
%%% End: 

\chapter{电离层及垂测电离图}
\label{cha2}

 本章主要介绍电离图自动度量的基础,包括电离层的形成及其特征、造成电离层扰动的原因、电离层的分层结构及各层的特征。同时本章也介绍了电离层探测方法、垂测电离图及电离图的人工度量经验。只有对电离层的形成原因和人工度量步骤有充分认识,并亲自度量大量电离图,才能设计出对各类电离图都适用的自动度量算法。

%============================================================================================================
\section{电离层的形成及其特征}
\label{2_1}

电离层是地球大气的一个电离区域。由于电离层受到了太阳高能辐射以及宇宙线的激励而发生电离产生的大气高层。极短紫外辐射(EUV)和X射线导致电离层的原子和分子电离为正离子、负离子和自由电子,形成电离层 ,但是正负离子和电子所带电荷量相互抵消,电离层总体上是不带电的。X射线会随着太阳活动不断变化,主要使电离层D层和E层发生电离。源自太阳的EUV比较稳定,但它随太阳黑子数量的变化存在相应的月变化和年变化。
     
太阳极短紫外线辐射会被原子和分子吸收,同时也会使原子与分子发生电离现象。随着太阳辐射穿透更深层的地球大气,它的强度也在减弱。电子产生速度与EUV强度和大气密度成正比。在大气层的顶部,尽管EUV强度高,但大气密度很小,因此电离程度比较弱。在E层底部大气密度很大,但EUV强度很小,产生的离子和电子也很少。介于E层和大气顶部之间的区域,会出现最大电子浓度,形成了某一电离层。
%-------------------------------------------------------------------------------------------------------------
\section{电离层扰动来源}
\label{2_2}

太阳活动和地磁活动是造成电离层扰动的主要原因\cite{徐彤2009中低纬电离层模型及其异常现象相关研究}。 此外,地震等自然灾害也会引起它的突变。
      
EUV和X射线是电离层中的空气发生电离的主要作用。由于X射线会随着太阳耀斑迅速增长且不规律变化,同时极短紫外辐射也会随着太阳黑子数量的变化发生相应的月变化和年变化,太阳辐射强度的变化会影响电离层中分子和原子被电离的程度。所以太阳活动会使电离层具有周期的日变化、日变化、年变化等特征。对于日变化,在白天,太阳辐射强,大气分子的电离速度快,电离层电离密度增加;接近晚上,让太阳辐射逐渐减弱,电离层电离密度减小。
       
地球自身的地磁活动会引起电离层扰动。出现磁暴现象时,磁层的高能粒子由于磁场的作用向下沉降,在这过程中会使中性大气升温,发生膨胀,引起大范围的电离层扰动,形成电离层暴。

%-------------------------------------------------------------------------------------------------------------
\section{电离层结构组成及各层特征}
\label{2_3}

电离层结构可以由电离层相关参数的空间分布来表示,比如:电离层中的电子、离子在单位体积内的个数或电离层中电子和离子温度等。在电离层的研究中主要利用电子密度(单位体积的自由电子数)的空间分布来表示电离层的分层结构。电子密度数值的大小与EUV和X射线的强度和大气密度有紧密的联系。
       
由于重力的作用地球大气在不同的高度上有不同的大气密度,同时不同高度上太阳辐射强度也不相同。如图~\ref{图2_1}为典型的电离层电子密度剖面图,图中的两条曲线分别为黑夜和白天的电子密度分布情况,其中纵坐标表示距离地面的高度、横坐标表示电子密度的大小。电离层电子浓度随高度变化,存在几个极大值区,称之为层。白天时电离层划分为D层、E层、F1层和F2层,晚上时只存在F2层(高度大约在200km以上),这时通常称为F层。
\begin{figure}[h]
  \centering
  \includegraphics[height=6cm]{图2_1}
  \caption{典型的电离层电子密度剖面图}
  \label{图2_1}    
\end{figure}

D层离地面60~90km高度范围内。X射线使D区的大气分子和原子发生电离现象,而形成了电离层D层。大多数高频电路的吸收几乎全部发生在D层。D层的电子浓度最大时也是吸收最大时。大规模的太阳耀斑爆发期,太阳辐射的X射线迅速增加,造成D层区域电离速度增加,电离层吸收也迅速增加。
       
E层高度从D层一直延伸到150km左右,E层的等离子体能够满足光化学平衡,是中等浓度的分子离子层。E层临界频率存在日变化规律,在中午时,太阳的仰角变大,电子产生的速度最大,这时E层的临界频率达到最大值。同样E层电离层还存在季节变化,夏季的太阳天顶角比冬季小,因此夏季入射到电离层的太阳辐射强度比冬季强,电子产生的速度变大,夏季时E层的临界频率也达到最大。 在晚上时,太阳辐射消失时,E层等离子体的在复合作用下快速消失,E层消失。因此电离层E层描迹仅在白天出现。
       
F层描迹在150~1000km的高度范围内,是电离层的主要组成部分,由于其电子密度在电离层中最大,因此它会对无线通信有重要意义。 F层白天F层分为F1层和F2层,F1层仅在白天出现。F1层和F2层具有不同的特性。
       
F1层受到光化学作用在白天出现,晚上消失。F2层在白天和晚上都存在。从白天到晚上,F1层和E层描迹在逐渐消失,而F2层描迹却始终存在。一个使$F2$层一直存在的原因是中性气体中的大气层风。在白天,低纬度地区的上层大气被太阳加热,约300千米高的大气层风吹向高纬度(例如极区)。中性气体的风沿水平方向吹,但离子和电子由于地球磁场的原因不能越过磁力线。然而,白天离子和电子趋向于沿磁力线吹往更低高度,在那里它们在再结合的过程中被消耗。夜间,低纬度的上层大气变冷,大气层风吹向赤道。离子和电子沿磁力线向上吹,在那里再结合发生的很慢。因此,F2层在夜间一直存在。
      
Es层(Sporadic E)是偶发E层,即E区的突发不均匀结构,称为Es层,它的厚度约为100米到2千米之间,水平尺度为200米到100千米之间,有时候Es层可能在更大的范围内延续。一般来说Es层是一个薄层,出现时间不确定。有时Es层会呈现不透明状而遮住更高的层。电子密度很高,反射无线电信号的能力很强。同时电离图Es层描迹会出现扩散现象,国际电离图度量标准制定义了多个Es层描迹。电离图Es层描迹的出现时间和描迹类型不定。
      
扩展F层是F区的突发不均匀结构。经常出现在极光椭圆区和地磁赤道区的夜间出现。它的扩散特性会使穿过它的无线电信号发生衰减。
    
在电离层中,地球磁场会使无线电波分离为O波(寻常波)和X波(非寻常波)。O波与X波在传播过程中是相互独立的,在电离图上会出现O波描迹与X波描迹。由电磁理论知识,电离层O波与X波之间存在频率差为半个磁旋频率。磁旋频率取值与地点的纬度有很大的关系,我国大部分地区都属于中纬度地区,中纬度地区的磁旋频率为1.2MHz。
     
%-------------------------------------------------------------------------------------------------------------
\section{电离层探测方法}
\label{2_4}
电离层探测技术的发展对于电离层领域的研究具有重要意义,同时也是电离层其他领域研究的基础。人类对电离层的研究从1924年12月Sir E. V. 阿普尔顿和巴尼特用连续波法进行电离层高度的实验,同时证明了电离层的存在。随后1925年布雷特和图夫发明了电离层垂直探测装置。在1949年人们开始利用火箭进行对电离层的直接探测实验,在1957年人们开始利用人造卫星在高空对电离层进行直接探测。随着科学技术的发展,探测方法在不断的建立和完善,电离层探测技术是利用电离层会对射入其空间的电磁波发生反射、吸收、散射、多普勒频移等。利用测高仪采集某个电离层区域的数据,可以获取电离层相关物理参量。
      
电离层的探测方式可以分为:直接探测、间接探测\cite{林晨2004软件无线电在电离层测高仪中的应用研究}。直接探测是指将装有探测仪器的火箭或卫星发射到电离层,探测仪器在电离层直接探测电离层的相关信息。间接探测是指在地面利用探测仪器接收电离层信息来获取电离层的相关特性。间接探测又可以分为主动和被动。间接主动探测是指利用探测仪主动的向电离层发送电磁波,同时探测仪器会在地面接收经电离层反射回来的电磁波,并记录相关数据,通过对探测仪纪录的数据进行处理,分析电离层的相关特性。间接被动探测是指探测仪不会主动的向电离层发送电磁波,而是利用探测仪接收来自电离层自身辐射,并记录相关数据,来分析电离层特性。
     
被动探测是指利用探测仪器接收电离层自身的辐射,通过计算电离层的特征信息。主动探测是指使用高频雷达向电离层主动发射电磁波,然后通过分析处理回波信号得到电离层的相关特征参数。

主动探测方法主要包括垂直探测、斜向探测、返回散射探测。尽管目前有众多的电离层探测技术,但垂直探测技术依然是最主要的电离层探测方法。
     
1925年人类发明了第一台电离层垂直探测仪,电离层垂直测高仪实质上是一种高频雷达。垂直测高仪垂直向上发射电磁波,当射入到电离层的电磁波频率满足其反射条件时,入射电磁波将被电离层反射回地面,探测仪通过纪录发射和接收脉冲之间的时间延迟$\tau$,获取电离层的高度信息$h'$。其中,由于电磁波在大气中传播时受到时延、折射、衰减等因素的影响,电磁波的传播速度小于光速,因此利用式~\ref{式2_1}计算出的电离层高度大于电离层的实际高度,所以称为虚高$h'$。
 \begin{equation}
h'=\frac{1}{2}c\tau
\label{式2_1}
\end{equation}
        
通过改变发射电磁波的频率可以得到不同频率上的反射时间延迟$\tau$的纪录,当发射频率$f$在整个短波波段范围(一般为0.5~30MHz,TYC-1型电离层探测仪为1~32MHz)内以一定的频率作为步长进行持续改变,就能纪录到虚高(发射时延$\tau$)随发射频率$f$变化的曲线图,即垂测电离图(频高图)。

电离层的倾斜探测是指在地面上用探测仪倾斜地向电离层发射电磁波,并在距发射点一定距离的位置接收反射回波,并记录信号在传播过程中的时间、频率信息。
  
电离层返回散射的探测,是由探测仪倾斜地向电离层发射电磁波,电磁波电离层返回地面,返回地面的电磁波向多个方向散射,接收器只接收散射后返回发射点的电磁波,并记录电磁波的散射回波的幅度、时延、频率等信息。返回散射探测的优点是可以获取较广范围内的电离层特性,这种方法经常用于检测短波无线电覆盖区、检测运动目标及探测电离层结构。
%-------------------------------------------------------------------------------------------------------------
\section{垂测电离图简介}
\label{2_6}

如图~\ref{图2_3}是中国电波传播研究所研制生产的第四代(TYC-1型)全自动数字式电离层垂直探测电离图的人工度量软件工作界面。垂测电离图自动度量软件读取的是我国某第在2010年2月26日15时探测得到的电离层数据。电离图的横轴代表扫频频率,频率变化范围为1MHz~20.2MHz,垂直轴方向为虚高,虚高的范围为0km~800km。电离图上的每个像素点代表一个回波信号点,信号点列坐标为信号点频率值,像素点行坐标为信号点的虚高值,信号点灰度值代表信号点的强度。软件的左下方共有14个电离图特征参数需要人工度量;界面的右下方可以对电离图上信号显示强弱进行调整。 电离图共有十四个参数需要进行度量:$fmin$、$foE$、$h'E$、$foEs$、 $fbEs$、 $h'Es$、  $foF1$、$fbEs$、$h'Es$、 $foF1$、$h'F$、$M3F1$、$h'F2$、$foF2$、$M3F2$、$fxI$、Es type。

\begin{figure}[h]
\centering
\includegraphics[height=6cm]{图2_3}
\caption{垂直探测电离图人工度量软件工作界面}
\label{图2_3}    
\end{figure}

$fmin$:表示在电离图所有描迹的最小频率;

$foE$:表示E区最低厚度的寻常波临界频率;

$h'E$:表示E层描迹的最低虚高;

$foEs$:表示Es层连续描迹寻常波分量的顶频;

$h'Es$:表示Es层描迹的最低虚高;
 
Es type:Es描迹共有11种类型,有$f$、$l$、$c$、$h$、$q$、 $r$、$a$、$s$、$d$、$d$、$n$、 $k$。在某个站点并不能观测到所有的Es类型,在中纬度经常出现的是$f$、$l$、$c$和$h$型,而在高纬度地区经常出现的是$a$、$r$型。

$fbEs$:是Es层的遮蔽频率,即是Es层允许从上面层反射的第一个频率。$fbEs$表示Es层透明度的度量。$fbEs$是由通过Es层观测到较高层寻常波分量的最低频率决定。

$foF1$:是F1层寻常波的临界频率。

$h'F$:是F1层寻常波的最低虚高。
  
$h'F2$:是F区最高稳定层寻常波描迹达到的最低虚高。

$foF2$:是F区最高稳定层寻常波临界频率。

$fxI$:可接收到的F层反射的最高频率,是关于F层散射的一个标志。

M因子:是一个从垂测频率获得给定距离斜向传播最大可用频率的转换因子,以3000千米作为标准传播距离的M因子,称为M(3000),它通常要附加上反射层的名称来表示,如$M(3000)F2$或$M(3000)F1 $。

目前,全世界每天有200多个电离层观测站在不间断的探测电离图,国内主要的观测站在北京、青岛、重庆、兰州、拉萨、满洲里、乌鲁木齐、长春、广州、海口、昆明、苏州、新乡、西安、伊犁、阿勒泰等城市。随着每天各地采集的电离图数量在不断增加,传统的人工度量电离图的方法已经不能满足现实的需求。电离图F区描迹的形状和变化相对比较固定,目前已存在比较成熟的垂测电离图自动度量算法。电离图E区描迹是由E层、Es层、E2层构成,Es层描迹的主要特点是出现没有规律,种类多同时还存在不同程度的扩散。所以,本文就垂测电离E区描迹的识别并对 $foE$、$h'E$、$foEs$、 $h'Es$、$fbEs$、 Es type 这些电离图E区相关参数的自动度量进行相关研究。

 %-------------------------------------------------------------------------------------------------------------
\section{垂测电离图E区描迹人工度量经验}
\label{2_7}
       
人工对电离图E区描迹的识别主要是根据描迹形状、位置等特征。电离图E区描迹主要包括E层、Es层、E2层,Es层描迹有$c$型、$h$型、$f$型、$l$型、$q$型、$a$型、$d$型、$k$型、$r$型、$n$型11种类型。下面对电离图E区常出现的各类描迹做简单介绍:
       
E层根据描迹形态可以分为典型E层、未观测到的E层、E层被Es层遮蔽、E层底部不水平、E层临频附件的衰减、微粒E层等几种情况。E层描迹完整的情况存在时延、时延是由各种电磁波在电离层中传播造成的。如图~\ref{图2_4}为典型的E层描迹。E层描迹的形状是上翘型,即描迹随着频率的增大,虚高呈现不断增大。E层描迹的最低虚高在115km~125km,E层O波描迹的最大频率小于$fminF$,仅在白天出现。

\begin{figure}[h]
\centering
\includegraphics[height=5cm]{图2_4}
\caption{E层描迹}
\label{图2_4}    
\end{figure}   
      
Es层描迹种类较多,共有11种类型,且每个站点观察到的描迹并不相同,在我国经常出现的有$l$型、$c$型、$h$型、$f$型、$n$型。在我国每个站点观测到的Es层类型与站点的位置有关。在低纬度、中纬度、高纬度地区经常出现的Es层类型不同。下面将详细地介绍各种Es层描迹的特征。
\begin{enumerate}     
\item Es层$c$型:也称为尖角型描迹,虚高在115km~125km之间,仅在白天出现。Es层$c$型的起始频率小于或等于$foE$。描迹形状为下降型,即描迹随着频率的增大,虚高有减小趋势,描迹斜率的变化从起始频率开始接近于垂直方向逐渐下降,然后趋于直线型。通常Es层$c$型描迹和正规的E层描迹相连(如图~\ref{图2_5}),E层和Es层$c$型描迹虚高相近。部分Es层$c$型描迹也会在没有E层描迹的情况下单独出现。
\begin{figure}[h]
\centering
\includegraphics[height=5cm]{图2_5}
\caption{Es层c型描迹}
\label{图2_5}    
\end{figure}   

\item Es层$h$型,也称为高型描迹,虚高在130km~180km,仅在白天出现。Es层$h$的起始频率等于或大于$foE$。如图~\ref{图2_6},正规的Es层$h$型描迹的最低虚高与E层描迹的最低虚高相差很大,同时与E层尖角不对称,即Es层$h$型描迹在最小频率处的虚高远高于E层描迹在最大频率处的虚高,即$h'Es$的值明显大于$h'E$。Es层$h$型描迹也会单独出现。
\begin{figure}[h]
\centering
\includegraphics[height=5cm]{图2_6}
\caption{Es层h型描迹}
\label{图2_6}    
\end{figure}   


\item Es层$f$型,也称为平型描迹,虚高在100km~130km,仅在夜间出现。如图~\ref{图2_7},Es层$f$型描迹为直线型,描迹的最低虚高不随频率的变化,同时描迹通常比较粗而浓。Es层$f$型描迹在任何纬度观测站的探测到的电离图中都存在此类Es层描迹。
\begin{figure}[h]
\centering
\includegraphics[height=5cm]{图2_7}
\caption{Es层f型描迹}
\label{图2_7}    
\end{figure}   

\item Es层$l$型:也称为低型,虚高在95km~110km之间,仅在白天出现。Es层$l$型描迹为直线型,描迹的最低虚高不随频率的变化。$l$型描迹的出现存在三种情况:$l$型O波单独出现的情况、$l$型OX波混合的情况、$l$型OX波分离的情况。$l$型O波单独出现和$l$型OX波混合时,这两种情况下的描迹都是没有间断的直线。为了确定度量参数$foEs$的值,对于一条没有间断的直线型描迹,我们要判读描迹属于$l$型O波单独出现的情况还是属于$l$型OX波混合的情况。对于判断OX波是否混合有两种判别方法:第一,如果F层有X描迹,则$ftEs\geq fminFx-fB/2$就认为l型描迹属于OX波混合的情况($fminFx$为F层X描迹的最小频率);否则为$l$型O波单独出现的情况。第二,如果F层X波不存在,但E层存在X波,同时$ftEs\geq fminEx-fB/2$($fminEx$为E层X波描迹的最小频率),就认为描迹为$l$型OX波混合的情况;否则为$l$型O波单独出现的情况。对于图~\ref{图2_8}我们采用上面的第一种判别方法,因为$foEs>fminFx-fB/2$,所以可认为描迹是OX波混合的$l$型。对于$l$型O波单独出现的情况,将描迹的顶频作为$foEs$;对于$l$型OX波混合出现的情况,用顶频减去$fB/2$求得$foEs$。

\begin{figure}[h]
\centering
\includegraphics[height=5cm]{图2_8}
\caption{Es层l型描迹}
\label{图2_8}    
\end{figure}   

\item Es层$q$型:也称为赤道型,扩散的和非遮蔽性的Es层描迹;白天和晚上都有可能出现,常在磁赤道附近的白天出现。如图~\ref{图2_9}所示,$q$型描迹在高度上呈现弱扩散,频率范围延伸的比较大同时没有清晰的下边缘。Es层$q$型描迹也经常与$l$型、$f$型描迹叠加出现。如果描迹的下边缘是一条严格的直线,同时在高度上存在弱扩散,频率范围延伸的比较大,就可认为是$q$型与$l$型、 或$q$型与$f$型叠加出现。

\begin{figure}[h]
\centering
\includegraphics[height=5cm]{图2_9}
\caption{Es层q型描迹}
\label{图2_9}    
\end{figure}   

\item Es层$a$型:也称为极光型,是一种扩散型描迹,描迹的虚高可以伸展到数百公里以上。Es层$a$型描迹的形状特征:外形有一个平缓的或缓慢上升的底边缘。$a$型描迹经常出现在高纬地区和有极光活动的中纬地区。如图~\ref{图2_10}为Es层$a$型描迹。

\begin{figure}[h]
\centering
\includegraphics[height=5cm]{图2_10}
\caption{Es层a型描迹}
\label{图2_10}    
\end{figure}   

\item Es层$s$型:也称为斜型,是一种扩散型描迹。描迹为上升型,虚高随频率稳定地增大。如图~\ref{图2_11}所示,Es层$s$型描迹常与Es层$l$型描迹同时出现,$s$型描迹主要从$l$型描迹的中间开始显露出来。Es层$s$型描迹主要在高纬度地区出现,在磁赤道区也存在比较弱的描迹。此类描迹只需要度量出描迹类型,不需度量参数$foEs$和$h'Es$。

\begin{figure}[h]
\centering
\includegraphics[height=5cm]{图2_11}
\caption{Es层s型描迹}
\label{图2_11}    
\end{figure}   

\item Es层$d$型:也称为D区型,是一种弱的扩散描迹,虚高低于95km,常在85km左右出现。因为Es层$d$型描迹并不是严谨的Es层,不需用来确定$fmin$、$foEs$、$h^{'}Es$。
  
\item Es层$k$型:也称为微粒E层,虚高在100~200km,经常出现在120km~180km,仅在晚上出现。在$k$型描迹出现时,F层的最小频率($fminF$)处描迹上翘,或至少同一时间三个国内站电离图$fminFx$处描迹上翘。常见的$k$型描迹形状为描迹右端上翘,如果在$fminF$的左端存在描迹,描迹的最大频率小于接近$fminF$,认为描迹是Es层$k$型O波描迹。如果在$fminF$的右端存在描迹,且描迹的最大频率接近$fminFx$,就认为描迹是Es层$k$型X波描迹。如图\ref{图2_12}为Es层$k$型O波描迹。

\begin{figure}[h]
\centering
\includegraphics[height=5cm]{图2_12}
\caption{Es层k型描迹}
\label{图2_12}    
\end{figure}   

\item Es层$r$型:也称为时延型,描迹在顶频附近随虚高增大,有小的扩散,$r$型描迹的穿透频率大于F层描迹的最低频率。如果描迹从$fminF$的右边出现,且描迹的最大频率大于$fminFx$,该描迹为Es层$r$型X波描迹。如图~\ref{图2_13}为Es层$r$型O波描迹。
\begin{figure}[h]
\centering
\includegraphics[height=5cm]{图2_13}
\caption{Es层r型描迹}
\label{图2_13}    
\end{figure}
 
\item Es层$n$型:用来表示不能归为以上这些类型的Es描迹。如图~\ref{图2_14}所示,在晚上描迹像$f$型,但是它的虚高不在$f$型描迹虚高的范围内,就认为描迹是$n$型。在白天,描迹为直线型,但它的虚高不在$l$型描迹的虚高范围内,也被认为是$n$型描迹。目前,经常遇到的$n$型描迹主要是以上这两种情况。

\begin{figure}[h]
\centering
\includegraphics[height=5cm]{图2_14}
\caption{Es层n型描迹}
\label{图2_14}    
\end{figure}
\end{enumerate}

E2层:通常出现在日出和日落前后的两个小时,虚高大于125km。E2 层的最大频率小于$fminF$。如图~\ref{图2_15}为常见的E2层描迹,描迹形状上翘。如果E2层在临频处衰减,E2层描迹可能变为一条直线。  

\begin{figure}[h]
\centering
\includegraphics[height=5cm]{图2_15}
\caption{E2层描迹}
\label{图2_15}    
\end{figure}
 
根据上面E区各种类型电离图的介绍,识别描迹主要根据以下特征:虚高、描迹形状、$foE$的经验数据、频率范围、描迹在高度上的离散情况、$fminF$、F层X波的最小频率值、Es层描迹的最大频率值、描迹的斜率变化。识别的过程简要概述为“有没有描迹→时间→从上到下→从左到右→再看反射”。我们将人工度量过程总结为以下几个步骤:
   
首先,判断 E 区内有没有描迹,如果有描迹,再看电离图数据的采集时间;如果没有描迹,E区电离图各项参数为零。
   
其次,电离图的采集时间可分为:白天、过渡期、晚上。 
   
在白天,电离图上可能出现的类型有E、E2、$l$、$c$、$h$、$a$、$q$、$r$($r$型一般都是在晚上出现,这里先不考虑)、$s$、$d$、$n$。下面为白天采集的且存在描迹的E区电离图人工判读过程:
\begin{enumerate}      
\item描迹的最低虚高小于95km,经常出现在80km左右,弱的扩散描迹,则为d型。
      
\item描迹的最低虚高在95km~110km之间,形状为直线型,且描迹不存在扩散现象,就认为描迹类型为Es层$l$型。接着判断描迹的OX波情况,如果直线中间有间断,同时间断处两侧描迹的最大频率相差0.7MHz,就认为描迹为OX波分离的情况。如果描迹是一条直线没有间断,这时我们需要判断描迹是OX波混合,还是O波单独出现的情况,判断的依据是:如果F层有X描迹,同时满足$ftEs\geq fminFx-fB/2$就认为描迹的X波存在,这时描迹为OX波混合的情况,否则描迹为O波单独存在的情况;如果F层没有X描迹,可以看是否存在E层的X波,如果E层的X波存在,同时$ftEs\geq fminEx-fB/2$,就认为描迹的X波存在;否则认为只有O波;当Es呈现全遮蔽(看不到F层),也认为是OX混合。
   
\item描迹的最低虚高在F层以下的高度范围内,形状为直线型,描迹在高度上呈弱扩散,频率范围延伸的比较大同时没有清晰的下边缘,这样的描迹是$q$型。在白天$q$型可能与$l$型叠加出现。判断有没有叠加的方法如下:如果描迹的下边缘是一条严格的直线,同时在高度上存在弱扩散,频率范围延伸的比较大,就可认为是$q$型与$l$型的叠加。
      
\item描迹的最低虚高在110km~125km之间,形状为直线型,描迹的最大频率小于F层的最小频率,同时描迹的最大频率与$foE$的经验数据相差很小,就认为描迹为E层(这是E层在顶频处出现衰减的情况)。
      
\item描迹的最低虚高在110km~125km之间,形状是直线型,描迹的最小频率与$foE$的经验值相接近,就可认为是$c$型(这种情况就是$c$型的低频处出现衰减)。接着再看是O波单独出现、OX混合、还是OX分离,具体的判别标准与$l$型相同。 

\item描迹的形状为直线型,描迹的最低虚高不在以上这些高度范围内,同时描迹也不具备Es层$q$型描迹的特征,那么就认为描迹类型为$n$型Es层描迹。 

\item描迹的最低虚高在110km~125km之间,同时描迹形状为右端上翘型,同时描迹的最大频率不能超过$fminF$,那么描迹为E层。从左向右观察,判断是否存在E层X波描迹。 
     
\item描迹的最低虚高110km~125km之间,如果描迹的形状与E层对称,为左端上翘型,就认为描迹为$c$型。$c$型和E层经常一起出现。如果E层和$c$型都没有衰减,E层的虚高值和$c$型的虚高值基本相等。再从左到右,判断$c$型描迹的OX波情况。 

\item描迹的最低虚高在125km~180km之间,描迹的形状为左端上翘或“U”字型,就认为该描迹为$h$型。如果E区电离图上存在E层,在$foE$处取到的E层描迹的最低虚高与此描迹的最低虚高存在一定的差值。所以对于存在E层的电离图,还可以根据E层描迹的虚高与$h$型描迹的虚高差,来判断$h$型的存在。 

\item如果描迹不属于以上任何一种类型,存在严重的扩散扩散现象,同时在高度上的扩散高度很大,那么就认为该描迹为$a$型。如果描迹存在扩散现象同时描迹下边缘为向上的斜线,则认为该描迹为$s$型描迹。
      
\item如果描迹不属于以上的任何一种类型,就可以认为该描迹为Es层$n$型描迹。 
\end{enumerate}  
   
在过渡期(指从白天到晚上、从晚上到白天过渡的两个时间段),依靠经验数据将过渡期固定在一个具体的时间段是不够精确的,在人工度量过程中,根据F层的描迹的变化情况来确定过渡期的出现时间。在过渡期内。除了上面白天出现的这些类型,还可能出现E2层描迹。对E2层的识别主要根据以下特点:虚高在125km以上,描迹的形状一种右端上翘,描迹的形状另一种像U字型,同时它的最大频率不超过F层的最小频率。如果有E层存在,那E层的最大频率等于E2层的最小频率;如果符合以上这些条件,但描迹形状为直线型,那么描迹为E2层,这是E2层顶频衰减的情况。
      
晚上主要出现的类型有$f$型、$r$型、$k$型、$n$型、$a$型、$q$型、$d$型、$s$型。
\begin{enumerate}       
\item描迹最低高度低于95km,同时为弱的扩散描迹,那么认为描迹为$d$型。 
      
\item描迹的最低高度在100km~130km之间,形状为直线,同时描迹粗而浓,那么描迹类型为$f$型。 然后判断描迹OX波的情况属于那种情况。OX波分离的判断与$l$型相同。$f$型OX波是否混合的判断依据:如果$ftEs\leq fB+0.25MHz$(fB一般取1.4MHz),则该描迹不存在X波,否则存在X波。
      
\item描迹在高度上呈弱扩散,频率范围延伸的比较大同时没有清晰的下边缘,就认为该描迹为Es层$q$型。在晚上可能出现Es层$q$型描迹与$f$型描迹叠加出现。判断有没有叠加方法如下:如果描迹的下边缘是一条严格的直线,同时在高度上存在弱扩散,频率范围延伸的比较大,就可认为是$q$型与$f$型叠加出现。 

\item描迹的形状右端上翘,描迹位于E区,如果描迹位于$fminF$左侧,且描迹的的最大频率大于$fminF$,则该描迹是$r$型描迹。如果描迹位于$fminF$的右侧,且描迹的最大频率大于$fminFx$,就认为描迹是$r$型的X波。存在$r$型,同时如果F层描迹在$fminF$处描迹上翘,认为此时还存在一个$k$型,$k$型描迹被遮蔽。 
      
\item描迹的最低虚高位于100km~200km之间,经常出现在120km~180km之间,$fminF$处描迹上翘,或至少同一时间三个国内站电离图$fminFx$处描迹上翘,描迹形状为右端上翘。如果描迹位于$fminF$左侧,描迹的最大频率小于接近$fminF$,认为描迹是$k$型的O波。如果描迹位于$fminF$的右侧,且描迹的最大频率接近$fminFx$,就认为描迹是$k$型X波。如果存在X波,从右向左可以找到$r$型的O波。
      
\item如果描迹不属于以上任何一种类型,存在严重的扩散扩散现象,同时在高度上的扩散高度很大,那么就认为该描迹为$a$型。如果描迹存在扩散现象同时描迹下边缘为向上的斜线,则认为该描迹为$s$型描迹。
          
\item如果描迹不属于以上的任何一种类型,就认为该描迹为$n$型描迹。
\end{enumerate}
      
最后,根据前面确定的描迹个数及类型,用描迹与它的多次反射描迹的最低虚高呈倍数关系,同时描迹与它的反射描迹形状相似等特征,来确定E区电离图上描迹的反射次数,并度量电离图的相关参数。

%-------------------------------------------------------------------------------------------------------------
\section{本文的整体方案流程}
\label{2_8}
由于电离层时变、色散等特殊性,并受太阳活动和地磁活动等因素的影响,电离图情况复杂多变。至今仍没有一个完善的方法适应复杂多变的电离图,电离图的自动判读研究仍有重要的意义。本文对电离图E区描迹自动度量进行了相关研究。电离图E区描迹由E层、Es层和E2层构成。Es层的主要特点是出现没有规律性,种类多,描迹还存在不同程度的扩散。在进行电离图E区描迹的自动度量算法设计之前,对大量电离图进行人工度量,并对人工识别E区描迹的过程进行总结。然后,利用图像处理及分析技术设计电离图E区描迹自动判读算法。人工对电离图E区描迹的判识过程主要有以下步骤:描迹查找提取、描迹特征、对描迹进行判识、获取电离图度量参数。
   
本文对电离图E区描迹的整体度量方案包括:垂测电离图E区F区分割、E区电离图类型识别及参数度量。根据电离层在E层和F层之间会存在无描迹信号的谷区,本文提出了水平投影积分法将电离图分为E区和F区电离图;利用图像处理和图像分析技术在E区电离图上进行候选描迹区域检测、非描迹区域排除;确定有效描迹区域后,根据描迹的扩散程度对描迹区域进行分类;通过对描迹区域内的信息进行分析将电离图分为五类:对于E区无描迹电离图直接给出度量结果;对于非常规复杂电离图进行标记,由人工进行度量;对于三类不同的电离图,本文根据电离图上描迹特征对其进行分割;利用Es层描迹多次反射的形成原理,确定描迹的反射次数并去除反射描迹;提取分割后描迹的特征,并对描迹类型进行识别与度量。本文工作的主要目标就根据E区电离图的人工度量经验,利用图像处理知识实现多种Es层描迹类型识别及$h^{'}Es$、$h{'}E$、$foE$、$foEs$、$fbEs$五个参数的值。如图~\ref{图2_16}为算法流程图。
\begin{figure}[h]
\centering
\includegraphics[height=16cm]{图2_16}
\caption{算法流程图}
\label{图2_16}    
\end{figure}
 
%-------------------------------------------------------------------------------------------------------------
\section{本章小结}
\label{2_9}

本章主要介绍了电离层相关内容,包括电离图的形成原因、造成电离层扰动的主要原因、电离层的结构组成及各层特征;然后列举了目前常用的电离层探测方法,并重点介绍了垂直探测技术及垂测电离图E区描迹的人工度量经验;最后提出了垂测电离图E区描迹自动度量流程。


