%%% Local Variables:
%%% mode: latex
%%% TeX-master: t
%%% End:

%\documentclass[bachelor,nofonts]{thuthesis}
\documentclass[master]{thuthesis}
%\documentclass[doctor]{thuthesis}
% \documentclass[%
%   bachelor|master|doctor|postdoctor, % mandatory option
%   winfonts|nofonts|adobefonts, % mandatory only for bachelor and Linuxer
%   secret,
%   openany|openright,
%   arialtoc,arialtitle]{thuthesis}
% 当使用 XeLaTeX 编译时,本科生、Linux 用户需要加上 nofonts 选项;
% 当使用 PDFLaTeX 编译时,adobefonts 选项等效于 winfonts 选项(缺省选项)。

% 所有其它可能用到的包都统一放到这里了,可以根据自己的实际添加或者删除。
\usepackage{thutils}
\usepackage{lineno,hyperref}
% 你可以在这里修改配置文件中的定义,导言区可以使用中文。
% \def\myname{薛瑞尼}
\usepackage{amsmath}
\usepackage{comment}
\begin{document}

% 定义所有的eps文件在 figures 子目录下
\graphicspath{{figures/}}


%%% 封面部分
\frontmatter

%%% Local Variables:
%%% mode: latex
%%% TeX-master: t
%%% End:

% 中国海洋大学研究生学位论文封面
% 参考:中国海洋大学研究生学位论文书写格式20130307.doc

% 为避免出现错误,下面保留[清华大学学位论文模板原有定义无需修改],
% 请直接跳到后面[中国海洋大学学位论文模板部分请根据自己情况修改]。

%%%%%%%%%%%%%%%%%%%%%%[清华大学学位论文模板原有定义无需修改]%%%%%%%%%%%%%%%%%%%%%%%
\secretlevel{绝密} \secretyear{2100}

\ctitle{清华大学学位论文 \LaTeX\ 模板\\使用示例文档}
% 根据自己的情况选,不用这样复杂
\makeatletter
\ifthu@bachelor\relax\else
  \ifthu@doctor
    \cdegree{工学博士}
  \else
    \ifthu@master
      \cdegree{工学硕士}
    \fi
  \fi
\fi
\makeatother


\cdepartment[计算机]{计算机科学与技术系}
\cmajor{计算机科学与技术}
\cauthor{薛瑞尼} 
\csupervisor{郑纬民教授}
% 如果没有副指导老师或者联合指导老师,把下面两行相应的删除即可。
\cassosupervisor{陈文光教授}
\ccosupervisor{某某某教授}
% 日期自动生成,如果你要自己写就改这个cdate
%\cdate{\CJKdigits{\the\year}年\CJKnumber{\the\month}月}

% 博士后部分
% \cfirstdiscipline{计算机科学与技术}
% \cseconddiscipline{系统结构}
% \postdoctordate{2009年7月——2011年7月}

\etitle{An Introduction to \LaTeX{} Thesis Template of Tsinghua University} 
% 这块比较复杂,需要分情况讨论:
% 1. 学术型硕士
%    \edegree:必须为Master of Arts或Master of Science(注意大小写)
%              “哲学、文学、历史学、法学、教育学、艺术学门类,公共管理学科
%               填写Master of Arts,其它填写Master of Science”
%    \emajor:“获得一级学科授权的学科填写一级学科名称,其它填写二级学科名称”
% 2. 专业型硕士
%    \edegree:“填写专业学位英文名称全称”
%    \emajor:“工程硕士填写工程领域,其它专业学位不填写此项”
% 3. 学术型博士
%    \edegree:Doctor of Philosophy(注意大小写)
%    \emajor:“获得一级学科授权的学科填写一级学科名称,其它填写二级学科名称”
% 4. 专业型博士
%    \edegree:“填写专业学位英文名称全称”
%    \emajor:不填写此项
\edegree{Doctor of Engineering} 
\emajor{Computer Science and Technology} 
\eauthor{Xue Ruini} 
\esupervisor{Professor Zheng Weimin} 
\eassosupervisor{Chen Wenguang} 
% 这个日期也会自动生成,你要改么?
% \edate{December, 2005}

% 定义中英文摘要和关键字
\begin{cabstract}
  论文的摘要是对论文研究内容和成果的高度概括。摘要应对论文所研究的问题及其研究目
  的进行描述,对研究方法和过程进行简单介绍,对研究成果和所得结论进行概括。摘要应
  具有独立性和自明性,其内容应包含与论文全文同等量的主要信息。使读者即使不阅读全
  文,通过摘要就能了解论文的总体内容和主要成果。

  论文摘要的书写应力求精确、简明。切忌写成对论文书写内容进行提要的形式,尤其要避
  免“第 1 章……;第 2 章……;……”这种或类似的陈述方式。

  本文介绍清华大学论文模板 \thuthesis{} 的使用方法。本模板符合学校的本科、硕士、
  博士论文格式要求。

  本文的创新点主要有:
  \begin{itemize}
    \item 用例子来解释模板的使用方法;
    \item 用废话来填充无关紧要的部分;
    \item 一边学习摸索一边编写新代码。
  \end{itemize}

  关键词是为了文献标引工作、用以表示全文主要内容信息的单词或术语。关键词不超过 5
  个,每个关键词中间用分号分隔。(模板作者注:关键词分隔符不用考虑,模板会自动处
  理。英文关键词同理。)
\end{cabstract}

\ckeywords{\TeX, \LaTeX, CJK, 模板, 论文}

\begin{eabstract} 
   An abstract of a dissertation is a summary and extraction of research work
   and contributions. Included in an abstract should be description of research
   topic and research objective, brief introduction to methodology and research
   process, and summarization of conclusion and contributions of the
   research. An abstract should be characterized by independence and clarity and
   carry identical information with the dissertation. It should be such that the
   general idea and major contributions of the dissertation are conveyed without
   reading the dissertation. 

   An abstract should be concise and to the point. It is a misunderstanding to
   make an abstract an outline of the dissertation and words ``the first
   chapter'', ``the second chapter'' and the like should be avoided in the
   abstract.

   Key words are terms used in a dissertation for indexing, reflecting core
   information of the dissertation. An abstract may contain a maximum of 5 key
   words, with semi-colons used in between to separate one another.
\end{eabstract}

\ekeywords{\TeX, \LaTeX, CJK, template, thesis}
%%%%%%%%%%%%%%%%%%%%%%%%%%%%%%%%%%%%%%%%%%%%%%%%%%%%%%%%%%%%%%%%%%%%%%%%%%%%%%%%

%%%%%%%%%%%%%%%%%%[中国海洋大学学位论文模板部分请根据自己情况修改]%%%%%%%%%%%%%%%%%%%
% 中国海洋大学研究生学位论文封面
% 必须填写的内容包括(其他最好不要修改):
%   分类号、密级、UDC
%   论文中文题目、作者中文姓名
%   论文答辩时间
%   封面感谢语
%   论文英文题目
%   中文摘要、中文关键词
%   英文摘要、英文关键词
%
%%%%%[自定义]%%%%%
\newcommand{\fenleihao}{}%分类号
\newcommand{\miji}{}%密级 
                    % 绝密$\bigstar$20年 
                    % 机密$\bigstar$10年
                    % 秘密$\bigstar$5年
\newcommand{\UDC}{}%UDC
\newcommand{\oucctitle}{基于图像处理的垂测电离图E区描迹自动判读方法}%论文中文题目
\ctitle{基于图像处理的垂测电离图E区描迹自动判读方法}%必须修改因为页眉中用到
\cauthor{孙晓庆}%可以选择修改因为仅在 pdf 文档信息中用到
\cdegree{工学硕士}%可以选择修改因为仅在 pdf 文档信息中用到
\ckeywords{\TeX, \LaTeX, CJK, 模板, 论文}%可以选择修改因为仅在 pdf 文档信息中用到
\newcommand{\ouccauthor}{孙晓庆}%作者中文姓名
%\newcommand{\ouccauthor}{***}%外审时用到
%\newcommand{\ouccsupervisor}{姬光荣教授}%作者导师中文姓名
%\newcommand{\ouccdegree}{博\hspace{1em}士}%作者申请学位级别
%\newcommand{\ouccmajor}{海洋信息探测与处理}%作者专业名称
%\newcommand{\ouccdateday}{\CJKdigits{\the\year}年\CJKnumber{\the\month}月\CJKnumber{\the\day}日}
%\newcommand{\ouccdate}{\CJKdigits{\the\year}年\CJKnumber{\the\month}月}
\newcommand{\oucdatedefense}{           }%论文答辩时间
%\newcommand{\oucdatedegree}{2009年6月}%学位授予时间
\newcommand{\oucgratitude}{谨以此论文献给我的导师和亲人!}%封面感谢语
\newcommand{\oucetitle}{Automatic traces scaling of E region from ionograms based on image processing}%论文英文题目
%\newcommand{\ouceauthor}{Haiyong Zheng}%作者英文姓名
\newcommand{\oucthesis}{\textsc{OUCThesis}}
%%%%%默认自定义命令%%%%%
% 空下划线定义
\newcommand{\oucblankunderline}[1]{\rule[-2pt]{#1}{.7pt}}
\newcommand{\oucunderline}[2]{\underline{\hskip #1 #2 \hskip#1}}

% 论文封面第一页
%%不需要改动%%
\vspace*{5cm}
{\xiaoer\heiti\oucgratitude

\begin{flushright}
---\hspace*{-2mm}---\hspace*{-2mm}---\hspace*{-2mm}---\hspace*{-2mm}---\hspace*{-2mm}---\hspace*{-2mm}---\hspace*{-2mm}---\hspace*{-2mm}---\hspace*{-2mm}---~\ouccauthor
\end{flushright}
}

%\begin{comment}

\newpage 
%\mbox{} 
%\newpage

% 论文封面第二页
%%不需要改动%%
\vspace*{1cm}
\begin{center}
  {\xiaoer\heiti\oucctitle}
\end{center}
\vspace{10.7cm}
{\normalsize\songti
\begin{flushright}
{\renewcommand{\arraystretch}{1.3}
  \begin{tabular}{r@{}l}
    学位论文答辩日期:~ & \oucunderline{2.5cm}{\oucdatedefense} \\
    指导教师签字:~ & \oucblankunderline{5cm} \\
    答辩委员会成员签字:~ & \oucblankunderline{5cm} \\
    ~ & \oucblankunderline{5cm} \\
    ~ & \oucblankunderline{5cm} \\
    ~ & \oucblankunderline{5cm} \\
    ~ & \oucblankunderline{5cm} \\
    ~ & \oucblankunderline{5cm} \\
    ~ & \oucblankunderline{5cm} \\
  \end{tabular}
}
\end{flushright}
}

\newpage 
%\mbox{} 
%\newpage

% 论文封面第三页
%%不需要改动%%
\vspace*{1cm}
\begin{center}
  {\xiaosan\heiti 独\hspace{1em}创\hspace{1em}声\hspace{1em}明}
\end{center}
\par{\normalsize\songti\parindent2em
本人声明所呈交的学位论文是本人在导师指导下进行的研究工作及取得的研究成果。据我所知,除了文中特别加以标注和致谢的地方外,论文中不包含其他人已经发表或撰写过的研究成果,也不包含未获得~\oucblankunderline{7cm}(注:如没有其他需要特别声明的,本栏可空)或其他教育机构的学位或证书使用过的材料。与我一同工作的同志对本研究所做的任何贡献均已在论文中作了明确的说明并表示谢意。
}
\vskip1.5cm
\begin{flushright}{\normalsize\songti
  学位论文作者签名:\hskip2cm 签字日期:\hskip1cm 年 \hskip0.7cm 月\hskip0.7cm 日}
\end{flushright}
\vskip.5cm
{\setlength{\unitlength}{0.1\textwidth}
  \begin{picture}(10, 0.1)
    \multiput(0,0)(0.2, 0){50}{\rule{0.15\unitlength}{.5pt}}
  \end{picture}}
\vskip1cm
\begin{center}
  {\xiaosan\heiti 学位论文版权使用授权书}
\end{center}
\par{\normalsize\songti\parindent2em
本学位论文作者完全了解学校有关保留、使用学位论文的规定,并同意以下事项:
\begin{enumerate}
\item 学校有权保留并向国家有关部门或机构送交论文的复印件和磁盘,允许论文被查阅和借阅。
\item 学校可以将学位论文的全部或部分内容编入有关数据库进行检索,可以采用影印、缩印或扫描等复制手段保存、汇编学位论文。同时授权清华大学“中国学术期刊(光盘版)电子杂志社”用于出版和编入CNKI《中国知识资源总库》,授权中国科学技术信息研究所将本学位论文收录到《中国学位论文全文数据库》。
\end{enumerate}
(保密的学位论文在解密后适用本授权书)
}
\vskip1.5cm
{\parindent0pt\normalsize\songti
学位论文作者签名:\hskip4.2cm\relax%
导师签字:\relax\hspace*{1.2cm}\\
签字日期:\hskip1cm 年\hskip0.7cm 月\hskip0.7cm 日\relax\hfill%
签字日期:\hskip1cm 年\hskip0.7cm 月\hskip0.7cm 日\relax\hspace*{1.2cm}}

%\end{comment}

\newpage 
%\mbox{} 
%\newpage

\pagestyle{plain}
\clearpage\pagenumbering{roman}

% 中文摘要
%%[需要填写:中文摘要、中文关键词]%%
\begin{center}
  {\sanhao[1.5]\heiti\oucctitle\\\vskip7pt 摘\hspace{1em}要}
\end{center}
{\normalsize\songti

  \indent
作为“太阳活动的反光镜”和“大气扰动的放大镜”,电离层研究对于学术研究有重要的意义,同时也有非常大的应用价值。当今电离层探测技术在不断的提高,目前全世界观测记录电离图的数字测高仪的数量也越来越多,使每年至少观测记录400万张以上的电离图,并且数量仍将持续增加。目前电离图的判读还是靠专业人员进行度量,这种传统的人工度量方法费时、费力、实时性差,且存在度量者主观因素造成的度量差异,已逐渐不能满足实际要求。随着人类对空间科学的研究不断深入,对电离层的长期、实时监测及预报要求的日趋迫切,因此垂测电离图的自动判读研究具有重要的意义。

本文总结了电离图人工度量经验及电离图E区描迹类型与特点,基于图像处理技术提出了垂测电离图E区描迹自动判断算法。主要工作包括:

\begin{enumerate}
    \item系统介绍了电离层结构及电离层垂直探测相关知识,给出了需要自动度量的电离图参数定义,并对国内外电离图自动判读算法进行总结分析,为本文方法的研究奠定了重要的基础。根据对电离层基础内容的分析,明确了本课题的工作目标及意义,并提出了本文判读方法的整体流程。算法主要分为电离图E区F区分割、E区描迹类型识别与参数度量两部分。
    \item电离图E区F区分割。本文根据原始电离图上的描迹分布特点对电离图进行了去除垂线噪声和阈值分割预处理。根据垂测电离图E区F区之间存在谷区的特点,利用水平投影积分法对预处理后的二值电离图进行零区查找,并结合电离图人工度量知识、参数经验值及零区特征,判断零区的形成原因,从中选择最优零区的中线作为垂测电离图E区和F区的分割线。
    \item 电离图E区描迹类型识别及参数度量。由于电离图E区描迹没有规律性,种类多并且还存在不同程度的扩散,因此设计E区电离图描迹判读算法中,本文全面地考虑了电离图E区上的11种Es层类型的描迹。E区电离图的自动判读算法主要包括以下步骤:1)根据描迹形成特点,提出了基于密集点查找的描迹自动检测算法,对存在描迹扩散现象和噪声影响的电离图都有很好的效果;2)为了对描迹的扩散程度进行评价,本文提出了扩散密度的概念,并利用扩散密度、区域宽度等特征将描迹区域分为三类;3)根据电离图上描迹区域的类型将电离图分为五类,并分别设计描迹判读算法。对于无描迹电离图直接给出度量结果;对于复杂电离图进行标记由人工进行度量;对于剩余三类电离图,根据描迹特征,结合人工度量经验,运用图像处理技术提出了描迹分割、去除Es层多次反射、描迹特征提取算法。4)根据度量规则,确定描迹类型并获取E区电离图的相关参数。
\end{enumerate}

利用不同地区采集的大量电离图对本文算法进行实验验证,实验结果表明了本文算法的有效性和可行性。

}
\vskip12bp
{\xiaosi\heiti\noindent
关键词:\hskip1em 图像处理;图像分析;垂测电离图;自动判读}

\newpage 
%\mbox{} 
%\newpage

% 英文摘要
%%[需要填写:英文摘要、英文关键词]%%
\begin{center}
  {\sanhao[1.5]\heiti\oucetitle\\\vskip7pt Abstract}
\end{center}
{\normalsize\songti

As the ``mirror of solar activity'' and ``magnifier of atmospheric disturbance'', ionospheric research has a very important academic significance and socio-economic value. With the development of ionosphere detection technology, the number of the ionosondes which is used to record ionograms is increasing in the whole world, and the annual number of ionograms is more than 400 million. The number of ionograms will continue increasing. But ionospheric characteristic parameters are still mainly obtained by the well-experienced operators, which is not only time-consuming and laborious, but also can not avoid the scaling error caused by the subjective factors. So it can not meet the practical needs. With a better understanding space science, the requirement to long and real time monitoring  and forecast of ionogram becomes urgent, so the automatic scaling of the ionogram is significant.

Through summarizing the artificial experiences and feature  of E Region trace, a new method for scaling the E region traces in ionogram based on the image processing is proposed. The work mainly contains:


\begin{enumerate}
\item  The background information of ionospheric sounding are introduced, which includes the definition of the ionospheric parameters in ionogram and the summary of research status, laying foundation for the research on scaling algorithm. According to the analysis of ionospheric basis knowledge, we confirm the research content and research sense of this subject and propose the overall flow of algorithm. The proposed algorithm includes two steps: segmenting E region image and scaling E region ionogram. 
\item Segmenting E region image  from ionogram. We do the preprocessing operations includes removing the vertical noise and threshold segmentation based on the distribution characters of the trace on the ionogram.  Because there is the blank valley between the E region and F region, we use the horizontal projection method to find the zero area in pre-processed binary image,and find the cause of the zero area in ionogram base on the artificial measurement knowledge, experimental reference data and zero area's features. Then we choose central line of the optimal zero area as the division line of E region and F region. 
\item  Scaling the E region ionogram. The traces in the E region have diffusing phenomenon and many types of trace. In the process of designing E region automatic scaling algorithm, we consider 11 Es types of trace. The E region automatic scaling algorithm mainly contains: 1)Trace detection algorithm based on dense point searching is proposed for effective trace extraction of E region from the ionograms with spread phenomenon and noise. 2) In order to evaluate the diffusion degree of the trace, we put forward the concept of diffusion density.  The traces are divided into three classes based on some features, such as diffusion density, width of the area. 3) We divided the ionogram into five classes.  If there are no trace on the ionogram, we can get the scale result directly;  The complex ionogram is marked automatically, and it will be scale artificially; For other three classes ionograms, the algorithms of trace segmentation, removing multiple reflections, feature extraction, trace identification are designed based on the image processing and artificial measurement experience. 4)Finally we obtain the related parameters of E region according to the scaling rules.

\end{enumerate}

In this thesis, we have performed a series of experiments to confirm the effectiveness and feasibility of the method for scaling the E region traces in ionogram.
}
\vskip12bp
{\xiaosi\heiti\noindent 
\textbf{Keywords:\enskip Image processing; Image analysis; Ionogram; automatic scaling}}
%%%%%%%%%%%%%%%%%%%%%%%%%%%%%%%%%%%%%%%%%%%%%%%%%%%%%%%%%%%%%%%%%%%%%%%%%%%%%%%%
%\newpage 
%\mbox{} 
%\newpage

% 设置 PDF 文档的作者、主题等属性
\makeatletter
\thu@setup@pdfinfo
\makeatother
%\makecover

% 目录
\tableofcontents

% 符号对照表
%\begin{denotation}

\item[HPC] 高性能计算 (High Performance Computing)
\item[cluster] 集群
\item[Itanium] 安腾
\item[SMP] 对称多处理
\item[API] 应用程序编程接口
\item[PI]	聚酰亚胺
\item[MPI]	聚酰亚胺模型化合物,N-苯基邻苯酰亚胺
\item[PBI]	聚苯并咪唑
\item[MPBI]	聚苯并咪唑模型化合物,N-苯基苯并咪唑
\item[PY]	聚吡咙
\item[PMDA-BDA]	均苯四酸二酐与联苯四胺合成的聚吡咙薄膜
\item[$\Delta G$]  	活化自由能~(Activation Free Energy)
\item [$\chi$] 传输系数~(Transmission Coefficient)
\item[$E$] 能量
\item[$m$] 质量
\item[$c$] 光速
\item[$P$] 概率
\item[$T$] 时间
\item[$v$] 速度
\item[劝  学] 君子曰:学不可以已。青,取之于蓝,而青于蓝;冰,水为之,而寒于水。
  木直中绳。(车柔)以为轮,其曲中规。虽有槁暴,不复挺者,(车柔)使之然也。故木
  受绳则直, 金就砺则利,君子博学而日参省乎己,则知明而行无过矣。吾尝终日而思
  矣,  不如须臾之所学也;吾尝(足齐)而望矣,不如登高之博见也。登高而招,臂非加
  长也,  而见者远;  顺风而呼,  声非加疾也,而闻者彰。假舆马者,非利足也,而致
  千里;假舟楫者,非能水也,而绝江河,  君子生非异也,善假于物也。积土成山,风雨
  兴焉;积水成渊,蛟龙生焉;积善成德,而神明自得,圣心备焉。故不积跬步,无以至千
  里;不积小流,无以成江海。骐骥一跃,不能十步;驽马十驾,功在不舍。锲而舍之,朽
  木不折;  锲而不舍,金石可镂。蚓无爪牙之利,筋骨之强,上食埃土,下饮黄泉,用心
  一也。蟹六跪而二螯,非蛇鳝之穴无可寄托者,用心躁也。\pozhehao{} 荀况
\end{denotation}



%%% 正文部分
\mainmatter

%%% Local Variables:
%%% mode: latex
%%% TeX-master: t
%%% End:

\chapter{绪论}
\label{cha1}

\section{课题研究背景及意义}
\label{1_1}

电离层是地球磁层的内界,由于它会影响无线电波的传播,同时有望成为自然灾害的预警器,因此对电离层的监测及研究对科技发展和人类生活都有重要意义。电离层的扰动会对电离层的飞行器、无线电通信、导航定位系统、雷达探测系统造成一定的影响。此外,研究表明,地球上的自然灾害例如火山爆发、地震、海啸、剧烈的台风和雷暴等会引起电离层的扰动\cite{liu2012seismo}\cite{maruyama2011ionospheric},因此,电离层研究对于学术研究有重要的意义,同时也有非常大的应用价值\cite{luque2009emergence}\cite{engwall2009earth}\cite{brambles2011magnetosphere}。
      
目前我国的信息化速度在高速发展,卫星通信、卫星导航和星载雷达系统已经被广泛的应用于人们的生活中,同时也用于我国的军事建设领域。现代的军事战争是高科技的战争也是信息化战争,基于空间技术的各种信息化军事平台和装备在信息获取、传输和使用中有重要作用\cite{许正文2005电离层对卫星信号传播及其性能影响的研究}。电离层约是地面60km以上到磁层顶之间的整个空间。电离层既可以反射来自地面的无线信号,同时也可以改变穿过电离层的高频电波的传播方向、相位、振幅及偏振状态、速度等状态\cite{李建胜2011电离层对雷达信号和导航卫星定位影响的分析与仿真研究}。通过观察电离层对星载无线电系统的影响可以知道空间天气对信息系统会产生影响,也可以利用信息系统的变化来辅助预测天气情况。电离层的强烈扰动异常会影响GPS定位、信号质量下降、天基雷达成像质量降低、通信异常。因此,电离层的实时监测和研究对保障无线电通信的可靠运行、提高卫星定位系统的精度、促进军事的信息化建设都有重要意义。     
     
同时,多项研究表明地震与电离层的异常活动存在一定的关联性,在地震前后电离层都会出现不同程度的扰动\cite{pulinets2004ionospheric}\cite{liu2004pre}\cite{maruyama2011ionospheric}。国际上几十年的观测研究表明,地震与电离层的异常活动之间存在一定的关系。文献\cite{maruyama2011ionospheric}\cite{liu2012seismo}提到了日本福岛和中国四川省汶川县分别发生了九级和八级的强烈地震都导致了电离层发生了激烈并且异常的变化。经过几十年的研究,科学界已证明,除地震外,我们熟知的海啸、飓风等现象都会导致电离层发生变化。所以在一定程度上可以通过监测电离层是否发生异常变化,来预测是否会发生严重的自然灾害。
          
人类从1899年就开始了电离层的相关研究,对电离层研究的主要方法是利用电波在等离子中的各种效应进行电离层探测。世界各国已经有了各种电离层探测方法,出现最早、使用最广泛的是垂直探测\cite{reinisch2009new}\cite{galkin2009accuracy}。
     
电离层垂直测高仪的原理是从地面向电离层发射具有不同频率的脉冲波,脉冲波经过反射回到地面,垂直探测仪在地面接收脉冲波的反射回波,同时记录在这一过程中所用的时间,由此得到我们熟知的以频率为横坐标、以虚高为纵坐标的电离图, 通过对电离图上的描迹的类型识别与参数度量,可以获得电离层的特征参数,再对大量电离的特征参数进行反演计算可以得到电子浓度剖面。研究者通过它能够了解被探测区域的电离层情况,这有利于更进一步对电离层的各种特性进行探索和对可能产生的电离层异常扰动进行预报。因此垂测电离图的自动度量算法研究具有重要意义,所以一直受到科研工作者的关注。
     
考虑到电离层监测对于通信和自然灾害预警的重要性,目前全世界已有数百台数字测高仪在连续的观测记录电离图,平均每天可达到1.1万张,同时世界各地对电离层研究的趋势仍在持续上升,并且随着信息处理技术的不断发展,电离层测高仪及其得到电离图的数量将会持续增长\cite{galkin2011global}。我国已在北京、广州、西安、新疆、青岛等不同地区建立了电离层观测站。目前对于垂测电离图的识别与度量主要还是由专业的度量人员依靠度量知识与经验人工完成度量工作。传统的人工度量方法,存在很强的主观性,即不同的度量者对于同一张电离图会得到不同的度量结果,为以后的电离层反演研究带来一定的误差。此外,由于目前的人工度量人员有限,无法实现24小时实时度量。综上所述,人工度量电离图的方式已不再满足现在的电离图研究需求。随着人类对空间科学研究的不断深入,研究者想要获得大量的、准确的电离图度量结果,因此垂测电离图的自动判读研究具有重要的意义。     
   
\section{国内外研究现状}
\label{1_2}

从二十世纪60年代至今,国内外科研工作者在垂测电离图自动判读方面做了很多工作,也取得了较大的进展。目前垂测电离图自动判读主要思路包括以下步骤:去除电离图噪声、查找及判识描迹、获取电离层相关参数。
      
首先,去除电离层测高仪探测数据中无用的数据信息;其次,根据电离层特性确定描迹信息,并去除噪声;再次,根据电离层描迹特征提取各层描迹;最后,根据URSI\cite{piggott1972handbook}国际判识标准运用曲线拟合等算法获取电离层相关参数。电离图的自动度量算法研究思路是相同的,但是不同研究者从不同的学科领域出发来完成电离图自动判读算法的每个步骤。
     
 目前应用比较广泛的电离图自动方法有以下三种:ARTIST\cite{galkin2008new}、Autoscala\cite{scotto2012automatic}和ESIR\cite{sojka2009sounding}。
        
美国洛威尔大学提出的ARTIST方法具有识别率高、应用广泛等特点\cite{reinisch1983automatic}\cite{reinisch2005recent}。此方法通过对每个反射回波设定阈值来去除电离图数据的噪声;然后利用特定角度的边缘点检测算法和人工神经网络进行描迹提取,最后利用曲线拟合法对电离图描迹进行识别与度量。该方法与UMLCAR 研制的 DPS 系列测高仪配合使用,可以得到较高的度量精度,但该算法用于中国电波传播研究所所用的测高仪并不能取得较好的效果。
      
意大利学者提出了Autoscala方法。该方法运用 correlation 技术结合曲线拟合方法对F2 层\cite{pezzopane2007automatic}\cite{scotto2008removing}\cite{pezzopane2010highlighting}、Es 层\cite{scotto2007method}和 F1 层\cite{pezzopane2008method}描迹分别进行描迹判识、描迹拟合和电离层参数获取,其优点是可忽略极化信息,利用长期的电离图度量所产生的经验值来判断电离图上是否存在某层描迹。
      
美国空间环境公司(SEC)研发的电离图自动度量方法为ERIS\cite{rice2009expert}\cite{sojka2009sounding}。该方法的特点是通过利用电离层的物理模型将原始电离图转换为ESIR 格式,然后对描迹数据的完整性和电离层是否为非常规的复杂电离图进行判断;如果电离图时常规电离图,基于模式识别技术对电离图进行度量。

此外,澳大利亚的Fox和Blundell\cite{fox1989automatic}也提出了电离图自动判读方法。该方法主要运用曲线拟合和外推法来提取描迹,并提取识别描迹类型需要的各类特征,同时判断原始电离图是否与描迹图一致,实现对电离图各层描迹的判识,最合根据URSI标准利用外推法获取电离图参数;但是该算法仅能用于只记录O波的测高仪。日本的研究者\cite{igi1993automatic}也对电离图自动判读做了相关研究,该方法主要运用了数学和模式识别技术对电离层各层参数进行自动判读,该方法的最大特点是不用区分寻常波与非寻常波。此外,台湾学者Tsai 和 Berkey 等\cite{tsai2000ionogram}、 俄罗斯的Pulinets\cite{pulinets1995automating} 和土耳其的 Arikan 等\cite{kan2002new}都对电离图自动判读研究做了有意义的工作。
      
近几年来,我国学者在电离图自动判读算法中做了大量的研究。中国科学院的宁百齐、丁宗华等\cite{丁宗华2006电离层频高图自动度量与分析的方法研究}\cite{ding2007automatic}\cite{丁宗华2007电离层频高图参数的实时自动度量与分析}通过对电离层的电子浓度剖面进行 Empirical Orthogonal Function分解的得到经验正交函数集,通过对经验正交函数集进行反演计算,得到对应的电离图描迹,并与探测到的电离图数据进行对比,找到与原电离最接近的描迹,获取电离图的度量参数。中国学者柳文等\cite{柳文2009基于}也提出了电离图自动判读算法。首先利用IRI模型反演电离图E层描迹,同时借鉴国外方法对F层描迹进行搜索,同时结合电离图的参数经验值,从搜索到的描迹中,选择有效描迹并进行判读。此外,中国电波所的研究者\cite{凡俊梅2009电离层斜向传播模式的智能识别}和哈尔滨工业大学的邓维波等\cite{李雪2010返回散射电离图智能判读}分别对斜测电离图和返回散射电离图的自动判读进行了相关研究。
 
综上,研究者对电离图自动判读方法的研究主要根据电离层的物理特性,利用电离图人工度量经验和度量参数经验值,运用数学等领域的方法对描迹进行自动提取和判读。目前存在的很多电离图自动判读方法基于电离图度量参数的经验值提出、依赖于电离层的物理模型\cite{bilitza2008international}、同时很多判读方法只适用于特定的测高仪,因此这些方法想要被广泛的推广和应用还存在一定的困难。
     
随着图像处理和模式识别领域的发展,为电离图的自动判读提供了新的思路。我们将电离图描迹的频率、虚高及其信号强度可以由电离图灰度图像的位置和灰度值表示。基于电离层的分层结构及各层描迹形态特征,运用图像处理、图像分析技术,结合电离图的解释与度量标准,实现电离图上的描迹检测与识别。
     
%=============================================================================================================================
\section{课题来源}
\label{1_3}
本文课题来源于青岛市科技发展计划项目“基于图像分析的垂测电离图自动判读研究”(批准号:13-1-4-223-jch)和中央高校基本科研业务费项目“电离层垂直探测频高图自动解释及度量”(批准号:201313011)。 本文利用中国电波传播研究所提供的中国不同地区不同时间的五万张电离图为数据集,进行了电离图E区描迹自动判读方法的研究。

%=============================================================================================================================
\section{主要工作及安排}
\label{1_4}

第一章:绪论。主要介绍了电离图自动度量的研究背景、意义及国内外的研究进展。
     
第二章:电离层及垂测电离图。包括电离图的形成原因及其特征、电离层的主要探测方法、垂测电离图及其人工度量经验,以及本文的整体方案流程。在电离层探测技术的介绍中重点介绍了垂直探测电离层技术。
     
第三章:垂测电离图分割。主要包括将垂测数据转换为电离图、电离图预处理、电离图E区F区分割。
     
第四章:E区电离图类型识别及参数度量。根据电离图E区描迹的人工度量步骤,提出了电离图E区描迹的自动度量算法,算法主要包括以下几个步骤:电离图E区描迹检测、基于扩散程度的描迹区域分类、E区电离图分类、描迹分割、去除Es层描迹多次反射、描迹特征提取、识别描迹类型并获取相关参数。
     
第五章:实验分析。运用本文的方法对中国大量电离图进行实验,并对结果进行分析。
     
第六章:总结与展望。对本文提出的电离图E区描迹自动度量方法的进行总结,并针对方法存在的问题和不足对课题进行展望。
%=============================================================================================================================




%%% Local Variables: 
%%% mode: latex
%%% TeX-master: t
%%% End: 

\chapter{电离层及垂测电离图}
\label{cha2}

 本章主要介绍电离图自动度量的基础,包括电离层的形成及其特征、造成电离层扰动的原因、电离层的分层结构及各层的特征。同时本章也介绍了电离层探测方法、垂测电离图及电离图的人工度量经验。只有对电离层的形成原因和人工度量步骤有充分认识,并亲自度量大量电离图,才能设计出对各类电离图都适用的自动度量算法。

%============================================================================================================
\section{电离层的形成及其特征}
\label{2_1}

电离层是地球大气的一个电离区域。由于电离层受到了太阳高能辐射以及宇宙线的激励而发生电离产生的大气高层。极短紫外辐射(EUV)和X射线导致电离层的原子和分子电离为正离子、负离子和自由电子,形成电离层 ,但是正负离子和电子所带电荷量相互抵消,电离层总体上是不带电的。X射线会随着太阳活动不断变化,主要使电离层D层和E层发生电离。源自太阳的EUV比较稳定,但它随太阳黑子数量的变化存在相应的月变化和年变化。
     
太阳极短紫外线辐射会被原子和分子吸收,同时也会使原子与分子发生电离现象。随着太阳辐射穿透更深层的地球大气,它的强度也在减弱。电子产生速度与EUV强度和大气密度成正比。在大气层的顶部,尽管EUV强度高,但大气密度很小,因此电离程度比较弱。在E层底部大气密度很大,但EUV强度很小,产生的离子和电子也很少。介于E层和大气顶部之间的区域,会出现最大电子浓度,形成了某一电离层。
%-------------------------------------------------------------------------------------------------------------
\section{电离层扰动来源}
\label{2_2}

太阳活动和地磁活动是造成电离层扰动的主要原因\cite{徐彤2009中低纬电离层模型及其异常现象相关研究}。 此外,地震等自然灾害也会引起它的突变。
      
EUV和X射线是电离层中的空气发生电离的主要作用。由于X射线会随着太阳耀斑迅速增长且不规律变化,同时极短紫外辐射也会随着太阳黑子数量的变化发生相应的月变化和年变化,太阳辐射强度的变化会影响电离层中分子和原子被电离的程度。所以太阳活动会使电离层具有周期的日变化、日变化、年变化等特征。对于日变化,在白天,太阳辐射强,大气分子的电离速度快,电离层电离密度增加;接近晚上,让太阳辐射逐渐减弱,电离层电离密度减小。
       
地球自身的地磁活动会引起电离层扰动。出现磁暴现象时,磁层的高能粒子由于磁场的作用向下沉降,在这过程中会使中性大气升温,发生膨胀,引起大范围的电离层扰动,形成电离层暴。

%-------------------------------------------------------------------------------------------------------------
\section{电离层结构组成及各层特征}
\label{2_3}

电离层结构可以由电离层相关参数的空间分布来表示,比如:电离层中的电子、离子在单位体积内的个数或电离层中电子和离子温度等。在电离层的研究中主要利用电子密度(单位体积的自由电子数)的空间分布来表示电离层的分层结构。电子密度数值的大小与EUV和X射线的强度和大气密度有紧密的联系。
       
由于重力的作用地球大气在不同的高度上有不同的大气密度,同时不同高度上太阳辐射强度也不相同。如图~\ref{图2_1}为典型的电离层电子密度剖面图,图中的两条曲线分别为黑夜和白天的电子密度分布情况,其中纵坐标表示距离地面的高度、横坐标表示电子密度的大小。电离层电子浓度随高度变化,存在几个极大值区,称之为层。白天时电离层划分为D层、E层、F1层和F2层,晚上时只存在F2层(高度大约在200km以上),这时通常称为F层。
\begin{figure}[h]
  \centering
  \includegraphics[height=6cm]{图2_1}
  \caption{典型的电离层电子密度剖面图}
  \label{图2_1}    
\end{figure}

D层离地面60~90km高度范围内。X射线使D区的大气分子和原子发生电离现象,而形成了电离层D层。大多数高频电路的吸收几乎全部发生在D层。D层的电子浓度最大时也是吸收最大时。大规模的太阳耀斑爆发期,太阳辐射的X射线迅速增加,造成D层区域电离速度增加,电离层吸收也迅速增加。
       
E层高度从D层一直延伸到150km左右,E层的等离子体能够满足光化学平衡,是中等浓度的分子离子层。E层临界频率存在日变化规律,在中午时,太阳的仰角变大,电子产生的速度最大,这时E层的临界频率达到最大值。同样E层电离层还存在季节变化,夏季的太阳天顶角比冬季小,因此夏季入射到电离层的太阳辐射强度比冬季强,电子产生的速度变大,夏季时E层的临界频率也达到最大。 在晚上时,太阳辐射消失时,E层等离子体的在复合作用下快速消失,E层消失。因此电离层E层描迹仅在白天出现。
       
F层描迹在150~1000km的高度范围内,是电离层的主要组成部分,由于其电子密度在电离层中最大,因此它会对无线通信有重要意义。 F层白天F层分为F1层和F2层,F1层仅在白天出现。F1层和F2层具有不同的特性。
       
F1层受到光化学作用在白天出现,晚上消失。F2层在白天和晚上都存在。从白天到晚上,F1层和E层描迹在逐渐消失,而F2层描迹却始终存在。一个使$F2$层一直存在的原因是中性气体中的大气层风。在白天,低纬度地区的上层大气被太阳加热,约300千米高的大气层风吹向高纬度(例如极区)。中性气体的风沿水平方向吹,但离子和电子由于地球磁场的原因不能越过磁力线。然而,白天离子和电子趋向于沿磁力线吹往更低高度,在那里它们在再结合的过程中被消耗。夜间,低纬度的上层大气变冷,大气层风吹向赤道。离子和电子沿磁力线向上吹,在那里再结合发生的很慢。因此,F2层在夜间一直存在。
      
Es层(Sporadic E)是偶发E层,即E区的突发不均匀结构,称为Es层,它的厚度约为100米到2千米之间,水平尺度为200米到100千米之间,有时候Es层可能在更大的范围内延续。一般来说Es层是一个薄层,出现时间不确定。有时Es层会呈现不透明状而遮住更高的层。电子密度很高,反射无线电信号的能力很强。同时电离图Es层描迹会出现扩散现象,国际电离图度量标准制定义了多个Es层描迹。电离图Es层描迹的出现时间和描迹类型不定。
      
扩展F层是F区的突发不均匀结构。经常出现在极光椭圆区和地磁赤道区的夜间出现。它的扩散特性会使穿过它的无线电信号发生衰减。
    
在电离层中,地球磁场会使无线电波分离为O波(寻常波)和X波(非寻常波)。O波与X波在传播过程中是相互独立的,在电离图上会出现O波描迹与X波描迹。由电磁理论知识,电离层O波与X波之间存在频率差为半个磁旋频率。磁旋频率取值与地点的纬度有很大的关系,我国大部分地区都属于中纬度地区,中纬度地区的磁旋频率为1.2MHz。
     
%-------------------------------------------------------------------------------------------------------------
\section{电离层探测方法}
\label{2_4}
电离层探测技术的发展对于电离层领域的研究具有重要意义,同时也是电离层其他领域研究的基础。人类对电离层的研究从1924年12月Sir E. V. 阿普尔顿和巴尼特用连续波法进行电离层高度的实验,同时证明了电离层的存在。随后1925年布雷特和图夫发明了电离层垂直探测装置。在1949年人们开始利用火箭进行对电离层的直接探测实验,在1957年人们开始利用人造卫星在高空对电离层进行直接探测。随着科学技术的发展,探测方法在不断的建立和完善,电离层探测技术是利用电离层会对射入其空间的电磁波发生反射、吸收、散射、多普勒频移等。利用测高仪采集某个电离层区域的数据,可以获取电离层相关物理参量。
      
电离层的探测方式可以分为:直接探测、间接探测\cite{林晨2004软件无线电在电离层测高仪中的应用研究}。直接探测是指将装有探测仪器的火箭或卫星发射到电离层,探测仪器在电离层直接探测电离层的相关信息。间接探测是指在地面利用探测仪器接收电离层信息来获取电离层的相关特性。间接探测又可以分为主动和被动。间接主动探测是指利用探测仪主动的向电离层发送电磁波,同时探测仪器会在地面接收经电离层反射回来的电磁波,并记录相关数据,通过对探测仪纪录的数据进行处理,分析电离层的相关特性。间接被动探测是指探测仪不会主动的向电离层发送电磁波,而是利用探测仪接收来自电离层自身辐射,并记录相关数据,来分析电离层特性。
     
被动探测是指利用探测仪器接收电离层自身的辐射,通过计算电离层的特征信息。主动探测是指使用高频雷达向电离层主动发射电磁波,然后通过分析处理回波信号得到电离层的相关特征参数。

主动探测方法主要包括垂直探测、斜向探测、返回散射探测。尽管目前有众多的电离层探测技术,但垂直探测技术依然是最主要的电离层探测方法。
     
1925年人类发明了第一台电离层垂直探测仪,电离层垂直测高仪实质上是一种高频雷达。垂直测高仪垂直向上发射电磁波,当射入到电离层的电磁波频率满足其反射条件时,入射电磁波将被电离层反射回地面,探测仪通过纪录发射和接收脉冲之间的时间延迟$\tau$,获取电离层的高度信息$h'$。其中,由于电磁波在大气中传播时受到时延、折射、衰减等因素的影响,电磁波的传播速度小于光速,因此利用式~\ref{式2_1}计算出的电离层高度大于电离层的实际高度,所以称为虚高$h'$。
 \begin{equation}
h'=\frac{1}{2}c\tau
\label{式2_1}
\end{equation}
        
通过改变发射电磁波的频率可以得到不同频率上的反射时间延迟$\tau$的纪录,当发射频率$f$在整个短波波段范围(一般为0.5~30MHz,TYC-1型电离层探测仪为1~32MHz)内以一定的频率作为步长进行持续改变,就能纪录到虚高(发射时延$\tau$)随发射频率$f$变化的曲线图,即垂测电离图(频高图)。

电离层的倾斜探测是指在地面上用探测仪倾斜地向电离层发射电磁波,并在距发射点一定距离的位置接收反射回波,并记录信号在传播过程中的时间、频率信息。
  
电离层返回散射的探测,是由探测仪倾斜地向电离层发射电磁波,电磁波电离层返回地面,返回地面的电磁波向多个方向散射,接收器只接收散射后返回发射点的电磁波,并记录电磁波的散射回波的幅度、时延、频率等信息。返回散射探测的优点是可以获取较广范围内的电离层特性,这种方法经常用于检测短波无线电覆盖区、检测运动目标及探测电离层结构。
%-------------------------------------------------------------------------------------------------------------
\section{垂测电离图简介}
\label{2_6}

如图~\ref{图2_3}是中国电波传播研究所研制生产的第四代(TYC-1型)全自动数字式电离层垂直探测电离图的人工度量软件工作界面。垂测电离图自动度量软件读取的是我国某第在2010年2月26日15时探测得到的电离层数据。电离图的横轴代表扫频频率,频率变化范围为1MHz~20.2MHz,垂直轴方向为虚高,虚高的范围为0km~800km。电离图上的每个像素点代表一个回波信号点,信号点列坐标为信号点频率值,像素点行坐标为信号点的虚高值,信号点灰度值代表信号点的强度。软件的左下方共有14个电离图特征参数需要人工度量;界面的右下方可以对电离图上信号显示强弱进行调整。 电离图共有十四个参数需要进行度量:$fmin$、$foE$、$h'E$、$foEs$、 $fbEs$、 $h'Es$、  $foF1$、$fbEs$、$h'Es$、 $foF1$、$h'F$、$M3F1$、$h'F2$、$foF2$、$M3F2$、$fxI$、Es type。

\begin{figure}[h]
\centering
\includegraphics[height=6cm]{图2_3}
\caption{垂直探测电离图人工度量软件工作界面}
\label{图2_3}    
\end{figure}

$fmin$:表示在电离图所有描迹的最小频率;

$foE$:表示E区最低厚度的寻常波临界频率;

$h'E$:表示E层描迹的最低虚高;

$foEs$:表示Es层连续描迹寻常波分量的顶频;

$h'Es$:表示Es层描迹的最低虚高;
 
Es type:Es描迹共有11种类型,有$f$、$l$、$c$、$h$、$q$、 $r$、$a$、$s$、$d$、$d$、$n$、 $k$。在某个站点并不能观测到所有的Es类型,在中纬度经常出现的是$f$、$l$、$c$和$h$型,而在高纬度地区经常出现的是$a$、$r$型。

$fbEs$:是Es层的遮蔽频率,即是Es层允许从上面层反射的第一个频率。$fbEs$表示Es层透明度的度量。$fbEs$是由通过Es层观测到较高层寻常波分量的最低频率决定。

$foF1$:是F1层寻常波的临界频率。

$h'F$:是F1层寻常波的最低虚高。
  
$h'F2$:是F区最高稳定层寻常波描迹达到的最低虚高。

$foF2$:是F区最高稳定层寻常波临界频率。

$fxI$:可接收到的F层反射的最高频率,是关于F层散射的一个标志。

M因子:是一个从垂测频率获得给定距离斜向传播最大可用频率的转换因子,以3000千米作为标准传播距离的M因子,称为M(3000),它通常要附加上反射层的名称来表示,如$M(3000)F2$或$M(3000)F1 $。

目前,全世界每天有200多个电离层观测站在不间断的探测电离图,国内主要的观测站在北京、青岛、重庆、兰州、拉萨、满洲里、乌鲁木齐、长春、广州、海口、昆明、苏州、新乡、西安、伊犁、阿勒泰等城市。随着每天各地采集的电离图数量在不断增加,传统的人工度量电离图的方法已经不能满足现实的需求。电离图F区描迹的形状和变化相对比较固定,目前已存在比较成熟的垂测电离图自动度量算法。电离图E区描迹是由E层、Es层、E2层构成,Es层描迹的主要特点是出现没有规律,种类多同时还存在不同程度的扩散。所以,本文就垂测电离E区描迹的识别并对 $foE$、$h'E$、$foEs$、 $h'Es$、$fbEs$、 Es type 这些电离图E区相关参数的自动度量进行相关研究。

 %-------------------------------------------------------------------------------------------------------------
\section{垂测电离图E区描迹人工度量经验}
\label{2_7}
       
人工对电离图E区描迹的识别主要是根据描迹形状、位置等特征。电离图E区描迹主要包括E层、Es层、E2层,Es层描迹有$c$型、$h$型、$f$型、$l$型、$q$型、$a$型、$d$型、$k$型、$r$型、$n$型11种类型。下面对电离图E区常出现的各类描迹做简单介绍:
       
E层根据描迹形态可以分为典型E层、未观测到的E层、E层被Es层遮蔽、E层底部不水平、E层临频附件的衰减、微粒E层等几种情况。E层描迹完整的情况存在时延、时延是由各种电磁波在电离层中传播造成的。如图~\ref{图2_4}为典型的E层描迹。E层描迹的形状是上翘型,即描迹随着频率的增大,虚高呈现不断增大。E层描迹的最低虚高在115km~125km,E层O波描迹的最大频率小于$fminF$,仅在白天出现。

\begin{figure}[h]
\centering
\includegraphics[height=5cm]{图2_4}
\caption{E层描迹}
\label{图2_4}    
\end{figure}   
      
Es层描迹种类较多,共有11种类型,且每个站点观察到的描迹并不相同,在我国经常出现的有$l$型、$c$型、$h$型、$f$型、$n$型。在我国每个站点观测到的Es层类型与站点的位置有关。在低纬度、中纬度、高纬度地区经常出现的Es层类型不同。下面将详细地介绍各种Es层描迹的特征。
\begin{enumerate}     
\item Es层$c$型:也称为尖角型描迹,虚高在115km~125km之间,仅在白天出现。Es层$c$型的起始频率小于或等于$foE$。描迹形状为下降型,即描迹随着频率的增大,虚高有减小趋势,描迹斜率的变化从起始频率开始接近于垂直方向逐渐下降,然后趋于直线型。通常Es层$c$型描迹和正规的E层描迹相连(如图~\ref{图2_5}),E层和Es层$c$型描迹虚高相近。部分Es层$c$型描迹也会在没有E层描迹的情况下单独出现。
\begin{figure}[h]
\centering
\includegraphics[height=5cm]{图2_5}
\caption{Es层c型描迹}
\label{图2_5}    
\end{figure}   

\item Es层$h$型,也称为高型描迹,虚高在130km~180km,仅在白天出现。Es层$h$的起始频率等于或大于$foE$。如图~\ref{图2_6},正规的Es层$h$型描迹的最低虚高与E层描迹的最低虚高相差很大,同时与E层尖角不对称,即Es层$h$型描迹在最小频率处的虚高远高于E层描迹在最大频率处的虚高,即$h'Es$的值明显大于$h'E$。Es层$h$型描迹也会单独出现。
\begin{figure}[h]
\centering
\includegraphics[height=5cm]{图2_6}
\caption{Es层h型描迹}
\label{图2_6}    
\end{figure}   


\item Es层$f$型,也称为平型描迹,虚高在100km~130km,仅在夜间出现。如图~\ref{图2_7},Es层$f$型描迹为直线型,描迹的最低虚高不随频率的变化,同时描迹通常比较粗而浓。Es层$f$型描迹在任何纬度观测站的探测到的电离图中都存在此类Es层描迹。
\begin{figure}[h]
\centering
\includegraphics[height=5cm]{图2_7}
\caption{Es层f型描迹}
\label{图2_7}    
\end{figure}   

\item Es层$l$型:也称为低型,虚高在95km~110km之间,仅在白天出现。Es层$l$型描迹为直线型,描迹的最低虚高不随频率的变化。$l$型描迹的出现存在三种情况:$l$型O波单独出现的情况、$l$型OX波混合的情况、$l$型OX波分离的情况。$l$型O波单独出现和$l$型OX波混合时,这两种情况下的描迹都是没有间断的直线。为了确定度量参数$foEs$的值,对于一条没有间断的直线型描迹,我们要判读描迹属于$l$型O波单独出现的情况还是属于$l$型OX波混合的情况。对于判断OX波是否混合有两种判别方法:第一,如果F层有X描迹,则$ftEs\geq fminFx-fB/2$就认为l型描迹属于OX波混合的情况($fminFx$为F层X描迹的最小频率);否则为$l$型O波单独出现的情况。第二,如果F层X波不存在,但E层存在X波,同时$ftEs\geq fminEx-fB/2$($fminEx$为E层X波描迹的最小频率),就认为描迹为$l$型OX波混合的情况;否则为$l$型O波单独出现的情况。对于图~\ref{图2_8}我们采用上面的第一种判别方法,因为$foEs>fminFx-fB/2$,所以可认为描迹是OX波混合的$l$型。对于$l$型O波单独出现的情况,将描迹的顶频作为$foEs$;对于$l$型OX波混合出现的情况,用顶频减去$fB/2$求得$foEs$。

\begin{figure}[h]
\centering
\includegraphics[height=5cm]{图2_8}
\caption{Es层l型描迹}
\label{图2_8}    
\end{figure}   

\item Es层$q$型:也称为赤道型,扩散的和非遮蔽性的Es层描迹;白天和晚上都有可能出现,常在磁赤道附近的白天出现。如图~\ref{图2_9}所示,$q$型描迹在高度上呈现弱扩散,频率范围延伸的比较大同时没有清晰的下边缘。Es层$q$型描迹也经常与$l$型、$f$型描迹叠加出现。如果描迹的下边缘是一条严格的直线,同时在高度上存在弱扩散,频率范围延伸的比较大,就可认为是$q$型与$l$型、 或$q$型与$f$型叠加出现。

\begin{figure}[h]
\centering
\includegraphics[height=5cm]{图2_9}
\caption{Es层q型描迹}
\label{图2_9}    
\end{figure}   

\item Es层$a$型:也称为极光型,是一种扩散型描迹,描迹的虚高可以伸展到数百公里以上。Es层$a$型描迹的形状特征:外形有一个平缓的或缓慢上升的底边缘。$a$型描迹经常出现在高纬地区和有极光活动的中纬地区。如图~\ref{图2_10}为Es层$a$型描迹。

\begin{figure}[h]
\centering
\includegraphics[height=5cm]{图2_10}
\caption{Es层a型描迹}
\label{图2_10}    
\end{figure}   

\item Es层$s$型:也称为斜型,是一种扩散型描迹。描迹为上升型,虚高随频率稳定地增大。如图~\ref{图2_11}所示,Es层$s$型描迹常与Es层$l$型描迹同时出现,$s$型描迹主要从$l$型描迹的中间开始显露出来。Es层$s$型描迹主要在高纬度地区出现,在磁赤道区也存在比较弱的描迹。此类描迹只需要度量出描迹类型,不需度量参数$foEs$和$h'Es$。

\begin{figure}[h]
\centering
\includegraphics[height=5cm]{图2_11}
\caption{Es层s型描迹}
\label{图2_11}    
\end{figure}   

\item Es层$d$型:也称为D区型,是一种弱的扩散描迹,虚高低于95km,常在85km左右出现。因为Es层$d$型描迹并不是严谨的Es层,不需用来确定$fmin$、$foEs$、$h^{'}Es$。
  
\item Es层$k$型:也称为微粒E层,虚高在100~200km,经常出现在120km~180km,仅在晚上出现。在$k$型描迹出现时,F层的最小频率($fminF$)处描迹上翘,或至少同一时间三个国内站电离图$fminFx$处描迹上翘。常见的$k$型描迹形状为描迹右端上翘,如果在$fminF$的左端存在描迹,描迹的最大频率小于接近$fminF$,认为描迹是Es层$k$型O波描迹。如果在$fminF$的右端存在描迹,且描迹的最大频率接近$fminFx$,就认为描迹是Es层$k$型X波描迹。如图\ref{图2_12}为Es层$k$型O波描迹。

\begin{figure}[h]
\centering
\includegraphics[height=5cm]{图2_12}
\caption{Es层k型描迹}
\label{图2_12}    
\end{figure}   

\item Es层$r$型:也称为时延型,描迹在顶频附近随虚高增大,有小的扩散,$r$型描迹的穿透频率大于F层描迹的最低频率。如果描迹从$fminF$的右边出现,且描迹的最大频率大于$fminFx$,该描迹为Es层$r$型X波描迹。如图~\ref{图2_13}为Es层$r$型O波描迹。
\begin{figure}[h]
\centering
\includegraphics[height=5cm]{图2_13}
\caption{Es层r型描迹}
\label{图2_13}    
\end{figure}
 
\item Es层$n$型:用来表示不能归为以上这些类型的Es描迹。如图~\ref{图2_14}所示,在晚上描迹像$f$型,但是它的虚高不在$f$型描迹虚高的范围内,就认为描迹是$n$型。在白天,描迹为直线型,但它的虚高不在$l$型描迹的虚高范围内,也被认为是$n$型描迹。目前,经常遇到的$n$型描迹主要是以上这两种情况。

\begin{figure}[h]
\centering
\includegraphics[height=5cm]{图2_14}
\caption{Es层n型描迹}
\label{图2_14}    
\end{figure}
\end{enumerate}

E2层:通常出现在日出和日落前后的两个小时,虚高大于125km。E2 层的最大频率小于$fminF$。如图~\ref{图2_15}为常见的E2层描迹,描迹形状上翘。如果E2层在临频处衰减,E2层描迹可能变为一条直线。  

\begin{figure}[h]
\centering
\includegraphics[height=5cm]{图2_15}
\caption{E2层描迹}
\label{图2_15}    
\end{figure}
 
根据上面E区各种类型电离图的介绍,识别描迹主要根据以下特征:虚高、描迹形状、$foE$的经验数据、频率范围、描迹在高度上的离散情况、$fminF$、F层X波的最小频率值、Es层描迹的最大频率值、描迹的斜率变化。识别的过程简要概述为“有没有描迹→时间→从上到下→从左到右→再看反射”。我们将人工度量过程总结为以下几个步骤:
   
首先,判断 E 区内有没有描迹,如果有描迹,再看电离图数据的采集时间;如果没有描迹,E区电离图各项参数为零。
   
其次,电离图的采集时间可分为:白天、过渡期、晚上。 
   
在白天,电离图上可能出现的类型有E、E2、$l$、$c$、$h$、$a$、$q$、$r$($r$型一般都是在晚上出现,这里先不考虑)、$s$、$d$、$n$。下面为白天采集的且存在描迹的E区电离图人工判读过程:
\begin{enumerate}      
\item描迹的最低虚高小于95km,经常出现在80km左右,弱的扩散描迹,则为d型。
      
\item描迹的最低虚高在95km~110km之间,形状为直线型,且描迹不存在扩散现象,就认为描迹类型为Es层$l$型。接着判断描迹的OX波情况,如果直线中间有间断,同时间断处两侧描迹的最大频率相差0.7MHz,就认为描迹为OX波分离的情况。如果描迹是一条直线没有间断,这时我们需要判断描迹是OX波混合,还是O波单独出现的情况,判断的依据是:如果F层有X描迹,同时满足$ftEs\geq fminFx-fB/2$就认为描迹的X波存在,这时描迹为OX波混合的情况,否则描迹为O波单独存在的情况;如果F层没有X描迹,可以看是否存在E层的X波,如果E层的X波存在,同时$ftEs\geq fminEx-fB/2$,就认为描迹的X波存在;否则认为只有O波;当Es呈现全遮蔽(看不到F层),也认为是OX混合。
   
\item描迹的最低虚高在F层以下的高度范围内,形状为直线型,描迹在高度上呈弱扩散,频率范围延伸的比较大同时没有清晰的下边缘,这样的描迹是$q$型。在白天$q$型可能与$l$型叠加出现。判断有没有叠加的方法如下:如果描迹的下边缘是一条严格的直线,同时在高度上存在弱扩散,频率范围延伸的比较大,就可认为是$q$型与$l$型的叠加。
      
\item描迹的最低虚高在110km~125km之间,形状为直线型,描迹的最大频率小于F层的最小频率,同时描迹的最大频率与$foE$的经验数据相差很小,就认为描迹为E层(这是E层在顶频处出现衰减的情况)。
      
\item描迹的最低虚高在110km~125km之间,形状是直线型,描迹的最小频率与$foE$的经验值相接近,就可认为是$c$型(这种情况就是$c$型的低频处出现衰减)。接着再看是O波单独出现、OX混合、还是OX分离,具体的判别标准与$l$型相同。 

\item描迹的形状为直线型,描迹的最低虚高不在以上这些高度范围内,同时描迹也不具备Es层$q$型描迹的特征,那么就认为描迹类型为$n$型Es层描迹。 

\item描迹的最低虚高在110km~125km之间,同时描迹形状为右端上翘型,同时描迹的最大频率不能超过$fminF$,那么描迹为E层。从左向右观察,判断是否存在E层X波描迹。 
     
\item描迹的最低虚高110km~125km之间,如果描迹的形状与E层对称,为左端上翘型,就认为描迹为$c$型。$c$型和E层经常一起出现。如果E层和$c$型都没有衰减,E层的虚高值和$c$型的虚高值基本相等。再从左到右,判断$c$型描迹的OX波情况。 

\item描迹的最低虚高在125km~180km之间,描迹的形状为左端上翘或“U”字型,就认为该描迹为$h$型。如果E区电离图上存在E层,在$foE$处取到的E层描迹的最低虚高与此描迹的最低虚高存在一定的差值。所以对于存在E层的电离图,还可以根据E层描迹的虚高与$h$型描迹的虚高差,来判断$h$型的存在。 

\item如果描迹不属于以上任何一种类型,存在严重的扩散扩散现象,同时在高度上的扩散高度很大,那么就认为该描迹为$a$型。如果描迹存在扩散现象同时描迹下边缘为向上的斜线,则认为该描迹为$s$型描迹。
      
\item如果描迹不属于以上的任何一种类型,就可以认为该描迹为Es层$n$型描迹。 
\end{enumerate}  
   
在过渡期(指从白天到晚上、从晚上到白天过渡的两个时间段),依靠经验数据将过渡期固定在一个具体的时间段是不够精确的,在人工度量过程中,根据F层的描迹的变化情况来确定过渡期的出现时间。在过渡期内。除了上面白天出现的这些类型,还可能出现E2层描迹。对E2层的识别主要根据以下特点:虚高在125km以上,描迹的形状一种右端上翘,描迹的形状另一种像U字型,同时它的最大频率不超过F层的最小频率。如果有E层存在,那E层的最大频率等于E2层的最小频率;如果符合以上这些条件,但描迹形状为直线型,那么描迹为E2层,这是E2层顶频衰减的情况。
      
晚上主要出现的类型有$f$型、$r$型、$k$型、$n$型、$a$型、$q$型、$d$型、$s$型。
\begin{enumerate}       
\item描迹最低高度低于95km,同时为弱的扩散描迹,那么认为描迹为$d$型。 
      
\item描迹的最低高度在100km~130km之间,形状为直线,同时描迹粗而浓,那么描迹类型为$f$型。 然后判断描迹OX波的情况属于那种情况。OX波分离的判断与$l$型相同。$f$型OX波是否混合的判断依据:如果$ftEs\leq fB+0.25MHz$(fB一般取1.4MHz),则该描迹不存在X波,否则存在X波。
      
\item描迹在高度上呈弱扩散,频率范围延伸的比较大同时没有清晰的下边缘,就认为该描迹为Es层$q$型。在晚上可能出现Es层$q$型描迹与$f$型描迹叠加出现。判断有没有叠加方法如下:如果描迹的下边缘是一条严格的直线,同时在高度上存在弱扩散,频率范围延伸的比较大,就可认为是$q$型与$f$型叠加出现。 

\item描迹的形状右端上翘,描迹位于E区,如果描迹位于$fminF$左侧,且描迹的的最大频率大于$fminF$,则该描迹是$r$型描迹。如果描迹位于$fminF$的右侧,且描迹的最大频率大于$fminFx$,就认为描迹是$r$型的X波。存在$r$型,同时如果F层描迹在$fminF$处描迹上翘,认为此时还存在一个$k$型,$k$型描迹被遮蔽。 
      
\item描迹的最低虚高位于100km~200km之间,经常出现在120km~180km之间,$fminF$处描迹上翘,或至少同一时间三个国内站电离图$fminFx$处描迹上翘,描迹形状为右端上翘。如果描迹位于$fminF$左侧,描迹的最大频率小于接近$fminF$,认为描迹是$k$型的O波。如果描迹位于$fminF$的右侧,且描迹的最大频率接近$fminFx$,就认为描迹是$k$型X波。如果存在X波,从右向左可以找到$r$型的O波。
      
\item如果描迹不属于以上任何一种类型,存在严重的扩散扩散现象,同时在高度上的扩散高度很大,那么就认为该描迹为$a$型。如果描迹存在扩散现象同时描迹下边缘为向上的斜线,则认为该描迹为$s$型描迹。
          
\item如果描迹不属于以上的任何一种类型,就认为该描迹为$n$型描迹。
\end{enumerate}
      
最后,根据前面确定的描迹个数及类型,用描迹与它的多次反射描迹的最低虚高呈倍数关系,同时描迹与它的反射描迹形状相似等特征,来确定E区电离图上描迹的反射次数,并度量电离图的相关参数。

%-------------------------------------------------------------------------------------------------------------
\section{本文的整体方案流程}
\label{2_8}
由于电离层时变、色散等特殊性,并受太阳活动和地磁活动等因素的影响,电离图情况复杂多变。至今仍没有一个完善的方法适应复杂多变的电离图,电离图的自动判读研究仍有重要的意义。本文对电离图E区描迹自动度量进行了相关研究。电离图E区描迹由E层、Es层和E2层构成。Es层的主要特点是出现没有规律性,种类多,描迹还存在不同程度的扩散。在进行电离图E区描迹的自动度量算法设计之前,对大量电离图进行人工度量,并对人工识别E区描迹的过程进行总结。然后,利用图像处理及分析技术设计电离图E区描迹自动判读算法。人工对电离图E区描迹的判识过程主要有以下步骤:描迹查找提取、描迹特征、对描迹进行判识、获取电离图度量参数。
   
本文对电离图E区描迹的整体度量方案包括:垂测电离图E区F区分割、E区电离图类型识别及参数度量。根据电离层在E层和F层之间会存在无描迹信号的谷区,本文提出了水平投影积分法将电离图分为E区和F区电离图;利用图像处理和图像分析技术在E区电离图上进行候选描迹区域检测、非描迹区域排除;确定有效描迹区域后,根据描迹的扩散程度对描迹区域进行分类;通过对描迹区域内的信息进行分析将电离图分为五类:对于E区无描迹电离图直接给出度量结果;对于非常规复杂电离图进行标记,由人工进行度量;对于三类不同的电离图,本文根据电离图上描迹特征对其进行分割;利用Es层描迹多次反射的形成原理,确定描迹的反射次数并去除反射描迹;提取分割后描迹的特征,并对描迹类型进行识别与度量。本文工作的主要目标就根据E区电离图的人工度量经验,利用图像处理知识实现多种Es层描迹类型识别及$h^{'}Es$、$h{'}E$、$foE$、$foEs$、$fbEs$五个参数的值。如图~\ref{图2_16}为算法流程图。
\begin{figure}[h]
\centering
\includegraphics[height=16cm]{图2_16}
\caption{算法流程图}
\label{图2_16}    
\end{figure}
 
%-------------------------------------------------------------------------------------------------------------
\section{本章小结}
\label{2_9}

本章主要介绍了电离层相关内容,包括电离图的形成原因、造成电离层扰动的主要原因、电离层的结构组成及各层特征;然后列举了目前常用的电离层探测方法,并重点介绍了垂直探测技术及垂测电离图E区描迹的人工度量经验;最后提出了垂测电离图E区描迹自动度量流程。



\chapter{垂测电离图分割}
\label{cha3}
在上个章节中,主要介绍了电离层及电离层探测技术,并提出了本文算法流程图。本章主要介绍垂测电离图分割算法,包括将垂测电离层探测数据转换为电离图、电离图预处理、电离图E区F区分割算法。

%=============================================================================================================
\section{垂测数据转换为电离图}
\label{3_1}

本文实验中所采用的数据是由中国电波传播研究所自主研制的TYC-1型垂测仪采集的,采集的回波数据由电离层垂直反射信号和噪声信号构成,并以二进制的形式存储在文件中,利用数据的形成原理,我们可以将垂测数据转换为灰度电离图(320*640的二维矩阵)。
       
电离图的垂测数据文件格式:
\begin{enumerate} 
\item每次探测的数据保存在一个二进制文件中:文件名:yyyymmddHHMMp30s1.0h0.0V.O,文件名中的yyyy表示数据的采集年份、mm表示数据的采集月份、dd表示数据的采集日、HH表示数据的采集小时、MM表示数据的采集分钟,p后的数据表示频率步长$f_{step}$(KHz),s后的数据表示初始频率$f_{0}$(MHz),h后的数据表示起始高度$h_{0}$(km)。高度固定步长为2.5km。例如201110191400p35s1.0h0.0V.O表示2011年10月19日14时00分以初始频率为1MHz、步长为35KHz 采集的电离层二进制数据。

\item文件大小固定:302k=162×640+100000(字节)
         
\item162×640数据块的格式: 640个的频率点,$f_{0}$到$f_{0}+f_{step}*640$; 160×2个高度点为$h_{0}$到$h_{0}+2.5*160*2$。
\end{enumerate}   
 
将垂测数据转换为电离图的具体步骤如下:
 
\begin{enumerate}         
\item读取源数据,将数据转换为十进制并保存到一个大小为207462的数组中;
         
\item为了便于将数据转换为大小为$162\times1280$的图像,需要去掉尾部$207462-162\times1280=102$个字节的数据;
         
\item将长度为204800个元素组成的数组变为一个$160\times1280$的矩阵;
         
\item将矩阵的奇数列作为寻常波,将矩阵的偶数列作为非寻常波,得到两个大小为$160\times640$的寻常波与非寻常波矩阵;
         
\item将寻常波矩阵与非寻常波矩阵进行相加形成新矩阵,将矩阵的每一行进行复制,并插入被复制行的下一行,得到一个$320\times640$的矩阵,再将矩阵倒置就可得到垂测电离图原始图像。
\end{enumerate}  
        
为了观测电离层的变化,在我国大多数地区进行观测时都是每隔15分钟或半个小时对电离层进行一次探测,得到一组电离层探测数据。如图~\ref{图3_1}所示,原始的垂测电离图是由测高仪探测数据转换得到的,该图具有垂直测高仪探测到的全部电离图数据,由电离图可以看出各层电离层描迹保持了一定的几何形态、同时反映出了电离图各层的特征。

\begin{figure}[h]
\centering
\includegraphics[height=4cm]{图3_1}
\caption{数据转换为电离图}
\label{图3_1}    
\end{figure}
 %=============================================================================================================
\section{电离图预处理}
\label{3_2}
电离图预处理操作对电离图E区F区的正确分割有重要作用。对电离图E区F区分割之前,首先对电离图进行预处理,去除电离图上的噪声。本文根据电离图所特有的噪声特点,从图像处理角度提出了电离图预处理方法。
         
因大量电离图存在垂线噪声,这些噪声会干扰F层临界频率的判读,所以先对电离图上的垂线噪声进行去除。本文中利用垂直投影法去除垂线噪声。 在上一步经垂测数据转换为灰度电离图,即电离图上每个信号点的取值范围为(0~255),为了便于计算垂直积分值,将灰度电离图转换为二值电离图。如果电离图上某像素点灰度值大于0,就将二值电离图上对应位置赋值为1;如果电离图上某像素点灰度值等于0,就将二值电离图上对应位置赋值为0。大小为m*n的二值电离图对应的灰度值函数用$I(x,y)$表示,那么第$y$列对应的垂直投影值为$VPV(y)=\sum_{x=1}^{m}I(x,y)$,计算电离图矩阵的每列的积分值,通过大量实验验证,我们选择70作为区分电离图上描迹和垂线噪声的阈值,当电离图上某列的积分值超过阈值就认为此列为垂线噪声,并在原电离图上将此列清零。如图~\ref{图3_2}为去除垂线噪声效果图。

\begin{figure}[h]
\centering
\includegraphics[height=4.5cm]{图3_2}
\caption{去除垂线噪声}
\label{图3_2}    
\end{figure}
  
电离图上每个像素点的灰度值大小代表探测数据信号的强弱,很难通过简单地根据像素值的大小,来区分描迹信号与电离图上的噪声。我们可以利用阈值滤波选取合适的灰度阈值,去除电离图上灰度值小于阈值的像素点,这样能够减少干扰噪声。
            
阈值分割对于区分图像上的目标和背景有重要作用。对电离图进行阈值分割可以很好的去除电离图上的噪声,同时保留描迹的重要信息。在阈值分割算法中,最关键的问题时如何为图像找到一个合适的阈值,如果阈值选择不适,可能将电离图上的描迹信息误认为噪声去掉或将噪声作为描迹信息保留。在计算分割阈值时,针对不同的图像特征,研究者会选择不同的图像特征来计算阈值,自动阈值分割方法划分为基于聚类、空间、目标等特征的多种方法\cite{sezgin2004survey}。
    
本文利用阈值分割对电离图上的描迹和噪声进行划分,将小于阈值的像素点作为噪声去掉,将大于阈值的像素点作为描迹。根据电离图像素灰度值分布特点,本文选择基于聚类的自动阈值方法进行阈值计算。基于聚类的自动阈值分割主要的方法有迭代法和最大类间方差法。迭代法计算出的阈值对物体边缘并很好的效果,同时如果图像的像素值发生波动,阈值也会发生相应变化,对分割结果产生影响。本文运用最大类间方差法对电离图进行阈值分割。
         
最大类间方差法\cite{otsu1975threshold},由日本的大津展之于1979年提出。根据图像的特征,将图像上的像素分为目标和背景两类,在两类之间存在类间方差,如果类间方差越大,说明目标和背景的差别越大,分割效果越好。运用概论统计知识找到使类间方差达到最大的灰度值作为分割阈值。
         
垂直探测电离图的信号幅度值范围为 $[0, L]$,那么某个像素值出现的概率为:
\begin{linenomath}
\begin{align}
P_{i}=\frac{n_{i}}{N},\quad i\in[0,L-1]
\label{式3_1}
\end{align}
\end{linenomath}
         
设背景像素属于A类,目标像素属于B类,坐标点$(x,y)$ 的像素灰度值为 $f(x,y)$。两类的概率和灰度均值分别为:
\begin{linenomath}
\begin{align}
P_{A}=\sum_{i=0}^{k}P_{i}  
\label{式3_2}
\end{align}
\end{linenomath}
\begin{linenomath}
\begin{align}
 \omega_{A}=\sum_{i=0}^{k}iP_{i}/P_{A}   
\label{式3_3}
\end{align}
\end{linenomath}  
\begin{linenomath}
\begin{align}
  P_{B}=\sum_{i=k+1}^{L-1}P_{i} 
\label{式3_4}
\end{align}
\end{linenomath}
\begin{linenomath}
\begin{align}
 \omega_{B}=\sum_{i=k+1}^{L-1}iP_{i}/P_{B} 
\label{式3_5}
\end{align}
\end{linenomath}

图像的灰度平均值为:
\begin{linenomath}
\begin{align}
\omega=\sum_{i=0}^{L-1}i*P_{i}
\label{式3_6}
\end{align}
\end{linenomath}

我们将最佳阈值定义为能得到最大类间方差的灰度值:
\begin{linenomath}
\begin{align}
t=argmax\{P_{A}(\omega_{A}-\omega)^2+P_{B}(\omega_{B}-\omega)^{2}\}
\label{式3_7}
\end{align}
\end{linenomath}


图像以$t$为阈值的分割结果表示为:
\begin{linenomath}
\begin{align}
f_t(x, y) = \left\{ \begin{array}{ll}
0 & f(x, y) < t\\
1 & f(x, y) \ge t
\end{array} \right.
\label{式3_8}
\end{align}
\end{linenomath}
           
将电离图中灰度值小于分割阈值的像素点划分为电离图描迹,其它像素点划分为噪声,通过该方法可以很好的去除电离图上的噪声。经过大量实验验证,运用最大类间方差法能够为垂测电离图找到合适的分割阈值,同时阈值一般非常稳定,利用阈值分割对图像进行预处理就可以很好的去除电离图上的离散噪声,减小干扰。因此本文选择运用大津法对电离图进行自动去噪。如图~\ref{图3_3}为电离图经过去除噪声和最大类间方差的阈值分割得到的电离图预处理结果。

\begin{figure}[h]
\centering
\includegraphics[height=4.3cm]{图3_3}
\caption{电离图预处理结果}
\label{图3_3}    
\end{figure}
 
因此,为了使电离图E区F区分割时找到合适的分割线,本文先对原始电离图进行预处理得到去除噪声的二值电离图。先去掉原始电离图上的垂线噪声,再利用最大类间方差法对电离图进行去噪。 

 %=============================================================================================================
\section{电离图E区F区分割}
\label{3_3}

正确的对电离图E区F区进行分割对电离图E区、F区描迹检测及简化后期算法的复杂度和更准确的度量电离图相关参数都有重要作用。本文将水平投影积分法、电离图的人工度量知识、电离图度量参数经验值共同运用到电离图E区F区分割算法中。
         
电离层F区内包括F层描迹及其反射和来自Es层描迹的多次反射,E区一般为E层、Es层、E2层描迹存在的区域。通过对大量电离图进行观察发现,在电离图E区和F区之间存在一定的无描迹区域,我们将这个区域称为E-F谷区,即垂直探测仪的探测盲区。所以本文的电离图E区F区分割算法利用水平投影积分法对电离图上的无描迹区域进行检测,并结合人工度量知识与经验确定电离图E区F区之间的最优分割线。
         
为了减少算法计算量并保证算法的准确性,需要设定一个包含各种情况下E区和F区分割线的参考区间。因F层的最低虚高可以为电离图E区、F区的分割线的选取提供标准,本文对中国不同地区和不同时段电离图的F层最低虚高的经验数据进行统计,确定了范围150km~350km(电离图矩阵的180行到260行)作为参考区间。
         
首先,利用水平投影积分法对参考区间内零区(即参考区间内没有信号点的连续行构成的区域)的个数进行统计,每张电离图的参考区间内可能存在多个零区,但是,每个零区形成的原因不尽相同,可能由于F层描迹的间断、未除掉的噪声、E区描迹的间断或E区与F区之间的间隙。
         
为了更好地判断零区是不是E区与F区的间隙,我们引入下面这些概念对零区的特征进行描述:
\begin{enumerate}          
\item零区宽度值($h_{width}^{'}$):如果零区位于电离图第$x1$行到第$x2$行,那么这个零区的宽度$h_{width}^{'}=h^{'}(x1)-h^{'}(x2)$;
         
\item非零区:有信号点的连续行构成的区域;
         
\item水平投影最大值(${HPV}_{max}$):对二值电离图$I(x,y)$进行水平投影积分,可以得到第$x$行对应的水平投影值,$HPV(x)=\sum_{y=1}^{n}I(x,y)$,那么$HPV_{max}=max(HPV)$;
         
\item水平投影最大值位置(${{h^{'}}_{HPV}}_{max}$):当$x=x^{*}$时,$HPV(x^{*})={HPV}_{max}$。$x^{*}$所对应虚高$h^{'}(x^{*})$,就是水平投影最大值位置。
\end{enumerate}    
      
然后,根据参考区间内零区个数,设计了不同的分割线查找方法。具体分割算法如图~\ref{图3_4}所示,

\begin{figure}[h]
\centering
\includegraphics[height=12cm]{图3_4}
\caption{电离图E区F区分割算法流程图}
\label{图3_4}    
\end{figure}
  
对于无零区电离图,取参考区间水平积分最小值对应的行(${x_{HPV}}_{min}$)作为电离图E区、F区的分割线。         
	 
对于一个零区电离图,该零区为电离图E区和F区之间的虚高间隙,取零区的中线作为电离图E区、F区的分割线。
	 
对于多个零区的电离图,我们主要通过对零区的特征值 ${{h^{'}}_{HPV}}_{max}$和$h_{width}^{'}$进行分析,排除描迹间断,选择最优零区作为E区和F区描迹间隙。具体算法如下:
\begin{enumerate}	 
\item查找电离图上非零区的(${{h^{'}}_{HPV}}_{max}$),即查找与参考区间内零区相邻且位于零区上面的非零区的${{h^{'}}_{HPV}}_{max}$。如图~\ref{图3_5}所示,位于参考区间内的零区有零区1和零区2,与零区相邻且位于零区上面的非零区为非零区1和非零区2。求非零区1和非零区2构成区域的${{h^{'}}_{HPV}}_{max}$,即在区间$[p1,p2]$内找到${{h^{'}}_{HPV}}_{max}$,通过计算,图~\ref{图3_5}的分割线位于341km处。

\begin{figure}[h]
\centering
\includegraphics[height=8cm]{图3_5}
\caption{E区F区分割图例}
\label{图3_5}    
\end{figure}

\item根据非零区的${{h^{'}}_{HPV}}_{max}$进一步定位分割线。$HPV_{max}$可能位于F层、Es层l型或f型描迹的二次反射、Es层n型描迹,根据这些描迹虚高的经验值,设定180km作为阈值对分割线进一步定位:a) 如果${{h^{'}}_{HPV}}_{max}>180km$,$HPV_{max}$位于F层、Es层l型或f型描迹的二次反射上,因此分割线位于${{h^{'}}_{HPV}}_{max}$下面的零区中;b) 如果${{h^{'}}_{HPV}}_{max} \leq180km$,$HPV_{max}$位于Es层n型描迹上,因此分割线位于${{h^{'}}_{HPV}}_{max}$上面的零区中。
		
\item在上一步定位的零区中,进行最优零区(E区F区描迹间隙)查找。我们利用零区的$h_{width}^{'}$特征来判断零区是不是描迹的间断,根据大量电离图数据的统计结果,设定区分E区和F区之间的描迹间隙与描迹间断之间的阈值为20km。如果在上一步定位的零区中存在$h_{width}^{'}>20km$的零区,选择距离${{h^{'}}_{HPV}}_{max}$最近且$h_{width}^{'}>20km$的零区作为最优零区;否则,选取上一步定位的零区中最宽的零区作为最优零区。	

\item将最优零区的中线作为E区F区的分割线。
	
\end{enumerate}  

如图~\ref{图3_5}所示,$HPV_{max}$(投影最大值)位于F层,E区F区分割线应位于${{h^{'}}_{HPV}}_{max}$下面的零区中。根据$h_{width}^{'}$可知,零区1为F层描迹的间断,故认为零区2为E区和F区之间的描迹间隙并将其中线作为E区F区的分割线。此方法可以避免由于F层描迹间断、噪声以及E区描迹间断等造成的分割错误,能够更加有效的进行E区F区分割。


%=============================================================================================================
\section{本章小结}
\label{3_4}

本章主要介绍了电离图E区F区分割算法,其中包括垂测数据转换为电离图、电离图预处理、电离图E区F区分割。运用本文算法对不同地区和不同时间段的49678张电离图进行实验,E区F区分割错误的电离图有1286张,正确率为97.41\%。



%%% Local Variables: 
%%% mode: latex
%%% TeX-master: t
%%% End: 

\chapter{E区电离图类型识别及参数度量}
\label{cha4}

由于电离层时变、色散等特殊性,并受太阳活动和地磁活动等因素的影响,电离图情况复杂多变\cite{柳文2009基于},同时Es层的主要特点是出现没有规律性,种类多并且描迹还存在不同程度的扩散。至今仍没有一个完善的方法适应复杂多变的E区电离图,E区电离图的自动判读研究仍然有重要的意义。目前存在的Es层描迹类型识别方法中,仅仅针对常见的Es类型,如$c$型、$l$型、$f$型、$h$型进行识别。本文的E区电离图类型识别与参数度量算法中,涉及了多种Es类型识别,即包括扩散型Es层描迹,也包括非扩散型Es层描迹。
        
人工对电离图E区描迹进行判读的步骤如下:找到描迹、提取描迹特征、识别描迹类型及获取参数度量值。我们借鉴人工度量经验,运用图像处理知识设计E区电离图类型识别与参数度量算法主要包括:垂测电离图E区描迹检测、基于扩散程度的描迹区域分类、E区电离图分类、描迹分割、去除Es层多次反射、描迹特征提取、描迹类型识别与度量。

%=============================================================================================================
\section{垂测电离图E区描迹检测}
\label{4_1}

E区描迹经常会存在着不同程度的扩散现象,同时在电离图上还存在着各种原因引起的噪声(如图~\ref{图4_1}(a))。如果直接对电离图进行去噪处理, 会导致描迹信息的丢失。通过对电离图进行观察,可以发现构成描迹的区域都是电离图上非零信号点(灰度值非零的像素点)分布比较密集的区域,于是本文提出了基于密集点查找的候选描迹区域自动检测算法。

%-------------------------------------------------------------------------------------------------------------
\subsection{候选描迹区域检测}
\label{4_1_1}

首先,查找密集点。利用积分图法(Summed area table)\cite{crow1984summed}在电离图上找到有值信号点比较密集的像素点,如图~\ref{图4_1}(b)所示,电离图上可以构成描迹的像素点之间相邻或距离较近,而噪声像素点比较离散,通过对构成描迹像素点的特征进行观察与总结,本文选择使用2×2和3×1窗口来遍历整幅电离图,找到所有满足2×2和3×1窗口全部有值的小区域,并把区域内的点标记为密集点保留(图~\ref{图4_1}(c))。运用积分图法可以快速的定位电离图上的密集点。积分图法介绍如下:

\begin{figure}[!ht]
\centering
\includegraphics[width=0.8\textwidth]{图4_1}
\caption{检测候选描迹区域}
\label{图4_1}
\end{figure} 

图像矩阵为$i(x,y)$,积分图的任意一点$(x,y)$表示原图像矩阵上图像原点与该点构成的矩型小区域内所有像素点灰度值的总和,即
\begin{linenomath}
\begin{align}
I(x,y)=\sum_{\substack{x^{'}<x\\y^{'}<y}}i(x^{'},y^{'})
\label{式4_1}
\end{align}
\end{linenomath}

其中$I(x,y)$表示积分图像。$I(x,y)$可用下式表示:
\begin{linenomath}
\begin{align}
I(x,y)=i(x,y)+I(x-1,y)+I(x,y-1)-I(x-1,y-1)
\label{式4_2}
\end{align}
\end{linenomath}

          
将原图像的像素值代入上式就计算积分表,利用积分表,我们就可计算出原图像上任意一个矩阵区域的积分值。如图~\ref{图4_2},图像区域ABCD的四个端点的坐标为A=(x0, y0),B=(x1, y0),C=(x0, y1),D=(x1, y1),那么:
\begin{linenomath}
\begin{align}
\sum_{\substack{x_{0}<x\le x_{1} \\y_{0}<y\le y_{1}}}i(x,y)=I(D)+I(A)-I(B)-I(C)
\label{式4_3}
\end{align}
\end{linenomath}

\begin{figure}[!ht]
\centering
\includegraphics[width=0.3\textwidth]{图4_2.png}
\caption{积分图}
\label{图4_2}
\end{figure} 
	        
  
为了便于密集点的查找,将原始灰度电离图转换为二值电离图。原始电离图的灰度值函数为$i(x,y)$,对于图像图像上的任意点$(x,y)$,如果$i(x,y)>0$ ,就令$i(x,y)=1$。根据二值电离图$i(x,y)$建立对应的积分表$I(x,y)$,运用积分图法依次用2×2和3×1窗口遍历整个电离图,并对小区域内值全为1的点进行标记,这些点被标记为密集点。如图~\ref{图4_1}(c)为密集点集的查找结果。
              
其次,确定包含密集点的连通区域。密集点集代表E区描迹的主要信息,为了确定包含描迹的闭合区域,对密集点集进行图像形态学闭运算操作,可以得到闭合的候选描迹区域。为了确定候选描迹区域的个数,对闭操作后的电离图进行连通区域检测(图~\ref{图4_1}(d))。

	


         
数学形态学可以作为数字图像处理与分析的基础,它可以用于简化图像的复杂度、提取图像的主要信息、去除图像上的不相干结构。数字形态学主要有4种运算:膨胀、腐蚀、开、闭运算。 设集合A、B是$Z^{2}$中的两个子集,

膨胀:A被B膨胀定义为: 
\begin{linenomath}
\begin{align}
 A \oplus B=\{z | {(\widehat{B})}_{z} \cap A \neq  \varnothing \};
\label{式4_4}
\end{align}
\end{linenomath}

A被B膨胀是所有位移$z$的集合,$\widehat{B}$和A至少有一个元素是重叠的。公式也可写为:
\begin{linenomath}
\begin{align}
A \oplus B=\{z | {[(\widehat{B})}_{z} \cap A]  \subseteq  A \};
\label{式4_5}
\end{align}
\end{linenomath}   
       
腐蚀:使用B对A进行腐蚀,用$A\ominus B$表示,并定义为:
\begin{linenomath}
\begin{align}
A \ominus B=\{z | {(\widehat{B})}_{z}  \subseteq  A \};
\label{式4_6}
\end{align}
\end{linenomath}

使用B对A进行腐蚀就是所有B对A中的点$z$的集合用$z$平移。
        
上面介绍了数字形态学的膨胀和腐蚀运算。膨胀可以扩散图像轮廓、修复断裂,而腐蚀可以消除不相关的细节。对图像进行膨胀操作可以使图像上被操作区域变大,而对图像进行腐蚀操作会使图像上的被操作区域变小。除了膨胀、腐蚀操作,形态学的基本操作还有开操作和闭操作。对图像进行开操作可以使目标的边界变得更加光滑,可以消除边界上的毛刺和突出物。对图像进行闭操作,会弥补图像上物体的间断和细长的鸿沟,同时可以填补图像上小的孔洞。
                
开运算:用结构元素B对集合A进行开操作,表示为$A \circ B$,定义为:
\begin{linenomath}
\begin{align}
 A \circ B=(A \ominus B) \oplus B;
\label{式4_7}
\end{align}
\end{linenomath}

         
闭运算:用结构元素B对集合A进行闭操作,表示为$A \cdot B$,定义为:
\begin{linenomath}
\begin{align}
A \cdot B=(A \oplus B)  \ominus B;
\label{式4_8}
\end{align}
\end{linenomath}
           
如图~\ref{图4_1}(d)为电离图E区检测到的密集点(如图~\ref{图4_1}(c)所示)经过闭运算后的结果。对闭运算后的电离图进行连通区域标记,及连通区域边界的查找。下面介绍连通区域标记算法:

连通分量标记法 \cite{di1999simple}可以对图像上不连续的有值像素区域进行标记,并用不同的数字依次递增表示,使图像操作更有目标性。在电离图E区描迹区域查找中,利用连通区域标记法可以为我们确定候选描迹区域的个数。常见的连通区域标记法有四连通区域和八连通区域。四连通区域是指$(x,y)$像素位置的四个领域位置$(x-1,y)$、$(x+1,y)$、$(x,y-1)$、$(x,y+1)$,称为四连通区域(四领域)。八连通是指对应像素位置的上、下、左、右、左上、右上、左下、右下,称之为8连通区域(八邻域),八领域也可定义为如式~\ref{式4_10}。 某点的八连通区域是指该点的四连通区域加上该点右下位置、右上位置、左下位置、左上位置。如图~\ref{图4_3}为四连通区域标记结果。
\begin{linenomath}
\begin{align}
 N8(p)=N4\cup (x+1,y+1),(x+1,y-1),(x-1,y+1),(x-1,y-1)
\label{式4_10}
\end{align}
\end{linenomath}
 
\begin{figure}[!ht]
\centering
\includegraphics[width=0.6\textwidth]{图4_3.png}
\caption{四连通区域标记}
\label{图4_3}
\end{figure} 	 
 
在本文中运用八邻域标记法对闭运算后的电离图进行连通区域标记,得到如图~\ref{图4_1}(d)所示的结果。
                       
然后,排除小面积连通区域。在上一步的连通区域查找结果中,存在多个连通区域,有的连通区域面积较小,无法构成描迹。根据统计和经验,设定13作为连通区域面积的阈值对非描迹区域进行排除,得到候选描迹区域(图~\ref{图4_1}(e))。

最后,标记候选描迹区域位置。利用连通区域标记和边界查找方法,确定候选描迹区域的闭合边界并在原电离图上标记候选描迹区域的位置。如图~\ref{图4_1}(f)所示,在原电离图上确定了4个闭合的候选描迹区域。候选描迹区域的闭合边界有利于我们对区域内的描迹进一步分析。
 
\subsection{非描迹区域排除}
\label{4_1_2}
检测到的候选描迹区域中仍存在非描迹区域(如图~\ref{图4_1}(f)所示),因此需要对候选描迹区域进一步分析并对非描迹区域进行排除。对于电离图上的有值信号点,我们并不能根据信号点值的大小来判断该信号点是否属于描迹。在本文中,对大量电离图的人工度量结果进行观察,将人眼感受到的用于判断候选描迹区域是否为描迹区域的视觉特征转换为在图像上可以量化的特征。通过对大量电离图进行实验,我们总结出以下几种特征及他们用于区分描迹区域和非描迹区域时的阈值。

定义一个代表候选描迹区域内像素点灰度信息的集合$S$,$S=\{(x,y,f(x,y))\}$,其中$x$、$y$、$f(x,y)$分别为候选描迹区域内一个像素点的横坐标、纵坐标、灰度值。定义集合$S1$为:
\begin{linenomath}
\begin{align}
 S1=\{ {(x,y,f(x,y))}\subset S: f(x,y)>0 \}
\label{式4_11}
\end{align}
\end{linenomath}
 
 
为了对候选描迹区域的特征进行描述,本文定义以下概念:
\begin{enumerate}  
 
\item区域的像素密度(${\rho}_{area}$):区域内像素值非零的像素点与区域面积的比值,如式~\ref{式4_12}所示。
\begin{linenomath}
\begin{align}
 {\rho}_{area}=card(S1)/card(S)
\label{式4_12}
\end{align}
\end{linenomath}

\item区域所在的位置($x_{max}$和$y_{max}$):根据对大量电离图进行观察与统计,如果描迹区域满足式~\ref{式4_13}和~\ref{式4_14},认为该描迹区域为目前没有合理解释的非描迹区域。
\begin{linenomath}
\begin{align}
 h_{step}*(m-x_{max}) \leq 95
\label{式4_13}
\end{align}
\end{linenomath}
\begin{linenomath}
\begin{align}
 f_{0}+y_{max}*f_{step}\leq f_{0}+15f_{step}
\label{式4_14}
\end{align}
\end{linenomath}

\item区域的长度($l_{area}$):描迹区域横坐标的右端最大值减去描迹区域的左端最小值。
\begin{linenomath}
\begin{align}
  l_{area}=max⁡(y)-min⁡(y)
\label{式4_15}
\end{align}
\end{linenomath}



\item区域内非零像素个数($N_{nonzero}$):区域内非零像素点(有值信号点的个数)。
\begin{linenomath}
\begin{align}
  N_{nonzero}=card(S1)
\label{式4_16}
\end{align}
\end{linenomath}

\end{enumerate}  
  
基于上述特征设计的非描迹区域排除算法的步骤如下:

\begin{enumerate}  
  
\item如果候选描迹区域的个数为零,电离图E区无描迹。
  
\item如果$N_{nonzero}<13$,此区域为非描迹区域。
 
\item如果候选描迹区域的位置位于目前没有合理解释的无效位置,此区域为非描迹区域。
 
\item如果${\rho}_{area}<0.5$,此区域为非描迹区域。

\item如果$l_{area}<3$,此区域内为垂线噪声即非描迹区域。
\end{enumerate}  

如图~\ref{图4_4}所示,根据上述非描迹区域排除算法,可知序号为1、3、4的候选描迹区域为非描迹区域,而序号为2的候选描迹区域为描迹区域,排除非描迹区域后便得到仅包含描迹区域的干净电离图。

\begin{figure}[!ht]
\centering
\includegraphics[width=0.5\textwidth]{图4_4.png}
\caption{非描迹区域排除结果}
\label{图4_4}
\end{figure} 	
%=============================================================================================================
\section{基于扩散程度的描迹区域分类}
\label{4_2}

E区最大的特点就是描迹类型的多样性(包括E层、E2层、$l$型、$f$型、$c$型、$h$型、$r$型、$k$型、$n$型、$a$型、$s$型、$q$型描迹)。在人对于描迹类型的判别过程中,首先,根据描迹的扩散特点将描迹分为两大类:$a$型、$q$型、$s$型描迹;E层、E2层、$l$型、$f$型、$c$型、$h$型、$r$型、$k$型、$n$型描迹。然后,再根据每类描迹的特点进行特征提取与描迹类型识别。

在本文自动度量算法中,在对描迹进行特征提取与类型识别之前,先根据描迹区域的扩散程度对描迹区域进行分类。在本章中,为了评价一个区域内描迹的扩散程度,提出扩散密度的概念。根据描迹区域的扩散程度及其它特征将描迹区域分为三类:第一类:描迹区域内的描迹为直线型、上升型、下降型(E层、E2层、$l$型、$f$型、$c$型、$h$型、$r$型、$k$型、$n$型);第二类:描迹区域内的描迹为$a$型、$q$型、$s$型;第三类:描迹区域内的描迹为直线型描迹与$s$型描迹混合的情况。

通过对扩散型描迹区域和非扩散型描迹区域进行观察与总结,本文提出扩散密度($\rho_{spread}$)的概念及其求法,用一个2*2掩膜在描迹区域内遍历,如果描迹在掩膜区域矩阵内只有在2个有像素值的点,我们就把这样的点标记为扩散点。运用积分图法可以确定电离图上的扩散点个数及所有非零像素点个数。$\rho_{spread}$就等于扩散点的个数与描迹区域内所有非零像素点个数之比。如图~\ref{图4_5}所示,通过对比不同扩散密度对应的电离图可知,$\rho_{spread}$值越大,满足条件的离散点越多,描迹的扩散程度越大。

\begin{figure}[!ht]
\centering
\includegraphics[width=0.8\textwidth]{图4_5}
\caption{扩散密度对比}
\label{图4_5}
\end{figure} 

通过对描迹区域进行观察、实验和分析,我们总结出以下特征用来对描迹区域的扩散程度进行分类:像素密度($\rho_{area}$)、区域扩散密度($\rho_{spread}$)、区域最大行密度($max(\rho_{row})$)、区域平均列高度($W_{col}$)以及区域内非零像素个数($N_{nonzero}$)。

\begin{enumerate}  
\item$\rho_{area}$ :与4.1节中所介绍的含义相同,这里就不在赘述。

\item$max(\rho_{row})$:等于区域所有行密度中的最大值。行密度($\rho_{row}$):描迹区域每行中非零值的总数与描迹区域非零列总数的比值。

\item$W_{col}$:区域中非零像素点个数比上区域列数。

\item$N_{nonzero}$: 与4.1节中所介绍的含义相同,这里就不在赘述。

\end{enumerate}  

正确的对描迹区域根据扩散程度进行分类,有利于对不同扩散程度和不同类型的电离图进一步分类。具体算法步骤如下:

\begin{enumerate} 
\item如果$\rho_{area}> 0.8$或$\rho_{spread} <0.6$,描迹属于第一类描迹。

\item对区域长度大于 100 个像素同时区域宽度大于 30 个像素的描迹区域基于区域行中线和区域列中线将区域分成 4 块,并计算每块的$\rho_{area}$和$\rho_{spread}$。如果 在 4 块小区域中存在一块区域的$\rho_{area}<0.1$ ,同时存在一块$\rho_{area}>0.1$同时 $\rho_{spread}>0.8$,描迹属于第三类描迹。

\item如果区域$W_{area}<15$、$max(\rho_{row})>0.8$或$W_{col}<8$,描迹为第一类描迹。如果不满足条件,描迹为第二类描迹。
\end{enumerate}
%=============================================================================================================
\section{E区电离图分类}
\label{4_3}

我们对每个描迹区域进行分类后,再根据 E 区电离图上所有描迹区域的信息对 E 区电离图进行分类。为了方便描迹自动度量算法的设计,将E区电离图分为以下五类:第一类,无描迹E区电离图;第二类,常规E区电离图(E层、 E2层、$l$型、$f$型、$c$型、 $h$型、$r$型、$k$型、$n$型);第三类,扩散型E区电离图(Es层$a$型、$q$型、$s$型);第四类:直线型与$s$型描迹混合的情况;第五类:复杂描迹E区电离图。E区电离图分类算法如下:

\begin{enumerate} 
\item E区电离图上描迹区域个数为0,电离图属于第一类(无描迹 E 区电离图)。

\item 如果E区电离图上的所有描迹区域都属于第一种描迹区域,那么电离图就属于第二类E区电离图;如果E区电离图上的所有描迹区域都属于第二种描迹区域,那么电离图属于第三类电离图;如果E区电离图上的描迹区域只有一个且属于第三种描迹区域,那么E区电离图属于第四类电离图;如果E区电离图属于同时包含两种以上的描迹区域,那么电离图属于第五类电离图(特殊电离图)。
\end{enumerate} 

对于第一类无描迹E区电离图,我们可以直接得到E区参数的度量结果;对于第五类电离图,由于描迹情况比较复杂,在算法中对复杂电离图进行标记,进行人工度量;对于剩余三类电离图,根据电离图上的描迹分布特点,设计了不同的描迹分割及特征提取算法。

有效的进行电离图分类,并确定无描迹E区电离图和复杂非常规E区电离图,可以有效的提高算法正确率并简化后期算法。对于不同特征的电离图,本文设计了不同的描迹分割、描迹特征提取及判读方法。

%=============================================================================================================
\section{描迹分割}
\label{4_4}

E区经常出现E层描迹和Es层描迹相连或两个Es层描迹之间距离很近的情况,这样会导致在候选描迹区域查找结果中存在不同层描迹信息被划分到同一个描迹区域内。所以,先要对区域内的描迹进行分割,并把每个描迹单独的提取出来。本文根据描迹的特点针对不同类型的电离图,设计不同的描迹分割方法。 对于第三类扩散性E区电离图无需进行描迹分割,对第二类与第四类电离图进行描迹分割,具体算法介绍如下。

\subsection{第二类E区电离图描迹分割}
\label{4_4_1}

第二类电离图为常规型电离图,包括E 层、 E2 层、$l$ 型、$f$型、$c$型、 $h$型、$r$型、$k$型、$n$型描迹,在电离图上的一个描迹区域内,经常出现两个描迹或多个描迹。为了对描迹进行精确的类型识别和度量,需要先对描迹进行分割。通过对大量电离图数据的观察,总结出对于常规型描迹分割主要有以下三个步骤:基于虚高的描迹一次分割、基于最高点的描迹二次分割、基于OX波分离的描迹三次分割。 分割算法如图~\ref{图4_6}所示。

\begin{figure}[!ht]
\centering
\includegraphics[width=0.7\textwidth]{图4_6.png}
\caption{描迹分割算法流程图}
\label{图4_6}
\end{figure}


1. 基于虚高的描迹一次分割

基于虚高的描迹一次分割主要对E区描迹区域内出现两个在虚高方向上不存在任何交集的描迹进行分割(如图~\ref{图4_7}所示)。对于经过预处理后的描迹区域,利用在E区F区分割算法中提到的水平积分投影法确定描迹区域内的零区。检测到的零区有两种可能:一种是描迹的间断,另一种是两个在虚高上无交集的描迹之间的间隙。通过对零区上下两个描迹区域的三种特征($N_{nonzero}$、最大行密度$max(\rho_{row}$)(行密度:描迹区域每行中非零值的总数与描迹区域非零列总数的比值)、${{h^{'}}_{HPV}}_{max}$)来判断描迹区域是否需要基于虚高进行分割:如果零区上(top)下(bot)描迹区域同时满足下面的条件:
\begin{linenomath}
\begin{align}
N_{nonzero}(top)>10    \quad  and \quad N_{nonzero}(bot)>10
\label{式4_17}
\end{align}
\end{linenomath}
\begin{linenomath}
\begin{align}
max(\rho_{row})(top)>0.5  \quad and \quad max(\rho_{row})(bot)>0.5 
\label{式4_18}
\end{align}
\end{linenomath}
\begin{linenomath}
\begin{align}
{{h^{'}}_{HPV}}_{max}(top)>5   \quad and \quad {{h^{'}}_{HPV}}_{max}(bot)>5   
\label{式4_19}
\end{align}
\end{linenomath}
认为描迹需要基于虚高进行分割;如果不满足则认为零区属于描迹间断故不需要基于虚高进行分割。对于需要划分的描迹区域,将零区的中线作为两个描迹的分割线。如图~\ref{图4_7}所示为基于虚高的描迹一次分割过程及结果。
\begin{figure}[!ht]
\centering
\includegraphics[width=0.6\textwidth]{图4_7.png}
\caption{基于虚高的描迹一次分割结果}
\label{图4_7}
\end{figure}

2.基于最高点的描迹二次分割

一次分割后的描迹可能会存在直线型和非直线型两种类型,非直线型描迹多数为E层与Es层$c$型、E层与Es层$h$型描迹同时存在等情况,对于这些描迹需要基于最高点再进行二次分割。

首先,需要进行直线型描迹判识:若检测到直线型描迹,则描迹分割结束;否则对其进行基于最高点的描迹二次分割。直线型描迹主要从以下两个步骤进行判识:


第一,基于图像处理对描迹区域进行形态学预处理、骨架化、曲线拟合和多项式求导实现直线型描迹初步判识。

对区域的描迹进行形态学闭运算可以使描迹轮廓更为光滑、修复描迹的断裂,并保持描迹原型。所以对描迹进行适当的形态学处理,有助于优化骨架化结果。

图像的骨架化结构能够减少图像的冗余信息,因此广泛用于图像处理的各个领域。对形态学处理后的描迹进行骨架化能够得到一个像素宽的骨架化结果。

Blum最早提出图像用骨架描述其结构的方法,它分别提出了烧草模型和最大内切圆盘\cite{blura1973biological}概念来定义图像骨架(如图~\ref{图4_8})。在他提出的骨架化方法中运用了中轴的概念:在$t=0$时,点燃图像边缘上的所有点,图像边界上的点以相同的速度向图像内部燃烧,当边界上不同的两个点发出的波相遇时,火焰就熄灭,火焰相遇的点构成的集合叫作中轴。同时我们还可以利用中轴函数(从点燃烧到x点所经历的时间t之间的函数关系)对图像进行重建。

\begin{figure}[!ht]
\centering
\includegraphics[width=0.7\textwidth]{图4_8.png}
\caption{两种骨架化模型}
\label{图4_8}
\end{figure}


对电离图进行骨架化操作时,希望能够保留较完整的描迹信息,能够通过骨架化结果判读出描迹的形状,同时可以通过对骨架化结果进行操作来还原描迹。常见的骨架化算法有基于拓扑细化的、基于形态学操作的、基于距离变化的\cite{廖志武20092}\cite{陈晓飞2004基于骨架的目标表示和识别技术研究}。 在本文中运用基于形态学操作的骨架化提取算法。

在基于形态学操作的骨架化提取算法中运用了击中或击不中变换算法。形态学中的击中或击不中变换算法是形状检测的基本工具。在本文第三章已经介绍过数字形态学的基本概念,下面介绍击中或击不中算法。A为待处理图像,B为满足下式的结构元素对,且B由B1和B2构成。
\begin{linenomath}
\begin{align}
\left \{
\begin{aligned}
B=B1 \cup B2\\
B1 \cap B2=\varnothing
\end{aligned} 
\right.
\label{式4_20}
\end{align}
\end{linenomath}

$B\{B1,B2\}$对A的击中或击不中变换$(HMT)$定义如下式:
\begin{linenomath}
\begin{align}
 {HMT}_{B}(A)=(A \ominus B_{1})\cap(A^{c} \ominus B_{2})
\label{式4_21}
\end{align}
\end{linenomath}

通过上式可知击中或击不中变换的结果同时包含了所有原点,$B_{1}$在A内找到匹配,同时$B_{2}$在$A^{c}$(A的补集)找到匹配。找到图像中与结构元素相匹配的所有像素点。运用击中或击不中变换对描迹进行骨架化方法的定义如下:
\begin{linenomath}
\begin{align}
A \otimes B =A-{HMT}_{B}(A)
\label{式4_22}
\end{align}
\end{linenomath}
 
结构元素对图像的细化有重要影响,在应用中用若干个结构元素对序列${B}={B^{1}, B^{2},...,B^{n}}$进行击中或击不中变换,如下式所示:
\begin{linenomath}
\begin{align}
A \otimes {B}=((...((A \otimes B^{1})\otimes B^{2})...) \otimes {B^{n}})
\label{式4_23}
\end{align}
\end{linenomath} 
  
公式~\ref{式4_23}表明,我们依次用结构元素对前一次的操作结果,继续进行细化,当骨架化结果不在改变时,可以停止细化过程。如图~\ref{图4_9}为常见的结构元素。
 
\begin{figure}[!ht]
\centering
\includegraphics[width=0.7\textwidth]{图4_9.png}
\caption{多方向细化结构元素对}
\label{图4_9}
\end{figure}

本文利用基于形态学操作的骨架化提取算法对描迹进行骨架化,得到单像素的骨架化结果。为确定的描迹的形状,我们对描迹的骨架化结果进行多项式曲线拟合。多项式拟合的定义如下:假设给定数据点$(x_{i},y_{i})$,$(i=0,1,...,m)$,$\phi$为次数小于$n$ $(n<m)$的多项式构成的函数类,找到$p_{n}(x)={\sum}_{k=0}^{n}a_{k}x^{k}\subset \phi$,使得
\begin{linenomath}
\begin{align}
I={\sum}_{k=0}^{m}[p_{n}(x_{i})-y_{i}]^{2}={\sum}_{i=0}^{m}{({\sum}_{k=0}^{n}a_{k}{x_{i}}^{k}-y_{i})}^{2}=min
\label{式4_24}
\end{align}
\end{linenomath} 

 当函数为多项式时,同时满足式~\ref{式4_24}称为最小二乘拟合多项式。通过大量实验验证,在本文中我们对描迹的骨架化结果进行4次多项式拟合。如果多项式次数太低会出现拟合曲线与用来拟合的描迹点之间存在较大的差值,反之如果选择的多项式次数太高拟合出来的曲线存在多个拐点,影响描迹形状的判断。
 
为了判断曲线拟合后描迹的形状,对拟合多项式进行求导。如图~\ref{图4_10}为不同形状描迹的曲线拟合与求导结果,观察可以发现直线型描迹的求导结果在零值附近,基于该特征对直线型描迹进行初步判识;对于存在着断裂等现象的直线型描迹,因骨架化拟合效果不好,未检测出其为直线,则需要进一步判识。
 
\begin{figure}[!ht]
\centering
\includegraphics[width=0.5\textwidth]{图4_10.png}
\caption{不同描迹形状的曲线拟合与求导结果}
\label{图4_10}
\end{figure}

 
第二,利用像素分布特征对可能存在断裂的直线型描迹进行进一步判识。我们采用的像素分布特征主要包括描迹的最大宽度($W_{max}$)和$\rho_{row}$:直线型相对于其他型描迹宽度较小且最大行密度$max(\rho_{row})$接近于1。

通过以上两步,可以检测出描迹是否为直线型描迹,如果是直线型描迹,描迹细分结束;如果是非直线型描迹,基于最高点进行描迹的二次分割。

然后,对于非直线型描迹,进行基于最高点的描迹二次分割:查找描迹区域内的最高点,如果存在多个相同高度的最高点,就利用$foE$的经验值选择最高点;再对最高点左右两侧的总像素个数进行统计,如果最高点左右非零信号点个数都大于20,认为最高点左右两侧都有描迹,进而将描迹划分为左侧描迹和右侧描迹。图~\ref{图4_11}为基于最高点的描迹分割结果。

\begin{figure}[!ht]
\centering
\includegraphics[width=0.6\textwidth]{图4_11.png}
\caption{基于最高点的描迹细分结果}
\label{图4_11}
\end{figure}

3.基于OX波分离的描迹三次分割

对二次分割结果的单侧描迹的$N_{nonzero}$进行统计,如果$N_{nonzero}<50$,认为描迹短小且不存在OX分离的情况,此类描迹可用端点位置分析法进行描迹类型识别与度量;对于$N_{nonzero}\geq 50$的描迹进行基于OX波分离的描迹三次分割:

首先,骨架化与连通区域查找。对单侧描迹的骨架化结果进行连通区域查找(图~\ref{图4_12}(a1)与(b1)),如果只有一个连通区域,无需进行OX波分离;如果存在大于一个连通区域的描迹,标记每个连通区域的左端最高点和右端最高点,我们把标记值为i的连通区域的左端最低点、右端最高点、右端最低点在电离图上的坐标用$(x_{l\_Low} (i) \quad y_{l\_Low} (i))$、$(x_{r\_Hight} (i) \quad y_{r\_Hight} (i) )$和$(x_{r\_Low} (i) \quad y_{r\_Low} (i) )$表示。

\begin{figure}[h]
%\begin{figure}[!ht]
\centering
\includegraphics[width=0.5\textwidth]{图4_12.png}
\caption{基于OX波分离的描迹三次分割结果}
\label{图4_12}
\end{figure}

其次,连通区域选择性合并。OX波分离时O波与X波描迹具有以下特征:O波与X波的虚高相差很小;O波与X波描迹形状相同;O波与X波描迹的最大频率相差半个磁旋频率(0.7MHz左右)(如图~\ref{图4_12}(b1))。对于连通区域个数大于1个的描迹,本文利用OX分离时描迹的这些特征,设计连通区域选择性合并算法如下:
 
第一,对于标记值为p的连通区域,从标记值大于p的连通区域中查找距其最近的连通区域(其标记值为q)。

第二,判断这两个连通区域是否满足以下条件:左侧描迹条件$0.6MHz<f_{step} [y_{r\_Low}(p)-y_{r\_Low}(q)]<0.8MHz$且$|x_{r\_Low} (p)-x_{r\_Low} (q)|\leq3$;右侧描迹条件为$0.6MHz<f_{step} [y_{r\_Low}(p)-y_{r\_Low} (q)]<0.8MHz$且$|x_{r\_Low} (p)-x_{r\_Low} (q)|\leq3$。如果不能满足以上条件,说明两个描迹连通区域非OX波分离的情况,对连通区域进行合并,即将标记值为p的连通区域重新标记为q(图~\ref{图4_12}(a2));否则,不进行合并(图~\ref{图4_12}(b2))。
 
第三,按照上述步骤依次遍历每个连通区域。

最后,根据骨架化后的连通区域选择性合并结果进行基于OX波分离的描迹三次分割。骨架化结果经连通区域选择性合并后,如果连通区域的个数为1,描迹为非OX波分离情况无需分割(图~\ref{图4_12}(a3));如果连通区域的个数为2,对描迹进行基于OX波分离的描迹三次分割(图~\ref{图4_12}(b3));如果连通区域的个数大于2,将连通区域中面积较小的连通区域作为描迹的毛刺去掉,只保留连通区域面积最大和次大的连通区域并进行分割。



\subsection{第四类E区电离图描迹分割}

第四类E区电离图是指E区只包含直线型与$s$型描迹混合情况时的电离图,对于这类电离图我们应把直线型描迹和s型描迹分开。

利用直线型描迹与$s$型描迹特征的区别,提出了直线型与$s$型描迹分割算法主要包括以下步骤:

由图~\ref{图4_13}(a)可知直线型描迹与$s$型描迹的扩散程度不同,因此将描迹区域单独放在一个矩阵中,按照矩阵的行中线和列中线将描迹分为四块,然后将四块描迹区域中$\rho_{area} < 0.1$的描迹区域内像素值非零的点赋值为 0;

\begin{figure}[!ht]
\centering
\includegraphics[width=0.5\textwidth]{图4_13}
\caption{第四类E区电离图描迹分割结果}
\label{图4_13}
\end{figure}

确定描迹区域每个非零行中从左到右第一个非零值所在的列值$col(i)$,从第一行开始比较第$i$行与第$i + 1$行是否满足$|col(i + 1)-col(i)| < 1/2max(col)$,如果满足继续比较,直到不满足条件停止;不满足条件时,对应的行值$i$就是$s$型描迹与直线型描迹的分割线;如图~\ref{图4_13}为第四类 E 区电离图描迹分割图例。



%=============================================================================================================
\section{去除Es层多次反射}
\label{4_5}
 
通过4.4节对描迹区域进行分割后,电离图E区上每个描迹区域内只存在一个描迹,并可以确定描迹区域的位置。对分割后的描迹去处多次Es层反射。

去除 Es 层多次反射,对于 F 区描迹度量具有重要意义。本文去除 Es 多次反射主要根据 Es描迹多次反射形成原理。反射回波描迹的虚高与一次回波描迹的虚高成倍数关系。即Es二次反射描迹的最低虚高应为Es层描迹最低虚高的二倍,Es层三次反射描迹的最低虚高应为Es层描迹虚高的三倍,更高次回波描迹可以依次类推。Es 层多次反射的具体算法如下:

首先,电离图经过 E 区 F 区分割得到 E 区电离图和 F 区电离图。对于F区电离图,用E区描迹查找方法对F区电离图进行候选描迹区域查找、非描迹区域排除,得到了只包含有效描迹的 F 区电离图。

其次,通过描迹分割,可以确定电离图 E 区共有多少个描迹以及每个描迹的 最低高度、最高高度、最小频率、最大频率。如图~\ref{图4_14}所示,可以把描迹确定在电离图上的一个小的矩阵区域内。

\begin{figure}[!ht]
\centering
\includegraphics[width=0.9\textwidth]{图4_14}
\caption{查找Es层描迹的多次反射}
\label{图4_14}
\end{figure}

然后,利用描迹的最低高度与其反射描迹的最低高度的倍数关系,我们就可以在F区电离图上确定多个候选反射区域的位置。

最后,对比候选反射区域内描迹的像素个数与原描迹像素个数的关系,排除无效的候选反射区域,只保留有效的反射区域。 图~\ref{图4_14}为 Es 层多次反射查找结果。


%=============================================================================================================
\section{描迹特征提取}
\label{4_5}


对大量电离图进行人工度量发现,人工对描迹类型的识别主要根据描迹形状(上升型、下降型、直线型)、扩散程度、描迹宽度、描迹的虚高、电离图采集 时间、描迹的最大频率、描迹最小频率等特征。在本文中利用图像处理和图像分析知识对描迹特征进行自动提取,同时基于人工度量知识,对描迹进行类型识别与参数判读。

\subsection{第二类E区电离图描迹特征提取}

通过对描迹区域内的描迹进行分割,就可以把分割后的每个描迹单独提取出来。经过对人工度量过程总结发现,描迹的形状、虚高、最大频率、最小频率以及电离图的采集时间都是判断E区描迹类型的重要依据。为了对描迹类型进行识别,我们对描迹的形状、虚高、频率特征进行自动提取。由于E区描迹的多样性,本文设计了不同的描迹特征提取方法:对于经过基于OX波分离描迹三次分割后输出为骨架化形式的描迹,采用描迹曲线拟合法来提取描迹特征;对于总像素值 比较少的短小型描迹和直线型描迹,利用端点位置分析法来提取描迹特征。

a) 描迹曲线拟合法

提取描迹形状特征:描迹曲线拟合时如果选择的多项式次数太低会导致拟合结果与实际曲线之间有很大的误差;如果选择的多项式次数太高,会导致拟合的曲线波动较大,影响求导结果。通过大量实验验证,多项式次数为4时效果最佳,所以本文选择对骨架化结果进行4次多项式拟合。为了确定描迹形状,对拟合多项式进行求导,进而确定描迹形状为上升型或下降型。

提取描迹的虚高和频率特征:如果直接对骨架化描迹进行度量,度量结果存在很大的误差。因此先对骨架化结果利用形态学操作进行描迹还原,然后对还原描迹进行度量。定义一个代表还原描迹区域在电离图上坐标的集合$S$,$S = {(x, y)}$,其中$x$、$y$分别为描迹区域内一个像素点的横坐标、纵坐标。利用下述公式分别计算描迹的最小频率$f_{min}$、最大频率$f_{max}$、描迹虚高${h^{'}}_{min}$。

\begin{linenomath}
\begin{align}
f_{min} = f_{0} + min(y)*f_{step}
\label{式4_25}
\end{align}
\end{linenomath}
\begin{linenomath}
\begin{align}
f_{max} = f_{0} + max(y)*f_{step}
\label{式4_26}
\end{align}
\end{linenomath}
\begin{linenomath}
\begin{align}
{h^{'}}_{min} = h_{step}*(m-max(x))
\label{式4_27}
\end{align}
\end{linenomath}


b) 端点位置分析法

在检测到的直线型描迹中混有一些描迹形状不明显的上升型、下降型、中间凸起型描迹,需要进行二次判识与度量;另外,在基于最高点的描迹二次分割中产生了$N_{nonzero}<50$的短小型描迹。对于这两类描迹利用描迹曲线拟合法,很难得到理想的效果。于是本文依据描迹的像素分布特点提出了端点位置分析法来提取描迹特征。

提取描迹形状特征:端点位置分析法主要利用描迹的左端最高点、右端最高点及描迹最高点之间的关系,来确定描迹形状。描迹的左端最高点、右端最高点及描迹最高点在电离图上的坐标分别为$(x_{l}, y_{l})$、$(x_{r}, y_{r})$、$(x_{m}, y_{m})$。

如果$x_{r}-x_{l} \geq 4$,描迹形状为下降型。

如果$x_{l}-x_{r}\geq 4$,描迹形状为上升型。

如果$|x_{r}-x_{l}|<4$,同时$1/2(x_{r}+x_{l} )-x_{m}\geq4$时,描迹形状为中间凸起型。

如果$|x_{r}-x_{l} |<4$,同时$1/2(x_{r}+x_{l} )-x_{m}<4$时,描迹形状为直线型。

提取描迹的虚高和频率特征:对于上升型、下降型、直线型描迹,利用公式~\ref{式4_28}、~\ref{式4_30}计算描迹的 $f_{min}$、$f_{max}$。把描迹放入集合S中,利用公式~\ref{式4_27}可得${h^{'}}_{min}$。对于中间凸起型描迹,按最高点将描迹分割为两个描迹并分别获取两个描迹的描迹形状及 $f_{min}$、$f_{max}$、${h^{'}}_{min}$。
\begin{linenomath}
\begin{align}
f_{min} = f_{0} + y_{l}*f_{step}
\label{式4_28}
\end{align}
\end{linenomath}
\begin{linenomath}
\begin{align}
f_{max} = f_{0} + y_{r}*f_{step}
\label{式4_30}
\end{align}
\end{linenomath}  
 
\subsection{第三类E区电离图描迹特征提取}

第三类E区电离图是指 E 区只包含扩散性描迹 (Es 层 $a$型、$q$型、$s$型)。 第三类E区电离图判读算法流程如下:
首先,检测$s$型描迹,确定描迹区域闭合边界,找到边界中每一列的中点,对所有中点利用4次多项式进行拟合,确定拟合曲线的两个端点;利用两个端点位置做描迹区域的中线,并计算中线的斜率;如果斜率大于 0.15,就认为该描迹为$s$型描迹;
其次,根据描迹区域的长度$l_{area}$,对$a$型和$q$型描迹进行区分,如果$l_{area} < 100$,区域内描迹为$a$型;否则,描迹为$q$型;
然后,获取$a$型和$q$型描迹区域的左端点、右端点、对应的频率值作为描迹的 $f_{min}$、$f_{max}$,把描迹区域的最低虚高作为该描迹的${h^{'}}_{min}$。

\subsection{第四类E区电离图描迹特征提取}

对分割后的直线型描迹与$s$型描迹直接进行描迹特征提取。根据描迹的左端点、右端点确定对应的频率值作为描迹的 $f_{min}$、$f_{max}$,同时描迹的最低非零行对应的虚高值,作为${h^{'}}_{min}$。

%=============================================================================================================
\section{描迹类型识别与度量}
\label{4_6}

通过对电离图E区上的每个描迹区域进行描迹分割、描迹特征提取,可以确定电离图E区上共有多少个常规型描迹及每个描迹特征(电离图采集时间、描迹形状、$f_{min}$、$f_{max}$、${h^{'}}_{min}$)。

首先,我们根据每个常规型描迹的描迹特征,结合电离图度量知识对每个描迹的类型进行识别,即判断描迹属于E层、E2层或Es层的某种类型; 对于扩散型描迹,在描迹分割和特征提取算法中已经得到描迹的类型及特征。然后,对电离图的F层描迹的最小频率进行自动度量。最后,根据电离图度量规则得到E区参数$foE$、$h^{'}E$、$h^{'}Es$、$foEs$、$fbEs$。如图~\ref{图4_15}所示为最终的参数度量结果。

\begin{figure}[!ht]
\centering
\includegraphics[width=0.7\textwidth]{图4_15}
\caption{度量结果}
\label{图4_15}
\end{figure}
      

%-------------------------------------------------------------------------------------------------------------
\section{本章小结}
\label{4_7}

本章考虑到电离图E区描迹的多样性,设计了电离图E区描迹自动判读算法。算法主要包括:基于密集点检查提出了描迹区域查找算法;对于检测到的描迹区域根据扩散程度对区域进行分类;根据描迹区域的特征和分类结果对电离图进行分类,并排除非常规电离图;对于每类电离图根据电离图特征、运用图像处理及曲线拟合等技术分别设计了描迹分割、去除Es层多次反射、描迹特征提取算法;根据度量规则获取电离图E区描迹的相关参数。



















%%% Local Variables: 
%%% mode: latex
%%% TeX-master: t
%%% End: 

\chapter{实验分析}
\label{cha5}
为了验证本文算法的可行性,本章主要介绍运用本文算法对我国不同地区采集的电离图进行实验,并对实验结果分析。
%=============================================================================================================
\section{实验结果}
\label{5_1}
本文采用测试数据集是由北京(2013年6月)、广州(2013年9月)、昆明(2011年12月)、满洲里(2014年2月)、新乡(2011年4月)五个地区不同月份采集的3576张电离图构成,其中每个月份的数据是由该月每一天连续整点采集的电离图构成。本文参考了URSI准则中使用的评价方法,对数据集内的电离图使用本文自动度量算法得到的结果进行分析。
        
参数$foEs$、$fbEs$、$foE$选取的误差允许范围为1MHz,即如果自动度量结果与人工度量结果之差的绝对值小于1MHz,就认为这个参数的自动度量结果是可以被接受的;参数${h^{'}E}$和${h^{'}Es}$选取的误差允许范围为10千米,即自动度量结果与人工度量结果之差的绝对值小于10千米,就认为自动度量结果是可以被接受的。表~\ref{表5_1}为自动度量参数$foEs$、$fbEs$、${h^{'}Es}$的可接受率,表~\ref{表5_2}为自动度量参数$foE$、${h^{'}E}$的可接受率。
\begin{table}
\caption{自动度量参数$foEs$、$fbEs$、${h^{'}Es}$的可接受率}
\centering
\begin{tabular}{l|c|c|c|c|c|c}
\toprule
\multirow{3}{*}{参数} & \multicolumn{3}{c|}{$N_{Es}$:1880张} & \multicolumn{2}{c|}{$N_{NoEs}$:1696张} & \multirow{2}{*}{$R_{all\_accepted\_*}$} \\
\cline{2-6}
& \multicolumn{2}{c|}{$N_{det\_Es}$} & \multirow{2}{*}{$N_{f\_det\_Es}$} & \multirow{2}{*}{$N_{undet\_Es}$} & \multirow{2}{*}{$N_{w\_undet\_Es}$} & \\
\cline{2-3}
& $N_{accepted\_*}$ & $N_{unaccepted\_*}$ & & & & \\
\cline{1-7}
$foEs$ & 1480 & 158 & 242 & 1418 & 278 & 81.03\% \\
$fbEs$ & 1360 & 278 & 242 & 1418 & 278 & 77.68\%\\
${h^{'}Es}$ & 1239 & 399 & 242 & 1418 & 278 & 74.29\%\\
\bottomrule
\end{tabular}
\label{表5_1}
\end{table}

由表~\ref{表5_1}可知,在3576张电离图的人工度量结果中,无Es层描迹的电离图总数($N_{NoEs}$)为1696,在1696张电离图的自动度量结果中,自动度量未检测到Es层描迹的电离图总数($N_{undet\_{Es}}$)为1418张。由公式~\ref{式5_1}可知,在人工度量电离图无Es层描迹的情况下,自动度量算法也未检测到Es层描迹的正确率($R_{undetected\_{Es}}$)为83.6\%。

\begin{linenomath}
\begin{align}
R_{undetected\_{Es}}=N_{undet\_{Es}}/N_{NoEs}
\label{式5_1}
\end{align}
\end{linenomath}

由表~\ref{表5_1}可知,在3576张电离图的人工度量结果中,存在Es层描迹的电离图总数($N_{Es}$)为1880张,在1880张电离图的自动度量结果中,未能成功检测到Es层描迹的电离图总数($N_{f\_det\_Es}$)为242张,成功检测到Es层描迹的电离图总数($N_{det\_Es}$)为1638张。由公式~\ref{式5_2}可知,在人工度量结果中电离图存在Es层描迹的情况下,自动度量算法检测Es层描迹的正确率($R_{detected\_Es}$)为87.13\%。   
\begin{linenomath}
\begin{align}
R_{detected\_{Es}}=N_{det\_{Es}}/N_{Es}
\label{式5_2}
\end{align}
\end{linenomath}

由表表~\ref{表5_1}可知,在自动检测到的存在Es层描迹的1638张电离图中,参数${h^{'}Es}$、$foEs$、$fbEs$的自动度量结果可以被接受的电离图总数($N_{accepted\_foEs}$)、$N_{accepted\_fbEs}$、$N_{accepted\_h'Es}$)分别为:1480张、1360张、1239张。根据公式~\ref{式5_3}、~\ref{式5_4}、~\ref{式5_5}可知,在自动度量算法检测到存在Es层描迹的电离图中,Es层参数自动度量结果的可接受率($R_{accepted\_foEs}$、$R_{accepted\_fbEs}$、$R_{accepted\_h'Es}$)分别为:90.35\%、83.03\%、75.64\%。
\begin{linenomath}
\begin{align}
R_{accepted\_foEs}=N_{accepted\_foEs}/N_{det\_Es}
\label{式5_3}
\end{align}
\end{linenomath}
\begin{linenomath}
\begin{align}
R_{accepted\_{fbEs}}=N_{accepted\_fbEs}/N_{det\_{Es}}
\label{式5_4}
\end{align}
\end{linenomath}
\begin{linenomath}
\begin{align}
R_{accepted\_{h'Es}}=N_{accepted\_h'Es}/N_{det\_{Es}}
\label{式5_5}
\end{align}
\end{linenomath}

为了对算法的整体正确率进行评价,考虑数据集内所有电离图的识别情况,可根据公式~\ref{式5_6}、~\ref{式5_7}、~\ref{式5_8}计算Es层参数$foEs$、$fbEs$、${h^{'}Es}$的可接受率($R_{all\_accepted\_foEs}$、$R_{all\_accepted\_fbEs}$、$R_{all\_accepted\_h'Es}$)分别为81.03\%、77.68\%、74.29\%。

\begin{linenomath}
\begin{align}
R_{all\_accepted\_foEs}={(N_{accepted\_foEs}+N_{undet\_Es})}/{(N_{Es}+N_{NoEs})}
\label{式5_6}
\end{align}
\end{linenomath}
\begin{linenomath}
\begin{align}
R_{all\_accepted\_{fbEs}}={(N_{accepted\_fbEs}+N_{undet\_Es})}/{(N_{Es}+N_{NoEs})}
\label{式5_7}
\end{align}
\end{linenomath}
\begin{linenomath}
\begin{align}
R_{all\_accepted\_{h'Es}}={(N_{accepted\_h'Es}+N_{undet\_Es})}/{(N_{Es}+N_{NoEs})}
\label{式5_8}
\end{align}
\end{linenomath}

\begin{table}
\caption{自动度量参数$foE$、${h^{'}E}$的可接受率}
\centering
\begin{tabular}{l|c|c|c|c|c|c}
\toprule
\multirow{3}{*}{参数} & \multicolumn{3}{c|}{$N_{E}$:1337张} & \multicolumn{2}{c|}{$N_{NoE}$:2239张} & \multirow{2}{*}{$R_{all\_accepted\_*}$} \\
\cline{2-6}
& \multicolumn{2}{c|}{$N_{det\_E}$} & \multirow{2}{*}{$N_{f\_det\_E}$} & \multirow{2}{*}{$N_{undet\_E}$} & \multirow{2}{*}{$N_{w\_undet\_E}$} & \\
\cline{2-3}
& $N_{accepted\_*}$ & $N_{unaccepted\_*}$ & & & & \\
\cline{1-7}
$foE$ & 1039 & 34 & 264 & 2048 & 191 & 86.32\% \\
${h^{'}E}$ & 757 & 316 & 264 & 2048 & 191 & 78.43\%\\
\bottomrule
\end{tabular}
\label{表5_2}
\end{table}

由表~\ref{表5_2}可知,在3576张电离图的人工度量结果中,无E层描迹的电离图总数($N_{NoE}$)为2239。这2239张电离图的自动度量结果中,自动度量也未检测到E层描迹的电离图总数($N_{undet\_E}$)为2048张。根据公式~\ref{式5_9},在人工度量电离图无E层描迹的情况下,自动度量算法也未检测到E层描迹的正确率($R_{undetected\_E}$)为91.47\%。
\begin{linenomath}
\begin{align}
R_{undetected\_E}=N_{undet\_E}/N_{NoE}  
\label{式5_9}
\end{align}
\end{linenomath}
   
由表~\ref{表5_2}可知,在3576张电离图的人工度量结果中,存在E层描迹的电离图总数($N_{E}$)为1337张,这1337张电离图的自动度量结果中,未能成功检测到E层描迹的电离图总数($N_{f\_det\_E}$)为264张,能够成功检测到E层描迹的电离图总数($N_{det\_E}$)为1073张。根据公式~\ref{式5_10}可知,在人工度量结果中电离图存在E层描迹的情况下,自动度量算法能检测E层描迹的正确率($R_{detected\_E}$)为80.25\%。  
\begin{linenomath}
\begin{align}
R_{detected\_E}=N_{det\_E}/N_{E}  
\label{式5_10}
\end{align}
\end{linenomath}

在自动算法检测到的存在E层描迹的1073张电离图中,自动度量参数$foE$、${h^{'}E}$的自动度量结果中可以被接受的电离图总数($N_{accepted\_foE}$、$N_{accepted\_h'E}$)分别为:1039张、757张;根据公式~\ref{式5_11}、~\ref{式5_12}可知,在自动度量算法检测到存在E层描迹的电离图中,自动度量结果的可接受率($R_{accepted\_foE}$、$R_{accepted\_h'E}$)分别为96.83\%, 70.55\%。
\begin{linenomath}
\begin{align}
R_{accepted\_foE}=N_{accepted\_foE}/N_{det\_E}  
\label{式5_11}
\end{align}
\end{linenomath}
\begin{linenomath}
\begin{align}
R_{accepted\_h'E}=N_{accepted\_h'E}/N_{det\_E}              
\label{式5_12}
\end{align}
\end{linenomath}    
为了对算法的整体正确率进行评价,考虑所有电离图的识别情况,可以根据公式~\ref{式5_13}、~\ref{式5_14},计算出E层参数$foE$、$h^{'}E$的可接受率($R_{all\_accepted\_foE}$、$R_{all\_accepted\_h'E}$),分别为86.32\%、78.43\%。
\begin{linenomath}
\begin{align}
R_{all\_accepted\_foE}=(N_{accepted\_foE}+N_{undet\_E})/(N_{E}+N_{NoE})    
\label{式5_13}
\end{align}
\end{linenomath}
\begin{linenomath}
\begin{align}
R_{all\_accepted\_h'E}=(N_{accepted\_h'E}+N_{undet\_E})/(N_{E}+N_{NoE})    
\label{式5_14}
\end{align}
\end{linenomath}
    
通过将数据集内的电离图用本文的自动度量方法得到的Es层描迹类型识别结果和人工度量得到的结果进行对比。如果电离图的自动度量方法检测到的Es层描迹个数和类型与人工度量结果完全相同,就认为该电离图的Es层描迹的类型识别是完全正确。本文算法在数据集内电离图Es层描迹的类型识别正确率为55.70\%。


%=============================================================================================================
\section{分析}
\label{5_2}
通过对实验结果进行分析可知,本文算法在描迹检测和参数度量方面取得了较高的正确率。本文的Es层描迹类型识别算法主要根据电离图人工度量经验以及人工度量知识提出的,实际上,人工度量知识也是根据早期电离层人工度量分析给出的,取决于人工经验具有很大的不确定性。本文算法对于常规的E区描迹具有较好的识别效果,但在部分电离图上还存在比较特殊的描迹,对于这类电离图,不同的度量者对同一张电离图的识别结果也存在一定的差异性。所以,特殊描迹电离图是降低Es层描迹识别正确率的主要原因。
















































\chapter{总结与展望}
\label{cha6}
本章主要对本文提出的垂测电离图E区描迹自动判读方法进行总结与展望。
%=============================================================================================================
\section{总结}
\label{6_1}

电离层探测结果的自动判读对电离层结构状态的研究提供大量的数据,同时可以克服人工度量费时、费力、度量结果存在主观差异等问题,而且可以为电离层相关研究提供实时的电离层监测数据。本文通过对大量电离图的人工度量经验进行总结,基于图像处理、图像分析方法,对垂测电离图E区描迹自动度量方法进行了研究,主要工作内容如下:
\begin{enumerate}
\item 电离层的相关知识及E区电离图的人工度量经验总结。本文对电离层形成机制、电离层分层结构和探测原理进行总结,为自动判读算法的提出奠定了基础;对垂测电离图参数度量标准进行介绍,并总结电离图E区描迹人工度量步骤,明确了研究意义并确定了电离图E区描迹自动度量流程。电离图的自动度量算法主要包括:电离图E区F区分割和电离图E区描迹自动判读。
\item电离图E区F区分割算法研究。将电离层垂直探测数据转换为垂测电离图,观察电离图像素灰度值分布特点,去除电离图上的噪声。利用水平投影积分法对去噪后的二值电离图进行电离图零区查找,提取零区特征结合电离图参数经验值,确定最优零区的中线作为E区F区的分割线。
\item电离图E区描迹自动判断算法研究。首先,本文的基于密集点查找的描迹自动检测算法,可以有效地克服传统的电离图预处理方法造成的描迹信息丢失现象;同时算法能够为具有扩散现象的描迹找到闭合的边界,方便于接下来的描迹提取工作;其次,本文提出了扩散密度的概念,对于检测到的描迹区域进行扩散程度评价,并结合其他特征对描迹的区域进行分类;再次,根据描迹区域的类型对电离图进行分类,排除复杂电离图和无描迹电离图;然后,根据描迹的形态特征对描迹进行分割、去除多次反射描迹、特征提取、类型识别与度量。在本文的算法中,创新性地将人工识别Es层描迹的经验转换为计算机量化电离图特征;同时针对E区描迹的多样性,设计了不同的描迹特征提取算法。
\item经过大量实验验证,本文算法对不同地区和不同时间段采集的电离图都有很好的适应性与发展前景,并取得较高精度。
\end{enumerate}

%=============================================================================================================
\section{展望}
\label{6_2}
电离层的形成机制和影响因素的复杂性导致垂直探测电离图的多样性与复杂性,对于特殊的频高图中,不同的度量者也会有不同的判读结果。通过实验验证,本文的电离图E区描迹自动判读算法是可行的,但在实际应用中,我们可以对很多方面进行完善:
\begin{enumerate}
\item本文的电离图E区描迹自动判读算法是基于人工度量方法提出的,目前对于人工度量流程的细化程度不够。比如在垂测电离图上的描迹不完整时,人工度量结果一般采用添加符号进行说明,但在本文算法中,对自动度量结果的细化程度不够,我们还需要对识别和度量进一步的细化(如添加特定符号来说明电离图的特殊情况等)。
\item本文所提出的垂测电离图E区描迹自动判读算法仅依靠人工度量经验,根据描迹的位置、形状、扩散状况等特征来判断描迹类型,而没有将电离层的物理模型和物理意义,融入到本文的算法中。在电离层的实际研究中,存在很多成熟的电离层参考模型,比如IRI模型。我们可以将现有的电离层参考模型与基于描迹特征的电离图识别算法相结合,来提高算法的准确率。
\item电离图E区描迹的种类多样性,同时同一类型描迹的大小、形状也都不尽相同,通过观察有限张电离图总结出的人工度量经验具有一定的局限性,并不能适用于所有的电离图。我们可以运用模式识别知识,将大量含有人工度量结果的电离图作为训练集,训练分类器,进而对未来探测的电离图进行类型识别与参数度量。
\item随着人工智能和云计算的发展,通过对大量电离图数据进行深度学习,形成分类器,依托云计算平台资源对大量电离图进行自动获取电离图参数,这样的方法将更加高效、智能。同时对于大量电离图的自动度量结果也可利用大数据技术进行分析。
\end{enumerate}



%%% 其它部分
\backmatter

% 本科生要这几个索引,研究生不要。选择性留下。
\makeatletter
\ifthu@bachelor
  % 插图索引
  \listoffigures
  % 表格索引
  \listoftables
  % 公式索引
  %\listofequations
\fi
\makeatother


% 参考文献
\bibliographystyle{thubib}
\bibliography{ref/refs}


% 致谢
%%% Local Variables:
%%% mode: latex
%%% TeX-master: "../main"
%%% End:

\begin{ack}
时光飞逝,研究生三年的时光将要结束。在海大生活、学习、科研的三年时间给我留下了很多美好的回忆。在此论文完成之际,向所有在过去三年硕士研究生生活中给予我的指导、关心和帮助的人表示最诚挚的感谢。

感谢我的导师郑海永教授。郑老师专注的科研精神、做事追求完美的态度给我树立了学习的榜样。同时感谢老师三年来对我的做人、做事、做科研的严格要求,让我认识到自己身上的很多不足,并让我养成了很多好的习惯。感谢我的师母冯丽颖对我的关心和照顾。

感谢我的老师姬光荣教授,感谢姬老师在三年研究生生活中对我的关心和照顾,同时老师平易近人的性格、积极乐观的生活态度、对工作无私奉献的精神都深深的影响了我。

感谢卢红光师哥、刘闯师哥、徐高峰师哥,在我的研究课题上给予我的帮助和启发。感谢赵红苗师姐和孙雪师姐在科研中对我的关心和帮助。感谢我的小伙伴邱欣欣、朱亚菲、战敏、荆纬、宋莹在研究生三年的生活中,带给我的快乐和美好的回忆。同时感谢信息学院南楼A202实验室所有的兄弟姐妹给我如家人般的温暖。同时感谢我的舍友杨洁和侯晓晴在生活中对我的关心和照顾。感谢我的姐姐姜晓玲对我的支持与鼓励。

感谢我的父母,感谢他们二十六年来对我的养育、教育和无私的爱。在以后的工作中我会继续努力不辜负他们的期望,也会多花时间陪在他们身边。

感谢青岛市科技发展计划项目“基于图像分析的垂测电离图自动判读研究”(批准号:13-1-4-223-jch)和中央高校基本科研业务费项目“电离层垂直探测频高图自动解释及度量”(批准号:201313011)的资助。


最后,感谢所有关心和帮助过我的人,祝愿你们幸福安康!
\end{ack}


% 附录
%\begin{appendix}
%%%% Local Variables: 
%%% mode: latex
%%% TeX-master: "../main"
%%% End: 

\chapter{}
\label{}

\chapter{}
\section{}

\chapter{}


%\end{appendix}

% 个人简历
\begin{resume}

  \resumeitem{个人简历}

  1990 年 7 月 27 日出生于 内蒙古 自治区 乌兰察布 市 白音察干镇。
  
  2009 年 9 月考入 赤峰 学院 物理与电子信息工程学院 应用物理学 专业,2013 年 7 月本科毕业并获得 理学 学士学位。
  
  2013 年 9 月考入 中国海洋大学 电子 系攻读 硕士 学位至今。

 
  \resumeitem{在学期间参加的研究项目} % 有就写,没有就删除
 
   青岛市科技发展计划项目“基于图像分析的垂测电离图自动判读研究”(批准号:13-1-4-223-jch)
   
   中央高校基本科研业务费项目“电离层垂直探测频高图自动解释及度量”(批准号:201313011)
 
  
\end{resume}

\end{document}
